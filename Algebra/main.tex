\documentclass[a4paper]{memoir}

\usepackage[ngerman]{babel}
\usepackage{bookmark}
\usepackage{amsmath}
\usepackage{amssymb}
\usepackage{amsthm}
\usepackage[T1]{fontenc}
\usepackage{makeidx}
\usepackage{enumitem}
\usepackage{mathtools}
\usepackage{upgreek}

\usepackage{microtype}
\usepackage{svg}
\usepackage{parskip}
\usepackage{hyperref} % Make TOC clickable

\usepackage{lipsum}
\usepackage{fancyhdr} % Heading customization
\usepackage{geometry} % Adjust page padding 
\usepackage{adjustbox}

\usepackage{xcolor}
\usepackage[most]{tcolorbox}

\definecolor{Black}{HTML}{292939}
\definecolor{CTheorem}{HTML}{FFFFFF}
\definecolor{CLemma}{HTML}{FDFFED}
\definecolor{CDefinition}{HTML}{EFF9F0}
\definecolor{DarkGray}{HTML}{6B6969}
\definecolor{ImportantBorder}{HTML}{F14747}


% Set up page layout
\geometry{
    a4paper,
    left=2cm,
    right=2cm,
    top=2cm,
    bottom=2cm
}
\linespread{1.1}
\makeindex

\pagestyle{fancy}
\fancyhf{} % Clear headre/footer
\fancyheadoffset[LE,RO]{0pt} % Adjust headsep for page number and title
\fancyhead[RE,LO]{}
\fancyhead[RE,RO]{\rightmark} % Add pape header title
\fancyhead[LE,LO]{\thepage}  % Add page number to the left
\renewcommand{\headrulewidth}{0pt} % Delete the line in the header 

% Define the \para command
\newcommand{\para}[2]{
    \clearpage % Start on a new page
    \thispagestyle{empty}% removes the top left page number and top right chapter name
    \begin{center} % Center the chapter title
        \vspace*{11em}
        \Huge   \bfseries \S #1 \quad  #2 % Chapter title with symbol and counter
    \end{center}
    \vspace{8em}
    \addcontentsline{toc}{chapter}{#1.\hspace{0.6em} #2} % Add chapter to table of contents
    \fancyhead[RE,RO]{#2} % Add pape header title
}

\newcommand{\nsec}[2]{
    \section*{\large  #1 \hspace{0.3em} #2}
    %\phantomsection
    \addcontentsline{toc}{section}{#1 \hspace{0.3em} #2}
}
\newcommand{\smalltitle}[2][]{
    %\phantomsection
    \subsubsection*{ #1 \hspace{0.2em} #2}
}
\newtcolorbox{ibox}[3][]{
	enhanced jigsaw,
	colback=#3,
	colframe=DarkGray,
	coltitle=Black,
	drop shadow,
	title={\textbf{#1\hspace{0.4em}#2}},
	%before skip = 1em,
	attach title to upper,
	after title={: \quad  },
}
\renewcommand{\d}[1]{\,\mathrm{d} #1}

\begin{document}
\pagecolor{White}
\color{Black}
\tableofcontents
\para{1}{Gruppen}
\begin{ibox}{1.1}{Gruppe}{CDefinition}
    \index{Gruppe}
    Eine nicht leere Menge $ G $ mit einer Verknüpfung $ \; \circ \; (\circ \; G \times G \rightarrow G )$ heißt Gruppe falls:
    \begin{description}
        \item[Assoziativität] $ (a  \circ b) \circ c = a  \circ (b  \circ  c) \; \; \forall \; a,b,c \in G $ 
        \item[Neutrales Element] Es gibt ein $ e \in G $ mit $ e  \circ a = a \circ e = a \; \forall \; a \in G $
        \item[Inverses Element] Zu jedem $ a \in G $ gibts ein $ a' \in G $, mit $ a \circ a' = a' \circ a = e $
    \end{description}
    
        Gilt zudem $ a \circ b = b \circ a \; \forall \; a,b \in G $ so nennen wir $ (G,\circ)\; \textit{abelsch} $

\end{ibox}

\smalltitle{Beispiele} 
\begin{itemize}
    \item  $ ( \mathbb{Z}, +), ( \mathbb{Q} \backslash \{0\}, \times), (V,+)$ für 
        Vektorraum V abelsche Gruppe.
    \item  $ ( \mathbb{Z}/_m \mathbb{Z}$ zyklische Gruppe, $ S_n\; $symmetrische Gruppe ( für $n \geq 3\; S_n  $ist nicht abelsch )
\end{itemize}

\smalltitle[1.3]{Bemerkung} 
Wir schreiben $ a^{-1}$ für das inverse Element und $ 1_G $ für das neutrale Element in $ G $.
\nsec{1.4}{Zykelgruppe Beispiel}
\index{Zykelgruppe}
    \indent Für eine Menge $ X \neq \emptyset $ definiert man: $ S_n = \{f:X \rightarrow X |\; bijektiv\}\; $ eine Gruppe bezuglich $ \circ $ mit $ id_X $ als neutrales Element (Umkehrabbildung = Inverses Element)\\
        Bei $ X={1, \dots , n} $ schreiben wir $ S_n $ statt $ S_x $ (symmetrische Gruppe von Grad $ n $) $ |S_n| = n \! $.
\smalltitle{Zykelschreibweise: }
Sei $ \pi \in S_n $, wir schreiben $ (a_1, \dots, a_n) $ für $ \pi $ falls \[ \pi(a_i) = \begin{cases}
            a_{i+1}, & \text{falls } 1 \leq i \leq r \\
            a_1, & \text{falls} i = r
        \end{cases}
          \]
          \[
        \pi(x) = \begin{cases}
            x, & \text{falls} x \notin {a_1, \dots , a_r}
        \end{cases}
          \]
Zum Beispiel:
$$ S_3 \ni (1\,2) = \begin{pmatrix}
    1 & 2 & 3 \\
    2 & 1 & 3

\end{pmatrix}
$$
$$ S_3 \ni (1\,2\,3) = \begin{pmatrix}
    1 & 2 & 3 \\
    2 & 3 & 1
\end{pmatrix}
$$
$ (1\,2\,3\,)(4\,5)\; $ist kein \textit{Zykel}, aber das Produkt zweir \textit{\textbf{disjunkter} Zykel} 
\par
\begin{ibox}{1.5}{Untergruppe}{CDefinition}
    \index{Untergruppe}
    Sei $ (G,\circ) $ eine Gruppe und $ U \subset G $. Diese Teilmenge $ U $ heißt Untergruppe (UG) von Gruppe G (geschrieben $ U \leq G $ , falls:
\begin{itemize}
    \item $ 1_G \in U $ 
    \item $ U^{-1} \in U \; \forall \; u \in U $ 
    \item $ u \circ u' \in U \; \forall \; u, u' \in U $ 
\end{itemize}
\end{ibox}

\smalltitle[1.6]{Bemerkung}
\begin{enumerate}
    \item $ U $ ist selbst eine Gruppe $ G \leq G \; \& \; \{1_G\} \leq G $ 
    \item Für eine Familie $ \{U_i\}_{i\in I} $ von Untergruppe von G ist $\bigcap_{i\in I}U_i$ eine Untergruppe von G
    \item $\bigcup_{i\in I}U_i$ ist \textbf{keine} Untergruppe
\end{enumerate}

\begin{ibox}{1.7}{Erzeugnis}{CDefinition}
    \index{Erzeugnis}\index{Erzeugte Untergruppe}
    Sei $ (G,\circ) $  eine Gruppe und $ S \subseteq G $ eine Menge. Weiterhin sei 
    $ \{S_i\} $ die Familie von aller möglichen Untergruppen von $ G $ .  Dann 
    heißt die Untergruppe $ \langle S \rangle  = \bigcap S_i $ 
    erzeugte Untergruppe oder das Erzeugnis von $ S $  
\end{ibox}

\smalltitle{Beispiel}
\begin{itemize}

    \item $ \langle \emptyset  \rangle \{1_G\} $
    \item $ \langle G \rangle = G $
    \item $ S_n = \langle Zykel \rangle = \langle \{ i,j | i \neq j ; \;  i,j \in \{1,\dots,n\}\} \rangle = \langle \{(i,i+1)|1 \leq i \leq n\} \rangle = \langle \{(1,2)(1,2,\dots ,n)\} \rangle $ . 
\end{itemize}
\vspace{2em}
        Sei $ (G, \circ) $ eine Gruppe und $ g \in G $. Dann $ \langle g \rangle := \langle \{g\} \rangle  $. Es gilt $ \langle g \rangle = \{g^i:i \in \mathbb{Z}\} $ 
            $$ g^i = \begin{cases}
                \underbrace{ g \cdot g \dots g}_{i\; mal}, & \text{ falls } i > 0 \\
                1_G, & \text{falls } i = 0 \\
                (g^{-i})^{-1},& \text{falls} i<0    
            \end{cases}
             $$ 
\begin{ibox}{1.8}{Ordnung}{CDefinition}
    Für eine Gruppe $ (G, \circ) \text{ mit } g \in G \text{ heißt } |G| $ die Ordnung von $ G \text{und} | \langle g \rangle |  $ die Ordnung von $ g $ . Wir schreiben $ ord(g) = | \langle g \rangle | $ . 
\end{ibox}
\smalltitle[1.9]{Beispiel}
\begin{enumerate}[label=\alph*)]
    \item Für $ G=S_4 $ gilt:
        $$ ord((1\,2))=2 $$
        $$ ord((1\,2\,3)) = 3 $$
        $$ ord((1\,2\,3\,4)) = 4 $$
        Für $ G=S_{10} $ gilt:
         $$ ord((1\,2\,3\,4\,5)(6\,7))=10 $$
         $$ ord(\underbrace{ (1\,2\,3\,4\,5)(1\,6)}_{disjunkt}) = ord((1\,6\,2\,3\,4\,5))= 6$$
     \item $ G $ heißt zyklisch, falls $ G= \langle g \rangle $ für eine $ g \in G $    
         $  $ 
          
\end{enumerate}
\vspace{1em}
\smalltitle{Alle Untergruppe von  $( \ensuremath{\mathbb{Z}}, + )  $}
\textbf{Behauptung: } Alle Untergruppe von $ \mathbb{Z}, +) $ sind in der Form $ m \mathbb{Z} $ 

\begin{proof}
    Sei $ U \leq \mathbb{Z} $ falls $ U = \{0_G\} $ (trivale Untergruppe)
    dann $ U = 0 \mathbb{Z} $ , ansonsten sei $ m $ minimal im $ U \cap ( \mathbb{N} \ 0) $ . Somit $ m \in U $ und $ m \mathbb{Z} \in U $ da $ m + m \in U, 2m + m \in U, -m \in U $. 
    \newline
    Sei $ n \in U/ m \mathbb{Z} $ . Dann $ n = am + r $ mit $ r \in {1,\dots, m-1}  $ (Kann nicht $ 0 $ sein, da $ n \in U/m \mathbb{Z} $ . Dann $ r = n - m \dot a = \underbrace{n}_{\in U} + \underbrace{(-ma)}_{\in m \mathbb{Z}} \in U $ . Wobei $ r < m $ . Wiederspruch zur Definition von $ m $ . Also $ U = m \mathbb{Z} $    
    
\end{proof}

\index{Gruppenhomomorphismus}
\index{Gruppenisomorphismus}
\begin{ibox}{1.10}{Gruppenhomomorphismus}{CDefinition}
    Seien $ (G, \cdot \,) $  und $ (H, * \,) $ Gruppen. Wir nenne $ \gamma : G 
    \to H $ Gruppenhomomorphismus falls
    $$ \gamma \left( g_1 \cdot g_2 \right) = \gamma(g_1) * \gamma(g_2) \quad \forall g_1, g_2  \in G $$
   $ G \cong H $ ($ G $ und $ H $ sind isomorph) falls eine bijektiven  Gruppenhomomorphismus gibt. 
\end{ibox}

\smalltitle[]{Beispiel}
Sei $ \gamma : ( \mathbb{Z}, + ) \to ( \mathbb{R}_{>0}, \circ),\quad x \mapsto e^{x} $ ist  ein Gruppenhomomorphismus
\begin{ibox}{1.11}{Lemma}{CLemma}
    Sei $ \emptyset \neq S \subset G $ Dann  $ \langle S \rangle = \{ s_1, \dots, s_r |\; r \leq 1, s_i \in S \cup S^{-1} \} =: H $.
    Wobei $ A^{-1} := \{a^{-1} | \; a \in A \} $ für $ A, B \subset G $ 
    \newline
    $ AB = \{ab|\; a \in A, b \in B \}$ 
\end{ibox}

\begin{proof}
    Plan: $ H \leq G, S \subseteq H, \langle S \rangle = H $ 
    \begin{itemize}
        \item $ 1_G = s_1 s_{1}^{-1} \in H  $ für $ s_1 \in S $ 
        \item $ H \cdot H \subseteq H $, da $ (s_1, \dots , s_r) (s'_1, \dots s'_2 ) =  s_1, \dots , s_r s'_1, \dots s'_2 \in H$ 
        \item Sei $ h \in H  $  mit $  h = s_1 \dots s_r  $. Dann $ h^{-1} = (s_1, \dots, s_r)^{-1}. \underbrace{S_r^{-1}}_{\in S \cup S^{-1}} \dots \underbrace{S_1^{-1}}_{\in S \cup S^{-1}} \in H$. 
    \end{itemize}
        Insgesamt $ H \leq G \text{ und } S \subseteq H. $ 
        \newline
        Sei $ U $ jetzt eine Untergruppe mit $ S \subset U $. Dann $ H \subseteq U $ (da $ U $ ein Gruppe ist.) 
        $$ \langle S \rangle = \bigcap_{\substack {U \leq G \\ S \subset U}} U = H \cap H = H \implies 
        \langle S \rangle = H$$
\end{proof}
        
\smalltitle[]{Bemerkung}
Sei $ G = \langle g \rangle  $ (zyklisch) mit $ |G| < \inf  $. Gilt $ g^n = 1_G  $ für ein $ n \in \mathbb{Z} $, dann $ |G| \bigm| n $.

\begin{proof}
    Sei $ d \in \mathbb{N} $  mit $ g^d = \{1_G, g^1, g^2, \dots \} $. Division mit Rest liefert wieder $ n = a \cdot d + 0  $ . Somit $ d \bigm| n $   
\end{proof}

\begin{ibox}{1.12}{Satz}{CTheorem}
    Sei $ (G, \circ) $  eine Gruppe und $ U \leq G $ 

    \begin{enumerate}[label=\alph*)]
        \item Wir definieren $ \sim_u $ auf $ G $ 
            $$ a \sim_u b \iff a^{-1}b \in U $$
        \item Die Menge der Äquivalenzklassen $  G/U  $ ist Linksnebenklassen
           $$ \left\{ aU \;|\; a \in G \right\}  $$
            
        \item Ist $ T $ eine Vertretungssystem der Äquivalenzklassen, so gilt $ G = 
            \mathop{\dot{\bigcup}}\limits_{t \in T} t U$ 
        \item $ |a U| = |u| \quad \forall a \in G $ 

            
    \end{enumerate}

\end{ibox}

\smalltitle[]{Bemerkung}
\begin{enumerate}[label=\alph*)]
    \item Mit '$ \sim_{u} $' gegeben durch $ a \sim_{u}' b \iff ab^{-1} \in U $
        wird wieder eine Äquivalenzrelation auf $ G $  definiert mit $ Ut $ 
        (Rechtsnebenklassen) als Äquivalenzklassen.
    \item $ G / U $ ist die Menge der Linksnebenklassen und $ U\\G $ ist die Menge
        der Rechtsnebenklassen
    \item $ \left| G/U \right|  $ ist die Index von U in G
\end{enumerate}

\begin{ibox}{1.13}{Satz von Lagrange}{CTheorem}
    Für eine endliche Gruppe $ G $ mit $ U \leq G $  gitl $ \left| G \right| = 
    \left| U \right| \left| G/U \right| $ , insbesondere $ \left| U \right| 
    \mid \left| G \right| $ 
\end{ibox}
\begin{proof}
    Sei $ T = \left\{ t_1, \dots , t_{r} \right\}$eine Vertretersystem der
    Äquivalenzklasse, das heißt $ G = UtU \leadsto \left| G \right| = 
    \sum_{t \in T} \left| tU \right| = \sum_{t \in T} \left| U \right| = 
    \left| T \right| \left| U \right| $ 
\end{proof}

\smalltitle[]{Beispiel}
\begin{enumerate}[label=\alph*)]
    \item Ist $ U \leq S_3 $ , dann $ \left| U \right| \in 
        \left\{ 1,2,3,6 \right\}  $ 
    \item Sei $ g \in G $ , dann $ ord\left( g \right) = \left<g \right> \mid 
        \left| G \right| $ In $ S_10 $ gibt es keine Untergruppe der Ordunug 11
\end{enumerate}
\textit{}

\printindex
\end{document} 
