%\documentclass[a4paper]{memoir}
%
\usepackage[ngerman]{babel}
\usepackage{bookmark}
\usepackage{amsmath}
\usepackage{amssymb}
\usepackage{amsthm}
\usepackage[T1]{fontenc}
\usepackage{makeidx}
\usepackage{enumitem}
\usepackage{mathtools}
\usepackage{upgreek}

\usepackage{microtype}
\usepackage{svg}
\usepackage{parskip}
\usepackage{hyperref} % Make TOC clickable

\usepackage{lipsum}
\usepackage{fancyhdr} % Heading customization
\usepackage{geometry} % Adjust page padding 
\usepackage{adjustbox}

\usepackage{xcolor}
\usepackage[most]{tcolorbox}

\definecolor{White}{HTML}{F8F8F3}
\definecolor{Black}{HTML}{292939}
\definecolor{CTheorem}{HTML}{FFFFFF}
\definecolor{CLemma}{HTML}{FDFFED}
\definecolor{CDefinition}{HTML}{EFF9F0}
\definecolor{DarkGray}{HTML}{6B6969}
\definecolor{ImportantBorder}{HTML}{F14747}


% Set up page layout
\geometry{
    a4paper,
    left=2.5cm,
    right=2.5cm,
    top=2.5cm,
    bottom=2.5cm
}
\linespread{1.1}
\makeindex

\pagestyle{fancy}
\fancyhf{} % Clear headre/footer
\fancyheadoffset[LE,RO]{0pt} % Adjust headsep for page number and title
\fancyhead[RE,LO]{}
\fancyhead[RE,RO]{\rightmark} % Add pape header title
\fancyhead[LE,LO]{\thepage}  % Add page number to the left
\renewcommand{\headrulewidth}{0pt} % Delete the line in the header 

% Define the \para command
\newcommand{\para}[2]{
    \clearpage % Start on a new page
    \thispagestyle{empty}% removes the top left page number and top right chapter name
    \begin{center} % Center the chapter title
        \vspace*{11em}
        \Huge   \bfseries \S #1 \quad  #2 % Chapter title with symbol and counter
    \end{center}
    \vspace{11em}
    \addcontentsline{toc}{chapter}{#1.\hspace{0.6em} #2} % Add chapter to table of contents
    \fancyhead[RE,RO]{#2} % Add pape header title
}

\newcommand{\nsec}[2]{
    \section*{\large  #1 \hspace{0.3em} #2}
    %\phantomsection
    \addcontentsline{toc}{section}{#1 \hspace{0.3em} #2}
}
\newcommand{\smalltitle}[2][]{
    %\phantomsection
    \subsubsection*{ #1 \hspace{0.2em} #2}
}
% Theorem env
%\newenvironment{ibox}[3]{
%    \phantomsection
%    \addcontentsline{toc}{section}{#1 \hspace{0.3em} #2}
%    \vspace{1.5em}
%    \begin{tcolorbox}[
%        enhanced jigsaw,
%        colback=#3 ,
%        colframe=DarkGray,
%        drop shadow, 
%		attach boxed title to top center={yshift=-2mm},
%        before upper={\vspace{0.5em}},
%        after upper={\vspace{0.5em}},
%        \textbf{ \large #1 \hspace{0.1em} #2\hspace{0.4em}}
%	]}{%
%    \end{tcolorbox}
%}
\newtcolorbox{ibox}[3][]{
	enhanced jigsaw,
	colback=#3,
	colframe=DarkGray,
	coltitle=Black,
	drop shadow,
	title={\textbf{ \large  #1\hspace{0.4em}#2}},
	before skip = 2.7em,
	attach title to upper,
	after title={:\quad \vspace{0.5em}\ },
	%toc = {#1 \quad* #2}{section}
	%IfValueTF={#1}{\addcontentsline{toc}{section}{#1 \hspace{0.3em}#2}}{}
	code={\addcontentsline{toc}{section}{#1 \hspace{0.3em}#2}}
}


%\begin{document}
\para{12}{Riemannsches Integral in $ \mathbb{R}^n $ }
\begin{ibox}[]{Definitionen}{CDefinition}
	Eine Menge $ Q = [a_1,b_1] \times \cdots \times [a_n, b_n] = \left\{ x = \left( x_1 , \cdots,  x_n \right) 
	| a_i \leq x_i \leq b_i, i = 1 , \cdots,  n \right\}  $, wobei $ a_i, b_i \in \mathbb{R}, a_i < b_i $ heißt $ n $-dimensionaler 
	Quader. Die Zahl $ | Q | = (b_1-a_1) \cdots (b_n-a_n) $ heißt \textit{Volumen} des Quaders. Die Zahl 
	$$ \delta (Q) = \sqrt{(b_1-a_1)^{2} + \cdots + (b_n-a_n)^{2} } $$
	heißt \textit{Durchmesser} des Quaders. Eine Zerlegung des gegebenen Quaders $ Q $ bauen wir aus Zerlegungen der Intervalle $ [a_i,b_i] $ 
	mit $ a_i = x_i^{(0)} < \cdots < x_i^{(p_i)} = b_i \; \left(  i = 1 , \cdots,  n \right)$. Daraus werden alle möglichen Teilquader der 
	Form $ \left[ x^{(k_1)}, x^{(k+1) } \right] \times \cdots \times \left[ x^{(kn) }, x^{(kn+1) } \right]  $ gebildet. Diese werden in 
	irgendeiner Reihenfolge durchnummeriert und mit $ Q_1 ,\cdots , Q_{m} $ bezeichnet. Die Menge $ Z = \left\{ Q_1, , \cdots,  Q_{m} \right\}$ 
	Die Menge $ Z = \left\{ Q_1 , \cdots,  Q_{m} \right\}  $ heißt \textit{Zerlegung von $ Q $ } und
	$$
	\|Z \| := max \left\{ \delta (Q_k) : k = 1 , \cdots,  m \right\} 
	$$ heißt die \textit{Feinheit der Zerlegung}. Für jedes $ \xi^{(k) } \in Q_k $ gilt dann mit $ m^{(k) } = \inf \left\{ f (x) | x \in Q_k \right\} , M^{(k) } = \sup \left\{ 
	f (x) | x \in Q_k \right\}  $ die Ungleichhungskette $ m^{(k) } \geq f (\xi^{(k)}) \geq M ^{(k) } $. Wir bilden die Obersumme und die Untersumme wie folgt
	$$
	\overline{S_{z}} (f) : = \sum_{k=1}^{m}  M ^{(k)} \left| Q_k \right| \qquad \qquad 
	\underline{S_{z}} (f) : = \sum_{k=1}^{m}  m ^{(k)} \left| Q_k \right| 
	$$ und die Riemann'sche Summe $ \xi = \left( \xi^{(1) } , \cdots,  \xi^{(m)} \right)  $ 
	$$
 \sum_{k=1}^{m} f \left( \xi ^{(k)} \right) \left| Q_k \right| 
$$
Offenbar gilt: $ \underline{S}_z \leq S_z (\xi,f) \leq \overline{S_{z}} (f)  $. Dann nennt man $ \overline{S} (f) = \inf \left\{ \overline{S_{z}}
(f)  : Z \text{ Zerlegung}\right\}  $ \textit{Oberintegral} und $ \underline{S} (f) = \sup \left\{ \underline{S_{z}}
(f)  : Z \text{ Zerlegung}\right\}  $ \textit{Unterintegral} 	
\end{ibox}

\begin{ibox}[]{Definition}{CDefinition}
	$ f : Q = [a_1,b_1] \times \cdots \times [a_n,b_n] \to \mathbb{R}$ sei beschränkt. Dann heißt $ f $ auf $ Q $ \textit{Riemann-integrierbar}, falls
$ \underline{S} (f) = \overline{S} (f)$, und man schreibt 
$$ \int_{ Q } f (x) \, \mathrm{d}x := \underline{S} (f) = \overline{S} (f) $$
\end{ibox}

\begin{ibox}[]{Definition}{CDefinition}
    Sei $ f : B \to \mathbb{R} $ beschränkt und $ B \subset \mathbb{R}^n $ kompakt, d.h abgeschlossen und beschränkt. Außerdem sei
		$ Q \subset \mathbb{R}^n $ ein Quader, der $ B $ enthält. Wir setzen
		$$\tilde{f} = \begin{cases}
			f (x) & \text{ falls } x \in B\\
			0 & \text{ falls } x \in Q\setminus B
		\end{cases}$$
		und definieren, falls $ \tilde{f} $ auf $ Q $ integrierbar ist, $ \int_{ Q } f (x) \d{x} = \int_{ Q } \tilde{f} (x) \d{x} $.
		Wir nennen $ f $, \textit{integrierbar auf $ B $ }, wenn $ \tilde{f} $ auf $ Q $ integrierbar ist.
		Eine kompakte Menge $ B \subset \mathbb{R}^n $ heißt \textit{(Jordan)-messbar}, wenn das Integral $ \int_{ B } \d{x} $ 
		existiert. $ |B| = \int_{ B } 1 \d{x} $ heißt dann \textit{Volumen von $ B $ } 
\end{ibox}
\begin{ibox}[53]{Satz}{CTheorem}
    $ f: Q \to \mathbb{R} $ ist genau dann integrierbar, wenn eine Zahl $ s \in \mathbb{R} $ existiert, sodass zu jedem 
		$ \varepsilon > 0 \text{ ein } \delta > 0  $ gibt derart, dass für jede Zerlegung $ z \text{ von } Q $ mit Feinheit 
$ \|z \| < \delta $ und jede Wahl von Stützstellen $ \xi \text{ zu } z $ gilt: $ \left| S - S_{z} \left( \xi,f \right)  \right| <
\varepsilon$. In diesem Fall ist dann $ S = \int_{ Q } f (x) \d{x} $ 
\end{ibox}
\begin{ibox}[54]{Satz (Fubini)}{CTheorem}
	Es sei $ f :Q \to \mathbb{R} $ eine integrierbare Funktion auf dem Quader $ Q = [a , b] \times [c,b] $. Existieren die Integrale 
	$ F (x_1) = \int_{ c }^{ d } f (x_1,x_2) \d{x_2} \forall x \in [a,b] $  und $ G (x_2) = \int_{ a }^{ b } f (x_1,x_2) \d{x_1}
	\forall x_2 \in [c,d]$. So folgt 
	$$ \int_{ Q } f (x) \d{x} = \int_{ a }^{ b } \left( \int_{ c }^{ d } f (x_1,x_2) \d{x_2} \right) \d{x_1} = 
	\int_{ c }^{ d } \left( \int_{ a }^{ b } f (x_1,x_2) \d{x_1} \right) \d{x_2}$$
	Analoges gilt für $ n $-dimensionale Integrale für $ Q = [a_1,b_1] \times \cdots \times [a_n,b_n] $ falls die auftreffenden 
	Integrale Existieren
	$$ \int_{ Q } f (x) \d{x} = \int_{ a_1 }^{ b_1 } \left( \int_{ a_2 }^{ b_2 } \left( \cdots \left( \int_{ a_n }^{ b_n } 
	f \left( x_1, \cdots , x_n  \right)\d{x_n}\right)\cdots\right)\d{x_2}\right)\d{x_1}$$
bzw. mit Vertauschung auf der rechten Seite.	
\end{ibox}
\begin{ibox}[55]{Satz}{CTheorem}
	Seien $ g, h \in C^{0} ([a,b]), g \leq h  $ und sie $ B = \left\{  \left( x_1, x_2 \right) | a \leq x_1 \leq b, g (x_1)
	\leq x_2 \leq h (x_1)\right\}  $. Ist $ f : B \to \mathbb{R} $ stetig, so gilt 
	$$ \int_{ B } f (x) \d{x} = \int_{ a }^{ b } \left( \int_{ g (x_1) }^{ h (x_1) } f (x_1,x_2) \d{x_2} \right) \d{x_1} $$
\end{ibox}

\smalltitle[]{Bemerkung}
Unter geeigneten Voraussetzungen gilt für den Normalbereich $ B  = \{ (x_1, \cdots,  x_n) : a \leq x_1 \leq b$,\\ $  
g_2 (x_1) \leq x_2 \leq h_2 (x_1), g_3 (x_1,x_2) \leq x_3 \leq h_3 (x_1,x_2) , \cdots, \; g_n ( x_1, \cdots,  x_n) \leq x_n \leq
h_n (x_1, \cdots,  x_{n-1})\} $ 
$$ \int_{ B } f (x) \d{x} = \int_{ a }^{ b } \left( \int_{ g_2 (x_1) }^{ h_2 (x_1)} \left( \int_{ h_3 (x_1,x_2) }^{ g_2 (x_1,x_2) }
\cdots \left( \int_{ g_n (x_1, \cdots, x_n)  }^{ h_n (x_1, \cdots, x_n)}  f \left( x_1, \cdots, x_n \right) \d{x_n} \right) 
\cdots \d{x_3}\right)\d{x_2}\right) \d{x_1}$$
\smalltitle[]{Folgerung}
Für $ B = \left\{ x \in \mathbb{R}^{n+1} : \left( x_1 , \cdots,  x_n \right) \in B^{*} \right\}  $ ist $ |B| = \int_{ B^{*}}
f (x_1, \cdots,  x_n) \d{x_1}\cdots\d{x_1} $ das Volumen zwischen der Ebene $ x_{n+1} = 0 $ und der durch $ f $ gegebenen Fläche.

\begin{ibox}[56]{Satz}{CTheorem}
    \begin{enumerate}[label=\alph*)]
    	\item Jede Stetige Funktion auf einer messbaren kompakten Menge $ B $ ist integrierbar.
			\item Sind $ f, g $ auf $ B $ integrierbar und ist $ c \in \mathbb{R} $, so sind $ f + g $ und $ c f $ integrierbar
				und es gilt: 
				$$ \int_{ B } (f+g) \d{x} =  \int_{ B } f \d{x} + \int_{ B } g \d{x} \qquad \qquad \int_{ B } c f \d{x} = 
			c	\int_{ B } f \d{x} $$
		\item Die kompakte Menge $ B $ sie zerlegt in kompakte Teilbereiche $ B_1 , \cdots,  B_m $ das heißt $ B = B_1 \cup B_2 \cup 
			\cdots \cup B_m$ und $ mathring{B_i} \cap \mathring{B_j} = \emptyset  $. Ist $ f $ auf jedem $ B_i $ integrierbar, so ist
			$ f $ auch auf $ B $ integrierbar und
			$$
			\int_{ B } f \d{x} = \int_{ B_1 } f \d{x} +  \cdots + \int_{ B_m }f \d{x}
			$$
		\item Ist auf $ B $ integrierbar, so ist es auch $ |f| \text{ und }  \left| \int_{ B } f \d{x}  \right| \leq \int_{ B }
			|f| \d{x}$ 
    \end{enumerate}
\end{ibox}
\begin{ibox}[]{Definition}{CDefinition}
    $ E \subset \mathbb{R}^n $ heißt Tordanische Nullmenge in $ \mathbb{R}^n $, wenn es zu jedem $ \varepsilon > 0  $ endlich
		viele Quader $ Q_k $, für $ k = 1, 2 , \cdots \text{ und } N = N (\varepsilon) $ gibt, so dass $ E \subset \cup_{k=1}^{N} Q_k
		\text{ und }  \sum_{k=1}^{N}|Q_k| < \varepsilon$. \\
		 Nimmt man unendliche viele Quader $ Q_k $ anstelle für endliche viele, bekommt man Lebegsche Nullmenge.
\end{ibox}
%\end{document}

