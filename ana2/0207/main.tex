\documentclass[a4paper]{memoir}

\usepackage[ngerman]{babel}
\usepackage{bookmark}
\usepackage{amsmath}
\usepackage{amssymb}
\usepackage{amsthm}
\usepackage[T1]{fontenc}
\usepackage{makeidx}
\usepackage{enumitem}
\usepackage{mathtools}
\usepackage{upgreek}

\usepackage{microtype}
\usepackage{svg}
\usepackage{parskip}
\usepackage{hyperref} % Make TOC clickable

\usepackage{lipsum}
\usepackage{fancyhdr} % Heading customization
\usepackage{geometry} % Adjust page padding 
\usepackage{adjustbox}

\usepackage{xcolor}
\usepackage[most]{tcolorbox}

\definecolor{White}{HTML}{F8F8F3}
\definecolor{Black}{HTML}{292939}
\definecolor{CTheorem}{HTML}{FFFFFF}
\definecolor{CLemma}{HTML}{FDFFED}
\definecolor{CDefinition}{HTML}{EFF9F0}
\definecolor{DarkGray}{HTML}{6B6969}
\definecolor{ImportantBorder}{HTML}{F14747}


% Set up page layout
\geometry{
    a4paper,
    left=2.5cm,
    right=2.5cm,
    top=2.5cm,
    bottom=2.5cm
}
\linespread{1.1}
\makeindex

\pagestyle{fancy}
\fancyhf{} % Clear headre/footer
\fancyheadoffset[LE,RO]{0pt} % Adjust headsep for page number and title
\fancyhead[RE,LO]{}
\fancyhead[RE,RO]{\rightmark} % Add pape header title
\fancyhead[LE,LO]{\thepage}  % Add page number to the left
\renewcommand{\headrulewidth}{0pt} % Delete the line in the header 

% Define the \para command
\newcommand{\para}[2]{
    \clearpage % Start on a new page
    \thispagestyle{empty}% removes the top left page number and top right chapter name
    \begin{center} % Center the chapter title
        \vspace*{11em}
        \Huge   \bfseries \S #1 \quad  #2 % Chapter title with symbol and counter
    \end{center}
    \vspace{11em}
    \addcontentsline{toc}{chapter}{#1.\hspace{0.6em} #2} % Add chapter to table of contents
    \fancyhead[RE,RO]{#2} % Add pape header title
}

\newcommand{\nsec}[2]{
    \section*{\large  #1 \hspace{0.3em} #2}
    %\phantomsection
    \addcontentsline{toc}{section}{#1 \hspace{0.3em} #2}
}
\newcommand{\smalltitle}[2][]{
    %\phantomsection
    \subsubsection*{ #1 \hspace{0.2em} #2}
}
% Theorem env
%\newenvironment{ibox}[3]{
%    \phantomsection
%    \addcontentsline{toc}{section}{#1 \hspace{0.3em} #2}
%    \vspace{1.5em}
%    \begin{tcolorbox}[
%        enhanced jigsaw,
%        colback=#3 ,
%        colframe=DarkGray,
%        drop shadow, 
%		attach boxed title to top center={yshift=-2mm},
%        before upper={\vspace{0.5em}},
%        after upper={\vspace{0.5em}},
%        \textbf{ \large #1 \hspace{0.1em} #2\hspace{0.4em}}
%	]}{%
%    \end{tcolorbox}
%}
\newtcolorbox{ibox}[3][]{
	enhanced jigsaw,
	colback=#3,
	colframe=DarkGray,
	coltitle=Black,
	drop shadow,
	title={\textbf{ \large  #1\hspace{0.4em}#2}},
	before skip = 2.7em,
	attach title to upper,
	after title={:\quad \vspace{0.5em}\ },
	%toc = {#1 \quad* #2}{section}
	%IfValueTF={#1}{\addcontentsline{toc}{section}{#1 \hspace{0.3em}#2}}{}
	code={\addcontentsline{toc}{section}{#1 \hspace{0.3em}#2}}
}


\begin{document}
\para{12}{Riemannsches Integral in $ \mathbb{R}^n $ }
\begin{ibox}[]{Definitionen}{CDefinition}
	Eine Menge $ Q = [a_1,b_1] \times \cdots \times [a_n, b_n] = \left\{ x = \left( x_1 , \cdots,  x_n \right) 
	| a_i \leq x_i \leq b_i, i = 1 , \cdots,  n \right\}  $, wobei $ a_i, b_i \in \mathbb{R}, a_i < b_i $ heißt $ n $-dimensionaler 
	Quader. Die Zahl $ | Q | = (b_1-a_1) \cdots (b_n-a_n) $ heißt \textit{Volumen} des Quaders. Die Zahl 
	$$ \delta (Q) = \sqrt{(b_1-a_1)^{2} + \cdots + (b_n-a_n)^{2} } $$
	heißt \textit{Durchmesser} des Quaders. Eine Zerlegung des gegebenen Quaders $ Q $ bauen wir aus Zerlegungen der Intervalle $ [a_i,b_i] $ 
	mit $ a_i = x_i^{(0)} < \cdots < x_i^{(p_i)} = b_i \; \left(  i = 1 , \cdots,  n \right)$. Daraus werden alle möglichen Teilquader der 
	Form $ \left[ x^{(k_1) }, x^{(k+1) } \right] \times \cdots \times \left[ x^{(kn) }, x^{(kn+1) } \right]  $ gebildet. Diese werden in 
	irgendeiner Reihenfolge durchnummeriert und mit $ Q_1 ,\cdots , Q_{m} $ bezeichnet. Die Menge $ Z = \left\{ Q_1, , \cdots,  Q_{m} \right\}$ 
	Die Menge $ Z = \left\{ Q_1 , \cdots,  Q_{m} \right\}  $ heißt \textit{Zerlegung von $ Q $ } und
	$$
	\|Z \| := max \left\{ \delta (Q_k) : k = 1 , \cdots,  m \right\} 
	$$ heißt die \textit{Feinheit der Zerlegung}. Für jedes 
\end{ibox}

\end{document}

