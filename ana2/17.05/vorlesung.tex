%\documentclass[a4paper]{memoir}

%
\usepackage[ngerman]{babel}
\usepackage{bookmark}
\usepackage{amsmath}
\usepackage{amssymb}
\usepackage{amsthm}
\usepackage[T1]{fontenc}
\usepackage{makeidx}
\usepackage{enumitem}
\usepackage{mathtools}
\usepackage{upgreek}

\usepackage{microtype}
\usepackage{svg}
\usepackage{parskip}
\usepackage{hyperref} % Make TOC clickable

\usepackage{lipsum}
\usepackage{fancyhdr} % Heading customization
\usepackage{geometry} % Adjust page padding 
\usepackage{adjustbox}

\usepackage{xcolor}
\usepackage[most]{tcolorbox}

\definecolor{Black}{HTML}{292939}
\definecolor{CTheorem}{HTML}{FFFFFF}
\definecolor{CLemma}{HTML}{FDFFED}
\definecolor{CDefinition}{HTML}{EFF9F0}
\definecolor{DarkGray}{HTML}{6B6969}
\definecolor{ImportantBorder}{HTML}{F14747}


% Set up page layout
\geometry{
    a4paper,
    left=2.5cm,
    right=2.5cm,
    top=2.5cm,
    bottom=2.5cm
}
\linespread{1.1}
\makeindex

\pagestyle{fancy}
\fancyhf{} % Clear headre/footer
\fancyheadoffset[LE,RO]{0pt} % Adjust headsep for page number and title
\fancyhead[RE,LO]{}
\fancyhead[RE,RO]{\rightmark} % Add pape header title
\fancyhead[LE,LO]{\thepage}  % Add page number to the left
\renewcommand{\headrulewidth}{0pt} % Delete the line in the header 

% Define the \para command
\newcommand{\para}[2]{
    \clearpage % Start on a new page
    \thispagestyle{empty}% removes the top left page number and top right chapter name
    \begin{center} % Center the chapter title
        \vspace*{11em}
        \Huge   \bfseries \S #1 \quad  #2 % Chapter title with symbol and counter
    \end{center}
    \vspace{8em}
    \addcontentsline{toc}{chapter}{#1.\hspace{0.6em} #2} % Add chapter to table of contents
    \fancyhead[RE,RO]{#2} % Add pape header title
}

\newcommand{\nsec}[2]{
    \section*{\large  #1 \hspace{0.3em} #2}
    %\phantomsection
    \addcontentsline{toc}{section}{#1 \hspace{0.3em} #2}
}
\newcommand{\smalltitle}[2][]{
    %\phantomsection
    \subsubsection*{ #1 \hspace{0.2em} #2}
}
\newtcolorbox{ibox}[3][]{
	enhanced jigsaw,
	colback=#3,
	colframe=DarkGray,
	coltitle=Black,
	drop shadow,
	title={\textbf{ \large  #1\hspace{0.4em}#2}},
	before skip = 2.7em,
	attach title to upper,
	after title={:\quad \vspace{0.5em}\ },
}
\renewcommand{\d}[1]{\,\mathrm{d} #1}


%\begin{document}
\para{6}{Totale Differezierbarkeit}
\begin{ibox}[]{Definition}{CDefinition}
    Sei $ U \subset \mathbb{R}^n  $ offen, $ x_0 = \left( x_{0_{1}} \cdots x_{o_{n}}  \right) \in U $ und 
	$ f : U \to \mathbb{R}  $ eine Funktion auf $ U $. $ f $ heißt \textit{(total) differenzierbar in $ x_0 $ }, 
	wenn es reele Zahlen $ a_1, \cdots , a_{n} \in R $ und eine Funktion $ \varphi : U \to \mathbb{R}  $ mit 
$ \lim_{x \to x_0} \frac{\varphi (x)}{|x -x_0|}  = 0 $ gibt, dass
	$$ f(x) = f(x_0) + \sum_{i=1}^{n}a_{i}(x_{i}-x_{0_{i}}) + \varphi(x) \;
	\forall x=\left( x_1, \cdots , x_{n} \right) \in U  $$ 
	gilt. Man bezeichnet den Vektor $ A:=\left( a_1, \cdots, a_{n} \right) $ als \textit{Differential} von $ f \text{ in } x $ 
\end{ibox}
\begin{ibox}[30]{Satz}{CTheorem}
    Seien $ U \subset \mathbb{R}^n  $ offen, $ x_0 \in U $ sei ferner $ f: U \to \mathbb{R}  $ total differenz
	in $ x_0 $ und $ A = \left( a_1, \cdots, a_{n} \right)  $ wie in der letzen Definition. Dann gilt:
	\begin{enumerate}[label=\alph*)]
		\item $ f $ ist stetig in $ x_0 $ .
		\item $ f $ ist partiell differenzierbar in $ x_0 $ und es gilt $ \frac{\partial f}{\partial x_{i}} (x_0) = a_{i} 
			 \; \forall i = 1, \cdots, n $ 
	\end{enumerate}
\end{ibox}
%\begin{proof}
%	\begin{enumerate}[label=\alph*)]
%		\item $ f(x)= f\left( x_0 \right) + \sum_{i=1}^{n}a_{i} (x_{i}-x_{0_{i}}) + \varphi(x) \xRightarrow{x \to x_0}
%			f(x) \to f(x_0) \implies f \text{ ist in } x_0 $ stetig. 
%		\item 
%			\begin{align*}
%				$ \frac{\partial f}{\partial x_{i}} (x_0) = \lim_{x_{i}\to x_{0_{i}}} \frac{a}{b} $ 
%		    \end{align*}
%	\end{enumerate}
%\end{proof}
\begin{ibox}[31]{Satz}{CTheorem}
    Sei $ U \subset \mathbb{R}^n  $ offen, die Funktion $ f: U \to \mathbb{R}  $ sei partiell differenzierbar auf 
	$ U $, und alle partielle Ableitungen $ D_i \left( \text{ oder auch }  \frac{\partial f}{\partial x_i} \right)  $,
	 $ 1 \leq i \leq n $, seien stetig in $ x_0 \in U $. Dann ist $ f $ in 
	$ x_0 $ total differenzierbar.
\end{ibox}

\smalltitle[]{Korollar}
Sei $ U \subset \mathbb{R}^n  $ offen, $ f:U\to \mathbb{R}  $ eine Funktion auf U und $ x_0 $. Dann gilt:
\begin{enumerate}[label=\alph*)]
	\item Ist $ f $ stetig partiell differenzierbar in $ x_0 $, so ist $ f $ an de stelle $ x_0 $ stetig.
	\item Ist $ f \; k$-mal partiell differenzierbar auf $ u $ und sind alle partiell Ableitungen 
		$ k $-te Ordnung von $ f $ stetig in $ x_0 $, so sind alle partielle Ableitungen der Ordnungen 
		$ \varphi , 0 \leq \varphi \leq k $ ebenfalls stetig in $ x_0 $  
\end{enumerate}

\begin{ibox}[]{Definition}{CDefinition}
    Sei $ U \subset \mathbb{R}^n  $ offen, $ x_0 = \left( x_{0_{1}}, \cdots , x_{0_{n}} \right) \in  U $ und 
	$ f: U \to \mathbb{R}^{n}  $ eine Abbildung auf $ U $, $ f $ heißt (total) differenzierbar in $ x_0 $, wenn es eine 
	Matrix $ A = \left( a_{ij} \right)_{i = 1, \cdots,m ; \; j = 1, \cdots, n}  $ und eine Abbildung 
	$ \varphi : U \to \mathbb{R}^m   $ mit $ \lim_{x \to 0} \frac{|\varphi(x)|}{\left| x - x_0 \right| }  = 0 $ gibt,
	sodass 
	$$ f(x) = f(x_0) + A(x-x_0) + \varphi (x) \; \forall x = \left( x_1, \cdots, x_{n} \right) \in U $$
	gilt. Der Deutlichkeit halber schreiben wir die obige Gleichung einmal komponentweise auf:
	\\
	Ist $ f = \left( f_1, \cdots, f_{m} \right)  $ mit $ f_{i}: U \to \mathbb{R}, \; \forall i = 1, \cdots, m $
	und entsprechend $ \varphi = \left( \varphi_{1}, \cdots \varphi_{m} \right)  $ mit $ \varphi_{i}: U \to \mathbb{R} 
	\; \forall i = 1, \cdots, m$ so ist die Bedingung $ \lim_{x \to x_0} \frac{| \varphi (x)|}{|x - x_0|} = 0 $ 
	gelichdeutend mit $ \lim_{x \to x_0} \frac{\varphi_i (x)}{|x - x_0|} = 0 \; \forall i = 1, \cdots, m $ und 
	die Gleichung bedeutet ausfrülich, dass  $ \forall x = (x_1, \cdots, x_{n}) \in  U	 $ 	
	$$ 
		\begin{pmatrix}
			f_1(x) \\ \vdots \\ f_{m}(x) 
		\end{pmatrix} = \begin{pmatrix}
			f_1(x_{0}) \\ \vdots \\ f_{m}(x_0)  
		\end{pmatrix}+
		\begin{pmatrix}
			a_{11} &\dots &a_{1n}\\
			\vdots & & \vdots \\
			a_{m_1} & \dots	& a_{mn}
		\end{pmatrix}
		\begin{pmatrix}
			x_1 - x_{0_{1}}\\
			\vdots \\
			x_{n} - x_{0_{n}}
		\end{pmatrix} + 
		\begin{pmatrix}
			\varphi_1(x) \\
			\vdots \\
			\varphi_{m}(x)
		\end{pmatrix}
	$$
\end{ibox}
%\end{document}
