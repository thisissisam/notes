%\documentclass[12pt, a4paper]{memoir}
	
%
\usepackage[ngerman]{babel}
\usepackage{bookmark}
\usepackage{amsmath}
\usepackage{amssymb}
\usepackage{amsthm}
\usepackage[T1]{fontenc}
\usepackage{makeidx}
\usepackage{enumitem}
\usepackage{mathtools}
\usepackage{upgreek}

\usepackage{microtype}
\usepackage{svg}
\usepackage{parskip}
\usepackage{hyperref} % Make TOC clickable

\usepackage{lipsum}
\usepackage{fancyhdr} % Heading customization
\usepackage{geometry} % Adjust page padding 
\usepackage{adjustbox}

\usepackage{xcolor}
\usepackage[most]{tcolorbox}

\definecolor{White}{HTML}{F8F8F3}
\definecolor{Black}{HTML}{292939}
\definecolor{CTheorem}{HTML}{FFFFFF}
\definecolor{CLemma}{HTML}{FDFFED}
\definecolor{CDefinition}{HTML}{EFF9F0}
\definecolor{DarkGray}{HTML}{6B6969}
\definecolor{ImportantBorder}{HTML}{F14747}


% Set up page layout
\geometry{
    a4paper,
    left=2.5cm,
    right=2.5cm,
    top=2.5cm,
    bottom=2.5cm
}
\linespread{1.1}
\makeindex

\pagestyle{fancy}
\fancyhf{} % Clear headre/footer
\fancyheadoffset[LE,RO]{0pt} % Adjust headsep for page number and title
\fancyhead[RE,LO]{}
\fancyhead[RE,RO]{\rightmark} % Add pape header title
\fancyhead[LE,LO]{\thepage}  % Add page number to the left
\renewcommand{\headrulewidth}{0pt} % Delete the line in the header 

% Define the \para command
\newcommand{\para}[2]{
    \clearpage % Start on a new page
    \thispagestyle{empty}% removes the top left page number and top right chapter name
    \begin{center} % Center the chapter title
        \vspace*{11em}
        \Huge   \bfseries \S #1 \quad  #2 % Chapter title with symbol and counter
    \end{center}
    \vspace{11em}
    \addcontentsline{toc}{chapter}{#1.\hspace{0.6em} #2} % Add chapter to table of contents
    \fancyhead[RE,RO]{#2} % Add pape header title
}

\newcommand{\nsec}[2]{
    \section*{\large  #1 \hspace{0.3em} #2}
    %\phantomsection
    \addcontentsline{toc}{section}{#1 \hspace{0.3em} #2}
}
\newcommand{\smalltitle}[2][]{
    %\phantomsection
    \subsubsection*{ #1 \hspace{0.2em} #2}
}
% Theorem env
%\newenvironment{ibox}[3]{
%    \phantomsection
%    \addcontentsline{toc}{section}{#1 \hspace{0.3em} #2}
%    \vspace{1.5em}
%    \begin{tcolorbox}[
%        enhanced jigsaw,
%        colback=#3 ,
%        colframe=DarkGray,
%        drop shadow, 
%		attach boxed title to top center={yshift=-2mm},
%        before upper={\vspace{0.5em}},
%        after upper={\vspace{0.5em}},
%        \textbf{ \large #1 \hspace{0.1em} #2\hspace{0.4em}}
%	]}{%
%    \end{tcolorbox}
%}
\newtcolorbox{ibox}[3][]{
	enhanced jigsaw,
	colback=#3,
	colframe=DarkGray,
	coltitle=Black,
	drop shadow,
	title={\textbf{ \large  #1\hspace{0.4em}#2}},
	before skip = 2.7em,
	attach title to upper,
	after title={:\quad \vspace{0.5em}\ },
	%toc = {#1 \quad* #2}{section}
	%IfValueTF={#1}{\addcontentsline{toc}{section}{#1 \hspace{0.3em}#2}}{}
	code={\addcontentsline{toc}{section}{#1 \hspace{0.3em}#2}}
}


	
%\begin{document}
\begin{proof}
	Es seien $ A := \frac{\partial F}{\partial x} (a,b) := \left( \frac{\partial F}{\partial x_i} (a,b) , \cdots,  \frac{\partial F}{\partial x_k} (a,b) \right)   $  und $ B := \frac{\partial F}{\partial y} (a,b) \neq 0 $ 
	Nach Definition der Differenzierbarkeit von $ F $ in $ (a,b) $ hat die durch 
	\begin{equation}
		F (x,y) = F (a,b) + A \cdot (x-a) + B \cdot (y-b) + \varphi (x,y) \; ; \forall x,y \in U_1 \times U_2
		\tag{1}
	\end{equation}
	definiert Fehlerfunktion $ \varphi : U_1 \times U_2 \to \mathbb{R}  $ die Eigenschaft 
	\begin{equation}
	\varphi (a,b) = 0 ; \; \lim_{ \substack{ a \neq x \to a \\ b \neq y \to b} } \frac{\varphi(x,y)}{ |x-a| + |y-b|} = 0  
	\tag{2}
\end{equation}

Wegen $ F (x, g (x) ) = 0 $ auf $ U_1 $  folgt aus (1) in Verbindung mit $ F (a,b) = 0 $ die Gleichung  
\begin{equation}
g (x) = b - \frac{1}{B} \cdot A \cdot (x-a) - \frac{1}{B} \cdot \varphi( x,g (x) ) \; \forall x \in U_1	
	\tag{3}
\end{equation}
so dass, wegen $ g (a) = b $ zum Beweis des Satzes nur noch zu zeigen ist:\\
\textbf{Behauptung 1: }
Die durch $ \psi := = \frac{1}{B} \varphi \left( x, g (x)  \right)  $ definiert Fehlerfunktion $ \psi : U_1 \to \mathbb{R}  $ hat die 
Eigenschaft $ \lim_{ \substack{ x \to a \\ x \neq a } }\frac{\varphi (x) }{|x-a|} = 0   $ 
Wir werden den Beweis der Behauptung 1 auf folgendes Resultat zurückführen:\\
\textbf{Behauptung 2:} Die Funktion $ g $ ist Lipschitz-stetig in $ a $, das heißt
$$
\exists 0 < \delta < r_1, \, \exists L > 0 : \; 
| g (x)  - g (a)| \leq L \cdot | x - a | \; \forall x \in B(a,\delta) \subset U_1
$$
\textit{\textbf{Beweis der Behauptung 2:} } Sei $ c_1 := \frac{|A|}{|B|}  $ die euklidische Länge des Vektors  $ \frac{1}{B} \cdot A 
\in \mathbb{R}^k$ und sei $ c_2 := \frac{1}{|B|}  $. Wegen (2) existiert ein $ 0 < \delta_1 < \text{ min} \left\{ r_1,r_2 \right\}   $ 
derart, dass $ | \varphi (x,y)| \leq  \frac{1}{2 c_2}  \left( |x-a| + |y-b| \right) \; \forall x \in  B(a, \delta) $ und 
$ \forall y \in  B(b,\delta_1) $ gilt. Da die Funktion $ g $ mit $ g (a) = b  $ an der Stelle $ a $ stetig ist, können wir ein 
$ 0< \delta < delta_1 $, so wählen, dass $ g (x)  \in B(b, \delta_1) \; \forall x \in B (a,\delta) $ erfüllt ist, Daraus folgt dann
$ |\varphi(x, g (x) ) | \leq  \frac{1}{2 c_2}  \left( |x-a| + |g(x) - g (a) | \right) \; \forall x \in B (a,\delta) $ und in 
Verbindung mit (3) erhält man $ |g (x) - g (a) | \leq  c_1 |x-a|+ c_2 |\varphi \left( x, g (x)  \right)| \leq 
c_1 |x-a| + \frac{1}{2} \left( |x-a| + |g (x) - g (a) | \right) \; \forall x \in  B (a, \delta) $  
Was zu $ |g (x) - g (a) | \leq (2c_1 + 1) |x - a| =: L |x-a| \; \forall x \in  B (a, \delta ) $.
\textit{\textbf{Beweis der Behauptung 1:} } Wir fixieren $ \delta > 0 \text{ und }  L > 0  $ wie in Behauptung 2. Ist $ \varepsilon > 0$ 
beliebig vorgegeben, so können wir wie beim Beweis der Behauptung 2 im Hinblick auf (2) und die Stetigkeit von $ g $ ein 
$ 0 < \delta' < \delta $ so wählen, dass die Abschätzung 
$$ | \psi (x) | = c_2 \left| \varphi \left( x, g(x)\right) \right| < \frac{\varepsilon }{1 + L} \left( |x-a| + |g (x) - g (a)|\right)
 \; \forall x \in B \left( a, \delta' \right)  $$
gültig ist. Wegen der Lipschitz-Stetigkeit von $ g $ in $ a $ ergibt sich sofort 
$$ | \psi|<  \frac{\varepsilon }{1 + L} \left( |x-a| + |g (x) - g (a)|\right) \leq  \frac{\varepsilon }{1 + L} 
\left( |x-a| + L|x-a|\right) \varepsilon |x-a| \; \forall x \in B (a, \delta') $$
womit sowohl Behauptung 1 als auch Satz 45 bewiesen wären.
\end{proof}
Ist in der Situation des Satzes 45 die Funktion $ F $ sogar \textit{stetig differenzierbar} auf $ U_1 \times U_2 $, so impliziert das 
Nicht-Verschwinden von $ \frac{\partial F}{\partial y} (a,b) $ die Existenz einer offenen Umgebung $ (a,b) \subset   V_1 \times V_2
\subset  U_1 \times U_2$ des Punktes $ (a,b) $ mit $ \frac{\partial F}{\partial y} (x,y) \neq 0 \; \forall  (x,y) \in  V_1 \times V_2	$.
In diesem Fall ist die implizit definierte Funktion $ g $ aufgrund der Stetigkeit der partiellen Ableitungen von $ F $ \textit{sogar 
differenzierbar auf }$ V_1 = V_1 (a)  $! In erster Linie bemerkenswert ist allerdings eine andere Tatsache: Die stetige
Differenzierbarkeit von $ F $ auf $ U_1 \times U_2 $ in Verbindung mit der Nicht-Entartungsbedingung $ \frac{\partial F}{\partial y} 
\neq 0$ garantiert \textit{die lokale Existenz} stetiger durch $ F \left( x,y \right) = 0 $ implizit definierte Funktion wie in 
Satz 45.

\begin{ibox}[46]{Satz}{CTheorem}
    Seien $ a \in \mathbb{R}^{k}, b \in R, r_1, r_2 > 0	$ sowie $ U_1 := B (a,r_1) \in \mathbb{R}^k, U_2 := B (a,r_2) \subset \mathbb{R} 
		$ Sei ferner
		\begin{align*}
			F: U_1 \times U_2 & \to \mathbb{R} \\
			(x,y) & \mapsto F(x,y)
		\end{align*}
	eine stetig differenzierbar Funktion mit $ F (a,b)  = 0, \; \frac{\partial F}{\partial y} (a,b) \neq 0 $. Dann gibt es offene 
	Umgebung $ V_1 = V_1 (a)  \subset U_1, \; V_2 = V_2 (b)  \subset U_2 $ und eine stetige Funktion $ g: V_1 \to V_2 $ mit 
	$ g (a) = b, \; F \left( x, g(x) \right) = 0 \; \forall  x \in U_1 $, welsche die Auflösung der Gleichung $ F (x,y) = 0 $ nach $ y $ 
	auf $ V_1 \times V_2 $ ermöglicht, insofern als gilt $ \forall (x,y) \in  V_1 \times V_2 $ mit $ F (x,y) = 0 $ folgt $ y = g (x)  $.
\end{ibox}
\smalltitle[]{Bemerkung}
Wie oben erklärt, kann man durch eventuelle Verkleinerung von $ V_1 = V_1 (a)  $ erreichen, dass $ g $ auf $ V_1 $ stetig
differenzierbar ist. Nach Satz 44 ist das Differential von $ g $ in $ x \in V_1 $ durch 
$$ \frac{\partial g}{\partial x_j} (x) = - \frac{ \frac{\partial F}{\partial x_j}( x, g (x))  }{ \frac{\partial F}{\partial y} 
(x, g(x))} $$
$ \forall j = 1, \cdots,  k  $ festgelegt.

\begin{ibox}[]{Fixpunkt}{CDefinition}
    Sei $ U \subseteq \mathbb{R}, f: U \to \mathbb{R}  $ eine Funktion. Dann heißt $ x^{n} \in U $ \textit{Fixpunkt} von $ f $, falls 
		$ f (x^n)  = x^{n} $. Ein Fixpunkt $ x^{n} $ von $ f $ heißt anziehend, falls es eine Umgebung $ V = V (x^n) \subseteq  $ gibt,
		sodass für jedes $ x_0 \in V $, die Folge $ (x_0) \subseteq U  $ mit $ x_{n+1} = f (x_0)  = f^{n+1} (x_0) = \underbrace{  f \circ \cdots 
		\circ f (x_0) }_{n+1 \text{ mal } }$ gegen $ x^{n} $ konvergiert. 
\end{ibox}
\begin{ibox}[]{Kontoaktionssatz}{CLemma}
    Seien $ a,b \in \mathbb{R} , a < b, f : \left[ a,b \right] \to \left[ a,b \right]  $ eine Kontraktion, d.h $ \exists c \in (0,1) 
		\text{ s.d. } \left| f (x) - f (y)  \right| \leq  c \left| x-y \right| \; \forall x,y \in \left[ a,b \right] $. Dann hat $ f $ einen
		eindeutigen Fixpunkt. $ x^{n} \in \left[ a,b \right] \text{ und } \forall x_0 \in [a,b] $ gilt $ \lim_{ n \to \infty  } f^{n} (x_0) =x $ 
\end{ibox}

%\end{document}
