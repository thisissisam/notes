%\documentclass[a4paper]{memoir}
%
\usepackage[ngerman]{babel}
\usepackage{bookmark}
\usepackage{amsmath}
\usepackage{amssymb}
\usepackage{amsthm}
\usepackage[T1]{fontenc}
\usepackage{makeidx}
\usepackage{enumitem}
\usepackage{mathtools}
\usepackage{upgreek}

\usepackage{microtype}
\usepackage{svg}
\usepackage{parskip}
\usepackage{hyperref} % Make TOC clickable

\usepackage{lipsum}
\usepackage{fancyhdr} % Heading customization
\usepackage{geometry} % Adjust page padding 
\usepackage{adjustbox}

\usepackage{xcolor}
\usepackage[most]{tcolorbox}

\definecolor{Black}{HTML}{292939}
\definecolor{CTheorem}{HTML}{FFFFFF}
\definecolor{CLemma}{HTML}{FDFFED}
\definecolor{CDefinition}{HTML}{EFF9F0}
\definecolor{DarkGray}{HTML}{6B6969}
\definecolor{ImportantBorder}{HTML}{F14747}


% Set up page layout
\geometry{
    a4paper,
    left=2.5cm,
    right=2.5cm,
    top=2.5cm,
    bottom=2.5cm
}
\linespread{1.1}
\makeindex

\pagestyle{fancy}
\fancyhf{} % Clear headre/footer
\fancyheadoffset[LE,RO]{0pt} % Adjust headsep for page number and title
\fancyhead[RE,LO]{}
\fancyhead[RE,RO]{\rightmark} % Add pape header title
\fancyhead[LE,LO]{\thepage}  % Add page number to the left
\renewcommand{\headrulewidth}{0pt} % Delete the line in the header 

% Define the \para command
\newcommand{\para}[2]{
    \clearpage % Start on a new page
    \thispagestyle{empty}% removes the top left page number and top right chapter name
    \begin{center} % Center the chapter title
        \vspace*{11em}
        \Huge   \bfseries \S #1 \quad  #2 % Chapter title with symbol and counter
    \end{center}
    \vspace{8em}
    \addcontentsline{toc}{chapter}{#1.\hspace{0.6em} #2} % Add chapter to table of contents
    \fancyhead[RE,RO]{#2} % Add pape header title
}

\newcommand{\nsec}[2]{
    \section*{\large  #1 \hspace{0.3em} #2}
    %\phantomsection
    \addcontentsline{toc}{section}{#1 \hspace{0.3em} #2}
}
\newcommand{\smalltitle}[2][]{
    %\phantomsection
    \subsubsection*{ #1 \hspace{0.2em} #2}
}
\newtcolorbox{ibox}[3][]{
	enhanced jigsaw,
	colback=#3,
	colframe=DarkGray,
	coltitle=Black,
	drop shadow,
	title={\textbf{ \large  #1\hspace{0.4em}#2}},
	before skip = 2.7em,
	attach title to upper,
	after title={:\quad \vspace{0.5em}\ },
}
\renewcommand{\d}[1]{\,\mathrm{d} #1}

%\begin{document}
\begin{ibox}[38]{Definition}{CDefinition}
    Seien $ U \subset \mathbb{R}^n  $ offen, $ f: U \to \mathbb{R}^m  $ eine total differenzierbare Abbildung. Seien
	$ a, b \in U $, sodass $ \left| \gamma _{ab} \right|  \subset U $. Dann gibt es $ \phi_1 , \cdots, \phi_{m} 
	\in \left| \gamma_{ab} \right| $, sodass :
	$$ f \left( b \right) = f \left( a \right) + \begin{pmatrix}
		Df_1( \phi_1) \\
		\vdots \\
		Df_{n}(\phi_{m})
	\end{pmatrix} \cdot (b-a)
	 $$
\end{ibox}
\begin{proof}
	Satz 37 auf $ f_1 , \cdots, f_{n} $ anwenden. 
\end{proof}
\smalltitle[]{Beispiel}
$ f: \mathbb{R} \to \mathbb{R}^{2}, \; t \mapsto (t^{2}, t^{3}) $ 
\\
Ist $ [a,b] \subset R $ beschränkt, $ f:[a,b] \to \mathbb{R}  $ stetig differenzierbare, so besagt der HDI:%
$$ f(b) - f(a) = \int_{a}^{b} f'(x) dx $$
Ist wieder $ \gamma_{ab}(f) = a + t(b-a) $, so erhält man durch Substitution 
$$ f(b) - f(a) = \int_0^{1} f'( \gamma_{ab}(t)) \gamma_{ab}'(t) dt = \left( \int_0^{1} f' \left( \gamma _{ab} (t) 
\right) dt \right) .(b-a) $$
\begin{ibox}[]{Definition}{CDefinition}
    Sei $ \left[ \alpha , \beta  \right] \in  \mathbb{R}  $ und  $ A: \left[ \alpha , \beta  \right] \to 
	\mathbb{R} ^{m \times n}, A(t) = \left( a_{ij}(t) \right)_{ \substack{ i = 1 , \cdots, m \\ j = 1 , \cdots, n }}$ 
	mit $ a_{ij}: \left[ \alpha , \beta  \right] \to \mathbb{R}  $ stetig für $ i = 1 , \cdots, m \text{ und }  j=1 , \cdots, n $ Dann ist die Integral über die \textit{matrixwertige Abbildung} $ A $ gegeben durch:
$$ \int_{ \alpha }^{ \beta }A(t) dt = \left( \int_{ \alpha }^{ \beta } a_{ij}(t) dt \right)_{\substack{ i=1 , \cdots, m \\
j = 1 , \cdots, n}} $$
\end{ibox}
\begin{ibox}[39]{Satz}{CTheorem}
    Seien $ U \subseteq \mathbb{R}^n  $ offen, $ f: U \to \mathbb{R}^n  $ stetig partiell differenzierbar. Seien 
	$ a,b \in U $ sodass $ \left| \gamma _{ab} \right| \subseteq U $. Dann gilt: 
	$$ f(b) - f(a) = \underbrace{ \left( \int_0^1 Df \left( \gamma _{ab} (t)  \right) dt \right)  }_{ \in \mathbb{R}^{
	m \times n}} \cdot \underbrace{ (b-a)}_{ \in \mathbb{R}^n } $$
\end{ibox}
\begin{proof}
	Wir wenden den HDI an auf $ g_{i}:= \left( f_i \circ \gamma_{ab} \right) \in C^{1}\left( [0,1] \right)$ für $ i = 1 , \cdots, m $ und
	erhalten mit der Kettenregel: 
	\begin{align*}
		f_i(b) - f_i(a) &= g_i(1) - g_i(0) = \int_{ 0 }^{ 1 } \frac{d}{dt} g_i(t) dt \\
						&=  \int_{ 0 }^{ 1 } \left( \sum_{j = 1}^{n} \frac{\partial f_i}{\partial x_j} ( \gamma_{ab}(t))
						\cdot (b_j - a_j)\right) dt \\
						&= \sum_{j=1}^{n} \underbrace{\left( \int_{ 0 }^{ 1 } \frac{\partial f_i}{\partial x_j} 
								\left( \gamma _{ab}(t) \right) \cdot
						dt \right)}_{v_j} \underbrace{ (b_j - a_j}_{u_j} \\
						&=  \left< \left( \int_{ 0 }^{ 1 } Df_i ( \gamma_{ab}(t)) dt \right) , (b-a) \right>
	\end{align*}
	Dies zeigt die $ i- $te Zeile der gewünschte Gleichung. 
\end{proof}

\begin{ibox}[]{Hilfssatz}{CDefinition}
    Ist $ \left[ \alpha , \beta  \right] \subseteq \mathbb{R}  $ ein Kompaktes Intervall, $ v : \left[ \alpha , \beta 
	\right] \to \mathbb{R}^n  $ stetige Abbildung. So gilt: 
	$$ \left| \left|  \int_{ \alpha }^{ \beta } v(t) dt  \right|  \right| _{2} \leq \int_{ \alpha }^{ \beta } \|v(t)\|_{1} dt $$
\end{ibox}
\begin{proof}
	Sei $ u := \int_{ \alpha }^{ \beta } v(t) dt \in \mathbb{R}^n  $ . Dann ist 
\begin{align*}
	\| u \|_{2}^{2} = \left<u,u \right> &= \sum_{j=1}^{n} \int_{ \alpha }^{ \beta } v_j(t) dt \cdot u_j\\
	&= \int_{ \alpha  }^{ \beta  } \left( \sum_{j = 1}^{n}v_j(t) \cdot u_j dt \right) \\
	&=  \int_{ \alpha  }^{ \beta  } \left<v(t), u \right> dt \\
  & \leq \int_{ \alpha  }^{ \beta } \left| \left<v(t),u\right> \right| dt \\
  & \stackrel{Schwz.}{ \leq } \int_{ \alpha  }^{ \beta  } \left( \| v(t) \|_{2} \cdot \|u \|_{2} \right) dt \\
  &=  \|u \|_{2} \int_{ \alpha  }^{ \beta  } \|v(t) \|_{2}dt
\end{align*}
Für
$$
\|u \|_{2} > 0 \implies \|u \|_{2} \leq  \int_{ a }^{ b } \|v(t) \|_{2} dt
$$
Für
$$
\|u \|_{2} = 0 \implies \|u \|_{2} = 0 \leq  \int_{ a }^{ b } \|v(t) \|_{2} dt
$$
\end{proof}

\smalltitle[]{Korollar}
Seien $ U \subseteq \mathbb{R}^n  $ offen, $ f: u \to \mathbb{R}^m  $ stetig partiell differenzierbar. Seien 
$ a,b \in U $ mit $ \left| \gamma _{ab} \right| \subseteq U$. Dann gilt: 
$$ \|f(b) - f(a) \|_{2} \leq  M \|b-a\|_{2} $$
für $ M := sup \left\{ \, \|Df(x) \| \,| \; x \in \left| \gamma _{ab} \right|  \right\}  $ Dabei ist die Matrix norm 
$ \|A \| $ für $ A \in \mathbb{R}^{m \times n} $ gegeben durch 
$$ \|A \| = \sup_{ v \in \mathbb{R}^n \setminus \{0\}}  \frac{\|Av \|_2}{\|v \|_{2}} 
=\sup_{\substack{v \in \mathbb{R}^n  \\ \|v \|_{2} = 1}} \frac{\|Av \|_{2}}{\|v \|_{2}}  $$

Es gilt: 
$$ M = \sup_{ \substack{ x \in \left| \gamma _{ab} \right|  \\ v \in  \mathbb{R}^n  \\ \|v \|_{2} = 1 }} 
\underbrace{ \frac{\| Df (x) \cdot v \|_{2}}{\|v \|_{2}} }_{G(x,v)} < \infty \text{ ,da } $$
$$ G: \underbrace{ \| \gamma_{ab} \| \times \partial \mathbb{B}(0,1) }_{kompakt} \to \mathbb{R} \text{ stetig }  $$

\begin{proof}
	Es ist : 
	\begin{align*}
		\|f(b) - f(a) \|_{2} &=  \left|\left| \left( \int_{ 0 }^{ 1 } Df \left( \gamma _{ab}(t) \right) dt \right)
		\cdot (b-a)\right|\right|_{2} \\
		& \leq \int_{ 0 }^{ 1 } \left|\left| Df ( \gamma _{ab}(t)) \cdot (b-a) \right|\right|_{2} dt \\
		& \leq \int_{ 0 }^{ 1 } M \cdot \|b-a \|_{2} dt \\
		&=  M \cdot \|b-a \|_{2}
	\end{align*}
\end{proof}

\para{8}{Die Taylorformel}

\begin{ibox}[]{Definition}{CDefinition}
    Seien $ U \subset \mathbb{R}^n  $ offen, $ x_0 \in U $ und $ f: U \to \mathbb{R}  $ eine $ k $-mal stetig partiell differenzierbar
	Funktion ( $ C^{k} $ ). Dann bezeichnet man für $ 0 \leq  m \leq k $ das durch
	$$ T_{f, x_0, m} \left(x\right) : \sum_{i=0}^{m} \left( \, \sum_{ \substack{ |\alpha| = L \\ \alpha \in \mathbb{N}_{0}^{n} }}
	\frac{D^{\alpha }f \left(x_0\right) }{\alpha !} \left( x - x_0 \right) ^{\alpha }\right) $$ 
		definiert Polynom $ T_{f,x_0,m} $ als \textbf{ $ m $-tes Taylorpolynom von $ f \text{ in } x_0 $ }. Hierbei wird mit 
	$ |\alpha| : = \alpha_1 +  \cdots + \alpha_n $ die \textbf{Länge des Multiindex} $ \alpha := \left( \alpha_1  \cdots  \alpha_{n} \right) \in \mathbb{N}_{0}^{n} $ und mit $ \alpha ! $ das Produkt $ \alpha ! := \alpha_1 ! , \cdots,  \alpha_{n} ! $ bezeichnet. Der 
	\textbf{Differentialoperator} $ D^{\alpha } $ ist für $ \alpha  \in \mathbb{N}_{0}^{n} $ gegeben durch $ D^{\alpha }f = 
	D_1^{\alpha } \cdots  D_n^{\alpha_n} f \in C^{|\alpha |}$ dabei ist $ D_{\gamma }^{\alpha r} := \underbrace{ D_{\gamma } 
	D_{\gamma }, \cdots,  D_{\gamma }}_{\alpha \gamma \text{ -mal } } \; \gamma = 1, \cdots,  n \text{ für } y = \left( y_1
, \cdots,  y_n\right) \in \mathbb{R}^n $	ist mit  $ y^{\alpha } $ die reelle Zahl $ y^{\alpha } = y_1^{\alpha_1}
\cdots  y_n^{\alpha_n} $  gemeint. Wie im Kapitel 7 werde für $ x,x_0 \in \mathbb{R}^n  $ mit $ \upgamma_{x_0x} $ die $ c^{ \infty } $
-Ableitung $ \upgamma_{x_0x}: \mathbb{R}  \to \mathbb{R}^n  $, $ t \mapsto \upgamma_{x_0x} \left(t\right) := x_0 + t(x-x_0)$ und mit
$  \left| \upgamma_{x_0x} \right| = \upgamma_{x_0x} \left( \left[ 0,1 \right]  \right)  $ die Verbindungsstrecke von $ x_0 \text{ und }
x $ bezeichnet. Die Taylorformel für $ C^{k+1} $ Fkten mehrerer veränderlichen lautet dann wie folgt:
\end{ibox}
\begin{ibox}[40]{Taylor}{CTheorem}
	Seien $ U \subset \mathbb{R}^n  $ offen, $ x_0 \in U $ und $ f:U \xrightarrow{c^{k+1}}$ eine $ (k+1) $-mal stetig 
	partiell differenzierbare Funktion. Ist dann $ x \in  U $ win Punkt mit $ \| \upgamma_{x_0x} \| \in U $, so existiert ein 
	$ \xi \in \left| \upgamma_{x_0x} \right|  $ mit 
	$$ f \left(x\right) = \sum_{l=0}^{k} \left( \, \sum_{ \substack{ |\alpha | = l \\ \alpha \in \mathbb{N}_{0}^{n} }} 
	\frac{D^{\alpha }f \left(x_0\right) }{\alpha !} \left( x - x_0 \right) ^{\alpha } \right) + \sum_{ \substack{ |\alpha | = l \\ 
\alpha  \in  \mathbb{N}_{0}^{n}}} \frac{D_{\alpha }f \left(x_0\right) }{\alpha !} \left( x-x_0 \right) ^{\alpha } $$
\end{ibox}
 
\smalltitle[]{Bemerkung:}
Unter den Voraussetzungen von Satz 40 wird die Funktion $ f $ an der Stell x durch das $ k $-te Taylorpolynom von $ f $ in $ x_0 $ 
approximiert. Der Fehler $ R_{k+1} \left(x\right)  := f \left(x\right) - T_{f, x_0, k} \left(x\right)  $ die Approximation, das 
\textbf{$ k $-te Restglied} in $ x  $ kann in de Form
$$
R_{k+1}\left(x\right)  =\sum_{ \substack{ |\alpha | = k+1 \\ \alpha \in \mathbb{N} _{0}^{n} }} \frac{D_{\alpha }f \left(x_0\right) }{
\alpha !} \left( x -x_0 \right) ^{\alpha }
$$ dargestellt werden, wobei $ \xi \in  \left| \upgamma_{x_0x}  \right|  $  geeignet zu wählen ist.
\begin{proof}
	Wir werden den Taylorschen Satz für Funktionen einer veränderlichen auf die Funktion $ g: \left( f \circ \upgamma_{x_0x} \right) 
	: [0,1] \xrightarrow{c^{k+1}} \mathbb{R} $ an. Danach existiert ein $ t_0 \in  (0,1) $ mit
	$$
	g (1) = g \left(0\right) + \sum_{l=1}^{k} \frac{g' \left(0\right) }{l !} \left(  1 - 0 \right) ^{l} + \frac{g^{k+1} \left(t_0\right) 
	}{\left( k+1 \right) !} \left( 1-0  \right) ^{k+1}.
	$$
Dies ist gleichbedeutend mit $ f \left(x\right)  = f \left(x_0\right) \sum_{l=1}^{n} \frac{g' \left(0\right) }{l !} + \frac{g^{k+1}
\left(t_0\right) }{\left( k+1 \right) !}  $ Zum Beweis der Taylorformel genügt es also zu zeigen: 
\end{proof}
\smalltitle[]{Behauptung:}
$ \forall  t \in  [0,1], \, l = 1 , \cdots,  k +1 :  $ 
\begin{equation}
	g^{\left(l\right) } \left(t\right) = \sum_{ \substack{ |\alpha | = l \\ \alpha  \in \mathbb{N}_{0}^{n}}} l! \frac{D^{\alpha}f 
	\left( \upgamma_{x_0x} \left(t\right)  \right) }{\alpha !} \left( x - x_0 \right) ^{\alpha } 
	\tag{1}
\end{equation}
Also Satz 40 für $ \xi := \upgamma_{x_0x} \left(t\right)  $ 
\begin{proof}
	Nach der Kettenregel ist $ g' \left(t\right) = \sum_{i=1}^{m}D_i f \left( \upgamma_{x_0x} \left(t\right)  \right) \cdot 
	\left( x_i-x_0 \right) \; \forall t \in [0,1]$ durch Induktion über $ l $ folgt ebenso mit Hilfe der Kettenregel, dass 
	\begin{equation}
		g^{\left(l\right) } \left(t\right) = \sum_{i_1, \cdots,  i_l = 1}^{n} D_{i_{l}} , \cdots,  D_{i_{1}} f \left( \upgamma_{x_0x}
		\left(t\right) \right) \cdot \left( x_{i_1} - x_{0i_1} \right) \cdots \left( x_{i_{l}} - x_{0i_{l}} \right) 
	\tag{2}
	\end{equation}
	für alle $ t \in [0,1] $ gilt zum Beweis von (1) müssen wir nur noch die Summanden in (2) geeignet zusammenfassen. Dabei nutzen wir:
\smalltitle[]{Hilfssatz}
Seien $ U \in \mathbb{R}^n  $ offen, $ 2 \leq  k \in \mathbb{N}, f \in C^{k} \left(U\right)  \text{ und }  \left( i_1 , \cdots,  i_{n} 
\right) \in \left\{ 1, \cdots,  n \right\} ^{k} $. Dann gilt für jede Permutation $ \pi : \left\{ 1, \cdots,  k \right\} \to 
\left\{ 1, \cdots,  k \right\} : \; D_{i_{n}} D_{i_{n-1}}  \cdots \; D_{i_2} D_{i_1} f = D_{i \pi \left(k\right) } D_{i \pi (k-1)} 
 \cdots D_{i \pi (1)} f  $ 
 \parskip=3pt
 Der Beweis ist leicht durch Induktion über $ k $ zu führen. Für $ k=2 $ geht der Hilfssatz in Satz 29 über, nach Induktionsvoraussetzung
 ist die Vertauschung benachbarter $ D_{\gamma } \cdot D_{\mu}$ erlaubt, der Induktionsschritte kann also vollzogen werden, wenn man 
 berücksichtigt, dass sich jede Permutation als Hintereinanderausführung solcher Vertauschungen beschreiben lässt
\end{proof}

Der Hilfssatz besagt, dass in Gl (2) alle Summanden $ D_{i_{l}} \cdots D_{i_1} f \left( \upgamma_{x_0x} \left(t\right)  \right) 
\cdot \left( x_{i_1} x_{0i_1} \right) \cdots \left( x_{i_{l}} - x_{0i_{l}} \right)   $ in denen jeder index $ \gamma  $ genau 
$ a_{\gamma }$-mal auftritt. $ \left( 1 \leq  \gamma  \leq n , \alpha_1 + \cdots + \alpha_n = l, \alpha = 
\left( \alpha_1 , \cdots,  \alpha_n \right) \in  \mathbb{N} _{0}^{n}\right)  $ folgt aus (2): $ \forall t \in [0,1] $ 

\begin{align*}
	g^{\left(l\right) }\left(t\right)  &= \sum_{i_1 , \cdots,  i_{l} = 1}^{n} D_{i_{l}} \dots D_{i_1} f \left( \upgamma_{x_0x}(t)\right) 
	\cdot \prod_{\alpha  = 1}^{n} \left( x_{\gamma } - x_{0 \gamma } \right) ^{\alpha _{\gamma }} \\
	&= \sum_{ \substack{ |\alpha | = l \\ \alpha \in \mathbb{N}_{0}^{n} }} \frac{l !}{\alpha_1 ! \cdots \alpha_n !} D_{1}^{\alpha_1}
	\cdots D_{n}^{\alpha_n} f \left( \upgamma_{x_0x} \left(t\right)  \right) \cdot \prod_{\alpha = 1}^{n} \left( 
		x_\gamma - x_{0 \gamma }\right)^{\alpha_{\gamma }}  \\
		&= \sum_{ \substack{ |\alpha | = l \\ \alpha \in \mathbb{N}_{0}^{n} }} l ! \frac{D^{\alpha }f \left( \upgamma_{x_0x} 
		\left(t\right) \right) }{\alpha !} \left( x - x_0 \right)^{\alpha }
\end{align*}
Damit ist sowohl (1), als auch der Taylorschen Satz bewiesen.


%\end{document}

