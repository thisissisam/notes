%\documentclass[a4paper]{memoir}
%
\usepackage[ngerman]{babel}
\usepackage{bookmark}
\usepackage{amsmath}
\usepackage{amssymb}
\usepackage{amsthm}
\usepackage[T1]{fontenc}
\usepackage{makeidx}
\usepackage{enumitem}
\usepackage{mathtools}
\usepackage{upgreek}

\usepackage{microtype}
\usepackage{svg}
\usepackage{parskip}
\usepackage{hyperref} % Make TOC clickable

\usepackage{lipsum}
\usepackage{fancyhdr} % Heading customization
\usepackage{geometry} % Adjust page padding 
\usepackage{adjustbox}

\usepackage{xcolor}
\usepackage[most]{tcolorbox}

\definecolor{Black}{HTML}{292939}
\definecolor{CTheorem}{HTML}{FFFFFF}
\definecolor{CLemma}{HTML}{FDFFED}
\definecolor{CDefinition}{HTML}{EFF9F0}
\definecolor{DarkGray}{HTML}{6B6969}
\definecolor{ImportantBorder}{HTML}{F14747}


% Set up page layout
\geometry{
    a4paper,
    left=2.5cm,
    right=2.5cm,
    top=2.5cm,
    bottom=2.5cm
}
\linespread{1.1}
\makeindex

\pagestyle{fancy}
\fancyhf{} % Clear headre/footer
\fancyheadoffset[LE,RO]{0pt} % Adjust headsep for page number and title
\fancyhead[RE,LO]{}
\fancyhead[RE,RO]{\rightmark} % Add pape header title
\fancyhead[LE,LO]{\thepage}  % Add page number to the left
\renewcommand{\headrulewidth}{0pt} % Delete the line in the header 

% Define the \para command
\newcommand{\para}[2]{
    \clearpage % Start on a new page
    \thispagestyle{empty}% removes the top left page number and top right chapter name
    \begin{center} % Center the chapter title
        \vspace*{11em}
        \Huge   \bfseries \S #1 \quad  #2 % Chapter title with symbol and counter
    \end{center}
    \vspace{8em}
    \addcontentsline{toc}{chapter}{#1.\hspace{0.6em} #2} % Add chapter to table of contents
    \fancyhead[RE,RO]{#2} % Add pape header title
}

\newcommand{\nsec}[2]{
    \section*{\large  #1 \hspace{0.3em} #2}
    %\phantomsection
    \addcontentsline{toc}{section}{#1 \hspace{0.3em} #2}
}
\newcommand{\smalltitle}[2][]{
    %\phantomsection
    \subsubsection*{ #1 \hspace{0.2em} #2}
}
\newtcolorbox{ibox}[3][]{
	enhanced jigsaw,
	colback=#3,
	colframe=DarkGray,
	coltitle=Black,
	drop shadow,
	title={\textbf{ \large  #1\hspace{0.4em}#2}},
	before skip = 2.7em,
	attach title to upper,
	after title={:\quad \vspace{0.5em}\ },
}
\renewcommand{\d}[1]{\,\mathrm{d} #1}

%\begin{document}
Unsere Ausgangsquader $ Q_0 = Q $ hat die geforderten Eigenschaften. Seine 
$ Q = Q_0 \supset Q_1 \supset Q_2 \supset \cdots$ bereits konstruiert. Es gelte $ Q_{m} = I_1 
\times I_2 \times \cdots I_{n} \; , I_{k} = \left[ a_{k}, b_{k} \right]  \subset \mathbb{R} \; 
\forall k = 1,2,\cdots, n$ . Wir splitten jedes der Intervalle $ I_{k} $ in der Intervalllemitte 
auf und erhalten Unterteilung $ I_{k} = I_{k_{1}} \cup I_{k_{2}} $ mit \\
$ I_{k_{1}} = \left[ a_{k}, \frac{a_{k}+ b_{k}}{2} \right] \; , I_{k_{2}} 
= \left[ \frac{a_{k}+ b_{k}}{2}, b_{k} \right] \; , 1 \leq k \leq n $. Die kartesischen Produckte
$ I_{1_{ \gamma_{1}}} \times I_{2_{ \gamma_{2}}} \times \cdots  I_{n_{ \gamma_{n}}} \; ,
\gamma \in {1,2} \; \forall 1 \leq k \leq n $ aller Intervallhälften der $ I_{k} \; , k = 1,2, \cdots n $ unterteilen den  abgeschlossener Quader $ Q_{m} $  in 
in insgesamt $ 2^{n} $  vielen kleinere abgeschlos. Quader  
Nur können wir folgendes Kriterium formulieren:
\begin{ibox}[25]{Satz von Heine-Borel}{CTheorem}
    Für ein Menge $ K \subset  \mathbb{R}^n  $ sind folgende Ausagen äquivalent:
	\begin{enumerate}[label=\alph*)]
		\item $ K $ ist abgeschlossen und beschränkt
		\item $ K $ ist kompakt
	\end{enumerate}
\end{ibox}

\begin{proof}
	Siehe Satz 21.
\end{proof}

\smalltitle[]{Korollar}
Für jede kompakte Teilmenge $ \emptyset \neq A \subset \mathbb{R}  $ gilt:
$$ sup A \in A, \; inf A \in A $$
\begin{proof}
	Da $ A $ als kompakte Menge insbesondere beschränkt ist, existeren $ sup A, \; inf A \in R $.
	Weil sowohl $ sup A  $ als auch $ inf A $ Häufungspunkte geeigneter Folgen in $ A $ sind, 
	impliziert die Abgeschlossenheit von $ A $ in Verbindung mit Satz 8, dass $ sup A \in  A
	\text{ und } inf A \in A $ gilt.
\end{proof}

\begin{ibox}[26]{Satz}{CTheorem}
    Ist $ U \subset  \mathbb{R}^n  $ offen $ K \subset  U $-kompakt und $ f:U \to \mathbb{R}^m $
	-stetige Abbildung, so ist $ f(k) \subset \mathbb{R}^m $ ebenfalls kompakt.
\end{ibox}

\begin{proof}
	Sei $ \left( U_{i} \right)_{i \in  I} $   eine offene Überdeckung von $ f(k) $. Sei 
	$ \forall i \in  I, \; V_{i}:= f^{-1} \left( U_{i} \right) $, $ f- $stetig $ \implies i \in I$
	$ V_{i}- $offen. Die Familie $ \left( V_{i} \right)_{i \in  I} $ ist dine Überdeckungvon $ K $
	, $ K $-kompakt $ \implies  \exists V_{i_{1}}, V_{i_{2}} , \cdots V_{i_{n}} \text{ s.d. } K 
	\subset \bigcup\limits_{p = 1}^{k} U_{i_{p}} \implies f(k) \subset  
	\bigcup\limits_{p = 1}^{k} U_{i_{p}} \implies  f(k) $-kompakt
\end{proof}

\begin{ibox}[27]{Satz}{CTheorem}
    Ist $ U \subset \mathbb{R}^n  $ offen, $ k \subset  U $ kompakt und $ f: U \to \mathbb{R}  $
	eine stetige Funktion auf $ U $, so gibt es Punkte $ p,q \in K $ mit $ f(p) = sup \left\{ 
f(x): x \in  K \right\}, f(q) = inf \left\{ f(x): x \in K \right\}   $ 
\end{ibox}

\begin{proof}
	Laut Satz 26 ist $ A := f(k) \subset  \mathbb{R}  $ kompakt, also im Hinblick auf dem Korollar
	zu Satz 25: $ sup f(k) \in  f(k), \; inf f(k) \in  f(k) $ . Deshalb können wir $ p,q \in K $ 
	mit gewünschten Eigenschaften finden
\end{proof}

\begin{ibox}[]{Definition}{CDefinition}
    Sei $ U \subset  \mathbb{R}^n  $ eine beliebige Teilmenge und $ f: U \to \mathbb{R}^m  $ eine 
	Abbildung. Dann nennt man $ f $ \textit{gleichmäßig stetig} auf $ U $ wenn gilt:
	$$ \forall \varepsilon > 0 \; \exists \delta > 0 \text{ s.d. } \left| x - y \right| < \delta , 
	 \; x,y \in  U \implies  \left| f(x) - f(y) \right| < \varepsilon  $$
\end{ibox}

\begin{ibox}[28]{Satz}{CTheorem}
    Sei $ R \in  \mathbb{R}^n $ kompakt und $ f: K \to \mathbb{R}^m $ eine stetige Abbildung.
	Dann ist $ f $ gleichmäßig stetig auf $ K $ 
\end{ibox}

\begin{proof}
	Sei $ \varepsilon > 0 $ vergegeben. $ \forall a \in  K \; \exists r(a)>0$  mit 
	$ \left| f(x) - f(a) \right| < \frac{ \varepsilon }{2} \; \forall x \in  B(a, r(a)) \cap K $
	(wegen der Stetigkeit von $ f $  in a) Offenbar ist $ K \subset \bigcup\limits_{a \in K}^{} 
	B(a,\frac{r(a)}{2} ) $. $ K $-kompakt $ \implies a_1, a_2, \cdots , c_{k} $ s.d $   K \subset 
	\bigcup\limits_{p = 1}^{} B \left( a_{p}, \frac{r \left( a_{p} \right) }{2}  \right) $.
	Man setz nun $ \delta := \frac{1}{2} min \left\{ r \left( a_{1} \right), \cdots, 
	\left( a_{k} \right) \right\}   $ Seien dann $ x_{1}, x_{2} \in K $ mit 
	$ \left| x_1-x_2 \right|  < \delta $ beliebig gewählt. Wir fixieren ein
	$ i \in  \left\{1,2,\cdots , k  \right\}  $  mit $ x_1 \in  B \left( a_{i},
	\frac{r \left( a_{i} \right) }{2}  \right)  $; dann gilt auch $ \left| x_2 - a_{i} \right| 
	\leq \left| x_2 - x_1 \right| + \left| x_{1}  + a_{i}\right| 
	< \delta + \frac{r(a_i)}{2} < r(a_i) $
	also liegt mit $ x_1 \text{ auch }  x_2 $ in $ B(a_{i}, r \left( a_{i} \right))
	$, und wir erhalten $ \left| f \left( x_1 \right) - f \left( x_2 \right)  \right| \leq 
	\left| f \left( x_1 \right) - f \left( a_{i} \right)   \right| + \left| f \left( a_{i} \right) 
	- f \left( x_2 \right) \right| $ $ < \frac{ \varepsilon }{2} + \frac{ \varepsilon }{2}  $.
	Damit ist die gleichmäßig Stetigkeit von $ f $ auf $ K $ beweisen.
\end{proof}
\para{5}{Partielle Ableitung}
\begin{ibox}[]{Definition}{CDefinition}
    Seien $ U \subset \mathbb{R}^n  $ offen, ferner 
	$$ f:U \to \mathbb{R}  (x_1,x_2,\cdots,x_{n}) \mapsto f(x_1,x_2, \cdots, x_{n} $$
     eine Funktion auf $ U $ und $ a = (a_1,a_2, \cdots, a_{n}) \in U $ 
	 \begin{enumerate}[label=\alph*)]
	 	\item $ f $ heißt in $ a $ \textit{partiell differenzierbar} nach $ x_i $ oder auch \textit{partiell 
			differenzierbar bezüglich der $ i $-ten Koordinate}, wenn der Grenzwert 
			$$	
				 D_{i}f(a): \frac{\partial f}{\partial x_{i}} (a)_i := \\= 
				\lim_{\substack{t \to a_{i} \\ t \neq  a_i}} \frac{f(a_1,\cdots ,
			a_{i-1}, t, a_{i+1}, \cdots , a_n) - f(a_1,\cdots ,a_{i-1}, a_i, a_{i+1}, \cdots , a_n) }{t-a_i}  
			$$	
			dann $ \frac{\partial f}{\partial x_i} (a) $ die \textit{partille Ableitung von $ f $ nach $ x_i $ an der 
			Stelle $ a $ } 
		\item $ f $ heißt in $ a $ partiall differenzierbar, wenn $ f $ in $ a $ nach \textit{ allen} $ x_i, i = 1,\cdots,n$ 
			partiell differenzierbarist. In diesem Fall nennt man den Vektor $ grad(f(a)) = \left( 
			\frac{\partial f}{\partial x_1}(a), \frac{\partial f}{\partial x_2}(a), \cdots ,
			\frac{\partial f}{\partial x_n}(a) \right)  $ den \textit{Gradient von $ f $ in $ a $ } 
		\item $ f $ heißt auf $ U $ \textit{ partiell differenzierbar }, wenn $ f $ partiell differenzierbar in $ a $ 
			für jedes $ a \in  U $ ist.
		\item $ f $ heißt in $ a $ \textit{stetig partiell differenzierbar} , wenn $ f $ auf $ U $ partiell differenzierbar
			ist und zusätzlich alle Ableitungen $ \frac{\partial f}{\partial x_{i}}  = D_i f_{i} U \to \mathbb{R} , 
			1 \leq i \leq  n$, an der Stelle $ a $ stetig sind.
		\item $ f $ heißt \textit{stetig partiell differenzierbar auf $ U $ }, wenn $ f  $ in jedem $ a \in  U $ stetig 
			partiell differenzierbar ist. Für die Menge aller solcher Funktion führen wir die Bezeichnung 
			$$ C^{1}(U) : = \{f: U \to \mathbb{R} : f \text{ ist stetig partiell differenzierbar auf } U\} $$ ein.
	 \end{enumerate}
\end{ibox}
\smalltitle[]{Beispiel}
... 
\begin{ibox}[]{Definition}{CDefinition}
	Seien $ U \subset \mathbb{R}^n  $ offen, ferner 
	$ f:U \to \mathbb{R}  (x_1,x_2,\cdots,x_{n}) \mapsto f(x_1,x_2, \cdots, x_{n} $
	eine Funktion auf $ U $ und $ a = (a_1,a_2, \cdots, a_{n}) \in U $. Seien $ k \in \mathbb{N}  $ und $ i \in {1,\cdots,n }$
	\begin{enumerate}[label=\alph*)]
		\item $ f $ heißt in $ a $ \textit{ $ k $-mal partiell differenzierbar nach $ x_i $  },wenn $ f \; (k-1) $ -mal partiell
			differenzierbar auf $ U $ ist und alle partiell Ableitungen $ (k-1) $-ter Ordnung von $ f $ an der Stell $ a $ partiell
			differenzierbar nach $ x_i $ sind.
		\item f heißt in $ a $ \textit{$ k $-mal partiell differenzierbar}, wenn $ f $ in $ a \; k$-mal partiell differenzierbar ist.
			Nach $ x_j $ für alle $ j \in  {1,2, \cdots, n} $ ist. Die partielle Ableitungen der Ordnung $ k $ von $ f $ in $ a $ 
			sind dann gegeben durch 
			\begin{align*}
				D_{i_k}\left( D_{i_{k-1}} \left( \cdots D_{i_2} \left( D_{i_1} (f) \right) \cdots \right)  \right)(a) &:= 
			\frac{\partial^{k} f}{\partial x_{i_{k}} \partial x_{i_{k-1}} \cdots \partial x_{i_1}} (a) \\
			&:= D_{i_{k}} D_{i_{k-1}} \cdots D_{i_1} f(a) \\
			&:= f_{x_{i_{k}} x_{i_{k-1}} \cdots x_{i_1}}^{\left( k \right) } (a)
			\end{align*}
			
			 wobei die Indizes $ i_{1}, \cdots, i_{k} $ voneinandere unabhängig die Menge $ {1,2,\cdots,n}  $ durchlaufen. 
		\item $ f $ heißt $ k $- \textit{mal partiell differenzierbar auf $ U $ }, wenn $ f $ in jedem $ a \in U \; k-$mal partiell
			differenzierbar sit. Die Funktionen $ D_{i_{k}} \cdots D_{i_1} f:U \to \mathbb{R} , i_1,i_2, \cdots, i_{k} \in{1,\cdots,n}$
			heißen \textit{partiell Ableitung $ k $-te Ordnung von f}.
		\item $ f $ heißt \textit{$ k $-mal stetig partiell differenzierbar auf $ U $ } wenn $ f k- $mal partiell differenzierbar auf 
			$ U $ ist und alle partiellen Ableitungen der Ordnung $ \leq  k $ auf $ U $ stetig sind. Für die Menge aller solcher 
			Funktionen führen wir die Bezeichnung 
			$$ C^{k}(U) : = \{f: U \to \mathbb{R} : f \text{ ist $ k $-mal stetig partiell differenzierbar auf } U \} $$ ein.
	\end{enumerate}
\end{ibox}
\begin{ibox}[29]{Satz}{CTheorem}
    Sei $ U \subset  \mathbb{R}^n  $ offen und sei $ f: U \to \mathbb{R}  $ zweimal partiell differenzierbar auf $ U $; außerhalb 
	seien alle partiellen Ableitungen 2.Ordnung von $ f $ stetig auf $ U $. Dann gilt :
	$$ \frac{\partial^{2} f }{\partial x_{i}  \partial x_{j} } =  \frac{\partial^{2} f }{\partial x_{j}  \partial x_{i} } 
	\forall  i,j \in {1,\cdots, n} $$
\end{ibox}

\smalltitle[]{Korollar}
Seien $ U \subset  \mathbb{R}^n  $ offen, $ k \geq 2 $ und sei $ f \in  C^{k}(U) $. Dann gilt für jedes $ k $-Tupel von Indizes \\
$ \left( i_1,i_2,\cdots,i_{k} \in  {1,2,\cdots,n}^{k} \right)  $ und für jedes Permutation $ \pi: \{1,\cdots,k\} \to \{1,\cdots,k\}: $ 
$$ D_{i_{k}} D_{i_{k-1}} \cdots D_{i_{1}} f = D_{i_{\pi(k)}} D_{i_{\pi(k-1)}} \cdots D_{i_{\pi(1)}}f  $$
\begin{proof}
	Der Beweis erfolgt durch vollständige Induktion über $ k $. Den Induktions Anfang bildet Satz 29 im Induktionsschluss verwendet 
	man die Tatsache, dass sich jede Permutation als Hintereinanderausführung endliche vieler Vertauchungen benachbarter 
	Glider darstellen lässt.

\end{proof}

%\end{document}
