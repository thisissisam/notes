
\documentclass[11pt,a4paper]{article}
\usepackage[utf8]{inputenc}
\usepackage{amsmath, amsthm, amssymb, amsfonts}
\usepackage{graphicx}
\usepackage{hyperref}
\usepackage{enumitem}

% Title, Author, Date
\title{Math Notes}
\author{ßßßßßß}
\date{\today}

% Theorem Styles
\newtheorem{theorem}{Theorem}[section]
\newtheorem{lemma}[theorem]{Lemma}
\newtheorem{Korollary}[theorem]{Korollary}
\theoremstyle{definition}
\newtheorem{definition}[theorem]{Definition}
\newtheorem{example}[theorem]{Beispiel}
\theoremstyle{remark}
\newtheorem*{remark}{Bemerkung}

% Custom Commands
\begin{document}

\maketitle
\tableofcontents
\newpage

\section*{Effektivzinsberechnung}
\subsection*{Beispiel 4.1}
Eine Kredit mit einer Kreditsumme vom $ K_0 = 11 000 \$ $ und einer Auszahlung von $ \hat{K_0} = 10368 \$ $ soll bei exponential Verzinsung mit (nominellen) Zinssatz $ r = 20 \% p.a $ durch zwei gleichhohe Annuitäten der Höhe $ A $ zurückgezahlt werden \break
Die Annuitäten berechnen sich gemäß des Äquivalenzprinzips aus (s.a (2))
$$ K_0 (1,2)^{2} = 10000\$ (1,2)^{2} = A 1,2 + A \implies A = 7200\$ $$
Die Äquivalenzprinzip bezieht sich hierbei auf die Kreditsumme $ K_0 $ und die Gegenleistung bei nominelle Zinssatz. Um die tatsächliche  Leistung $ \hat{K_0} $ und die Gegenleistung äquivalent zu machen, bräuchte es einen angepassten Zinssatz:
$$ \hat{K_0} \left( 1 + r_{eff} \right)^{2} = A \left( 1 + r_{eff} \right) + A \implies 10368 (1+ r_{eff})^{2} = 7200 ( 1 + r_{eff} +A)\implies r_{eff} = 25\%  $$
Unter dem \underline{Effektivzinssatz} einer Zahlungsreihe versteht man der Zinssatz, bei dessen Anwendung Leistungen und Gegenleistung finanzmathematische äquivalent sind. Der Effektivzinssatz wird als Vergleichskriterium für unterschiedliche Anlage- oder Kreditgeschäfte genutzt. 

Bei einem Annuitätendarlehen mit Kreditsumme $ K_0 > 0 $, Disagio (in Prozent) $ d \in (0,1) $, nominellem Zinssatz $ r > 0 $ und 
Laufzeit $ N \in \mathbb{N} $ hat man (s.a (2)) die Annuitäten

$$
	A = K_0 r \frac{\left( 1 + r \right) ^{N}}{\left( 1 + r \right) ^{N}-1}  (3)
$$
und \begin{math}
	(1-d) K_0 (1 + r_{eff})^{N} = A \sum_{i=0}^{N-1} (1+ r_{eff})^{i} = A \frac{\left( 1 + r \right) ^{N}-1}{r_{eff}} (4)
\end{math}

es ist eine Gleichung $ N $-te Grades (in der Unbekannten $ r_{eff} $ ) und daher i.A nicht explizit lösbar.

\subsection*{Lemma 4.2}
Es existiert eine eindeutigen Lösung $ r_{eff}>0 $ von (4). Es gilt $ r_{eff} \in \left( r, \frac{A}{(1-d)K_0}  \right)  $ 

\subsection*{Beweis}
Unter Verwendung von (3) ist (4) äquivalent zu
$$ r_{eff} = \frac{A}{(1-d)K_0} \frac{ \left( 1 + r_{eff} \right)^{N} -1 }{ \left( 1 + r_{eff} \right)^{N}}  $$
$$=  \frac{r}{1-d} \frac{(1+r)^{N}}{  \left( 1 + r \right)^{N} -1 } \frac{ \left( 1 + r_{eff} \right)^{N} -1 }{ \left( 1 + r_{eff} \right)^{N} }  $$
("Fixpunktgleichung")\break
Sei $ f: (0, \infty) \to \mathbb{R} $ 
$$ f(x) = \frac{r}{1-d} \frac{\left( 1 + r \right) ^{N}}{\left( 1 + r \right) ^{N} -1} \left( 1 - \frac{1}{\left( 1 + r \right) ^{N}}  \right)   $$
Beachte, dass
$$ \frac{A}{(1-d)K_0} = \frac{r}{1-d} \frac{\left( 1 + r \right) ^{N}}{\left( 1 + r \right) ^{N} -1 } > r $$

Es gilt: 
\begin{enumerate}[]
	\item $ f $ ist strikt wachsend und stetig,
	\item $ \lim_{ x \to \infty} f (x) = \frac{A}{(1-d) K_0}  $,
	\item aus (1) und (2) folgt $ f \left( \frac{A}{(1-d)K_0}  \right) < \frac{A}{(1-d)K_0}  $,
	\item $ f(r) = \frac{r}{1-d} > r $,
	\item f ist strikt konkav
\end{enumerate}

Sei $ F : (0, \infty) \to \mathbb{R}, F(x) = f(x) -x$. Aus (4) folgt $ F(r) > 0 $ und aus (3) folgt $ F \left( \frac{A}{(1-d) K_0}  \right) < 0 $. Daher zeigen (1) und der ZWS, dass es eine $ x^{*} \in \left(  r,\frac{1}{(1-d)K_0} \right)  $ gibt, s.d $ F(x*) = 0 $ Damit hat $ f $ einen Fixpunkt (d.h es ex. eine Lösung von (4)). Wegen  (5)  ist $ F $ strikt konkav und somit der Fixpunkt eindeutig.
\break
Zur numerischen Berechnung von $ r_{eff} $ kann man Verfahren zur Nullstellenbrechung anwenden (z.B Bisektionsverfahren, Sekantverfahren, Newton-Verfahren, Regula falsi ...)

\subsection*{Regula falsi}
Ziel: Finde Nullstell von $ F: \mathbb{R} \to \mathbb{R} \in C^{0} $ \break
Start : $ a_0, b_0 \in \mathbb{R} $ mit $ F(a_0) F(b_0) < 0 $ \break
Ansatz: Lege Gerade durch $ a_0, F(a_0) \text{ und }  (b_0, F(b_0)  $ und bestimme deren Nussstelle $ c_1 $ (wie bei Sekantenverfahren).
Falls $ F(a_0) \text{ und }  F(c_1) $ verschiedene Vorzeichnen haben, nimm $ a_0 $ und $ c_1 $ als neue Intervallgrenze, sonst $ b_0 $ 
und $ c_1 $ (ähnlich zu Bisektionsverfahren). \break
Iteration: Setze $$ c_{k} := \frac{a_{k-1}F(b_{k-1}) - b_{k-1}F(a_{k-1L})}{F(b_{k-1}) - F(a_{k-1})}  $$
Falls : $ F(c_k) \approx = 0 $: Ende. \break
Falls : $ F \left( a_{k-1} \right) F (c_{k}) < 0 : $ Setze $ a_k = a_{k-1} \text{ und }  b_k = c_k $ \break
Sonst : Setze $ a_k = c_k \text{ und }  b_k = b_{k-1} $ 


\section*{I.5 Festverzinsliche Wertpapiere}
Ein \underline{Wertpapiere} (genaür: eine \underline{Wertpapierurkunde}) gibt der Investorin (Besitzerin) ein Forderungsrecht auf finanzielle Gegenleistung (z.B in Form von klar definierten Zahlungen zu definierten Zeitpunkten) des \underline{Emittenten} (Verkäufers)
\break
Wir betrachten hier nur \underline{festverzinsliche Wertpapiere}. Dies sind Wertpapiere, bei denen die Höhe der regelmäßigen Gegenleistung
fest vereinbart wird. Der \underline {Nennwert} $ N> 0 $ ist der mit der Wertpapierurkunde verbriefte Darlehnsbetrag, der als
Bezugsgröße für Preis, Verzinsung und Tilgung dient.

\break 
Charakteristische Leistungen und Gegenleistung von der \underline{Emission} (Erstausgabe) bis zur Rücknahme:
\begin{itemize}
	\item Leistung: Zum Emissionszeitpunkt $ i=0 $ zahlt die Investorin $ C_0N $ Mann nennt $ C_0 $ den \underline{Emissionskurs}
	\item Gegenleistung:
		\begin{itemize}
			\item Während der Laufzeit zahlt der Emittent (z.B Bank, Statt Unternehmen, ... ) Zinsen (in diesem Kontext auch 
				\underline{Kuponzahlungen} genannt) der Höhe $ p N $ pro Jahre. Dabei ist $ p $ der (jähliche) Zinssatz.
			\item Am Ende des letzten Laufzeitjahres zahlt der Emittent eine Tilgung der Höhe $ C_n N $.  Man nennt $ C_n $ den
				\underline{Rücknahmekurs}
		\end{itemize}
\end{itemize}

Wenn die Rücknahme zum Nennwert erfolgt ( $ C_n = 100 \%) $ sagt man auch "\underline{zu pari}" Auch $ C_n> 100 \% $ is üblich. \break
Auch $ p=0 $ ist möglich. In diesem Fall handelt es sich um eine \underline{Nullkuponanleihe}.

\subsection*{Beispiel 5.1} Bei Nennwert $ N = 200 $, Emissionskurs $ C_0 = 96\% $ Laufzeit $ n=7 $, Zinssatz $ p = 8\% p.a. $ und den 
Rücknahmekurs $ C_n= 103 \% $ zahlt die Investor zu Beginn $ 192 $ erhält am Ende jedes $ i $-ten Jahres $ 16 \; (i \in \left\{ 1 , \cdots,  6 \right\}  $ und erhält am Ende des letzten Jahres $ 16 + 206 = 222 $. Die \underline{Rendite} (oder der \underline{Effektivzinssatz})
einer Anliehe ist der der Jahreszinssatz, für den die Leistung der Investor (also der Kaufpreis) finanzmathematische äquivalent zu den gegenleistungen des Emittenten ist. \break
Die rendite $ r_{eff} $ bei Laufzeit $ n $, Emissionskurs $ C_0 $, Rücknahmekurs $ C_n $ und Zinssatz $ p	 $ bestimmt man anhand der Gleichung: 
$$  \right) C_0 = \frac{C_n}{\left( 1 + r_{eff} \right)^{n} }   +\frac{P}{\left( 1 + r_{eff}\right)  }   \frac{\left( 1 + r_{eff} \right)^{n}-1}{r_{eff}}  $$
\end{document}
