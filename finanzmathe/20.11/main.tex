\smalltitle[1.19]{Bemerkung}
Sei $ \left( \Omega , \mathcal{P} (\Omega), \mathbb{P}  \right) $ eine Wahrscheinlichkeitsraum mit $ \left| \Omega  \right| < \infty $ und
$ \mathcal{G} $ eine $\sigma$-Algebra  auf $ \Omega  $. Wenn $ \xi : \Omega \to \mathbb{R} $ unabhängig von $ \mathcal{G} $ ist, dann 
gilt $ \forall A \in \mathcal{G} $ und $ \forall x \in \xi (\Omega) $, dass $ \mathcal{P} \left[ \left\{ \xi = x \right\} \cap A \right]
= \mathbb{P}  \left[ \left\{ \xi = x \right\} \right] \mathbb{P} [A]$. 

\para{III.2}{Bedingte Erwartung}
In diesen Abschnitt setzen wir einen Wahrscheinlichkeitsraum $ \left( \Omega , \mathcal{P} (\Omega), \mathbb{P}  \right) $  mit 
$ \left| \Omega  \right| < \infty $ und $ \forall \omega  \in \Omega : \mathbb{P} [{ \omega }] > 0 $ voraus. 

\smalltitle[]{Wiederholung:}
Betrachte $ m \in \mathbb{N}, x_1 , \cdots,  x_m \in \mathbb{R} \text{ und } X : \Omega  \to \left\{ x_1 , \cdots,  x_m \right\} $. Dann 
ist der Erwartungswert von $ X $ gegeben durch 
$$ \mathbb{E} \left[ X \right] = \sum_{\omega \in \Omega } X (\omega ) \mathbb{P} [\{\omega \}] = \sum_{j = 1}^{m} x_j \mathbb{P} 
\left[ X = x_j \right] $$

\smalltitle[2.1]{Beispiel}
Wir betrachten einen Würfelwurf, also $ \Omega = \left\{ 1, 2,3,4,5,6 \right\} \text{ und }  \mathbb{P} [\{ \omega  \}] = \frac{1}{6}, 
\omega  \in \Omega $ mit $ X : \Omega \to \left\{ 1, 2,3,4,5,6 \right\}, X (\omega) = \omega   $, beschreiben wir die geworfene Augenzahl. 
Dann ist $ \mathbb{E} [X] = \sum_{\omega = 1}^{6} \omega \cdot \frac{1}{6} = \frac{21}{6} = 3,5 $ .\\
Man kann $ \mathbb{E}  $ als die "beste" Schätzung von $ X $ ohne weitere Informationen interpretieren. 

\smalltitle[]{Wiederholung:}
Wenn $ D \in \Omega  $ ist mit $ \mathbb{P} [D] >0 \left( \iff D \neq \emptyset \text{ hier }  \right) $ und $ A \subseteq \Omega  $
dann ist: 
$$ \mathbb{P} \left[ A | D \right] := \frac{ \mathbb{P} \left[ A \cap D \right]}{ \mathbb{P} \left[ D \right]}  $$
die elementare bedingte Wahrscheinlichkeit. Die Abbildung $ \mathbb{P} \left[ \cdot | D  \right] : \mathcal{P} (\omega) \to [0,1] $ ist 
ein Wahrscheinlichkeitsmaß.

\smalltitle[2.2]{Bsp}
Fortsetzung von Bsp. 2.1.  Für $ \omega \in \left\{ 1, 3, 5 \right\} $ gilt:
$$ \mathbb{P} \left[ \{\omega \} | \left\{ 1, 3, 5 \right\} \right] = \frac{ \mathbb{P} \left[ \left\{ \omega  \right\} \cap \left\{ 1,3,5 \right\} \right]}{ \mathbb{P} \left[ \left\{ 1,3,5 \right\} \right]} = \frac{\frac{1}{6} }{\frac{3}{6} } = \frac{1}{3} $$

\begin{ibox}[2.3]{Definition}{CDefinition}
    Die IndikatorFunktion einer Menge $ A \in \Omega  $  ist definiert als 
		$$ \mathbb{1}_A : \Omega  \to \mathbb{R} , \mathbb{1}_{A} = \begin{cases*}
			1,  & $\omega \in A$, \\ 
			0, & $\omega \notin A $
		\end{cases*}
		 $$
\end{ibox}
		 Für $ A, D \subseteq \Omega   $ mit $ \mathbb{P} \left[ D \right]> 0 $ gilt $ \mathbb{E} [ \mathbb{1}_A ] = \mathbb{P} \left[ A \right]$ und 
		 $$ \frac{ \mathbb{P} \left[ A \cap D \right]}{ \mathbb{P} \left[ D \right]} = \frac{ \mathbb{E} \left[ \mathbb{1}_{A} \mathbb{1}_{D} \right]}{ \mathbb{P} \left[ D \right]}  $$
		 
\begin{ibox}[2.4]{Definition}{CDefinition}
    Sei $ X : \omega  \to \mathbb{R} $ eine Zufallsvariable und sei $ D \subseteq \Omega  $ mit $ \mathbb{P} \left[ D \right] >0 $. Dann
		heißt die reelle Zahl  
		$$ \mathbb{E} \left[ X | D \right] = \frac{ \mathbb{E} \left[ X \mathbb{1}_D \right]}{ \mathbb{P} \left[ D \right]}  $$
		\underline{elementare bedingte Erwartung}
\end{ibox}

\smalltitle[2.5]{Bemerkung}
Seien $ m \in \mathbb{N}, x_1 , \cdots,  x_m \in \mathbb{R} $  und $ X : \Omega  \to \left\{ x_1 , \cdots,  x_m \right\} $. Sei $ D \in 
\Omega $ und $ \mathbb{P} \left[ D \right] >0 $. Dann ist 
\begin{align*}
	\mathbb{E} \left[ X | D \right] &= \sum_{\omega \in \Omega } X (\omega) \mathbb{1}_{D} (\omega) \frac{\mathbb{P} \left[ \left\{ \omega  \right\} \right]}{\mathbb{P} \left[ D \right]}  =  \sum_{\omega \in \Omega }X (\omega) \frac{\mathbb{P} \left[ \left\{ \omega  \right\} \cap D \right]}{ \mathbb{P} \left[ D \right]} \\ 
	&= \sum_{\omega \in \Omega } X (\omega) \mathbb{P} \left[ \left\{ \omega  \right\} | D \right] = \sum_{j=1}^{m}x_j \mathbb{P} \left[ X = x_j | D \right]
\end{align*}

\smalltitle[]{Erinnerung:}
Zu jeder $\sigma$-Algebra $ \mathcal{G} $ auf $ \Omega  $ gibt es eine (eindeutige) Partition mit endliche vielen Elementen, die $ \mathcal{G} $ erzeugt (wegen $ \left| \omega  \right| < \infty $ und Proposition III.1.10)

\begin{ibox}[2.7]{Definition}{CDefinition}
    Sei $ X : \Omega  \to \mathbb{R} $ eine Zufallsvariable. Sei $ \mathcal{ G} $ eine $\sigma$-Algebra auf $ \Omega  \text{ und }  
		\mathcal{D} (y) = \left\{ D_1 , \cdots, \; D_n \right\}$. Die \underline{bedingte Erwartung} $ \mathbb{E} \left[ X | Y \right] $ ist
		die Zufallsvariable $ \mathbb{E} \left[ X | Y \right]: \Omega  \to \mathbb{R} $, 
\begin{align*}
	\mathbb{E} \left[ X | Y \right] (\omega) &:= \begin{cases*}
		 \mathbb{E} \left[ X | D_1 \right] , \; \; \; \; &falls  $\omega  \in D_1$  \\   
		 \vdots \\
		 \mathbb{E} \left[ X | X_n \right], & falls $ \omega \in D_n $ 
	\end{cases*}\\
																					 &= \sum_{i = 1}^{n}\mathbb{E} \left[ E | D_i \right] \mathbb{1}_{D_i}(\omega)
\end{align*}
\end{ibox}
Beachte: Die elementare bedingte Erwartung ist eine reelle Zahl, nährend die bedingte Erwartung $ \mathbb{E} \left[ X | \mathcal{G} \right] $ (vgl. Def 2.7) eine $ \mathcal{G}$-messbare Abbildung 

\smalltitle[2.8]{Bsp}
Fortsetzung von Bsp 1.16, 2.1, 2.6 mit $ \mathcal{G} = \sigma (\xi) $ also $ \mathcal{D} ( \mathcal{G}) = \left\{  \left\{ 1,3,4 \right\},
\left\{ 2,4,6 \right\} \right\} $  ist 
$$ \mathbb{E} \left[ X | \mathcal{G} \right] (\omega) = \begin{cases*}
	\mathbb{E} \left[ X | \left\{ 1,3,5 \right\} \right] = 3, & falls $ \omega  \in \left\{ 1,3,5 \right\} $  \\
	\mathbb{E} \left[ X | \left\{ 2,4,6 \right\} \right] = 4, &falls $ \omega  \in \left\{ 2,4,6 \right\} $ 
\end{cases*}
 $$
Die bedingte Erwartung $ \mathbb{E} \left[ X | \mathcal{G} \right] $ kann als die "beste" Schätzung von $ X $ (der Augenzahl)	unter der 
Information $ \mathcal{G}	$ (gerade oder ungerade) interpretiert werden.

Einige wichtige Eigenschaften der bedingten Erwartung: 

\smalltitle[2.9]{Proposition}
Seien $ X, Y : \Omega  \to \mathbb{R} $ Zufallsvariablen, $ a,b \in \mathbb{R} $ und $ \mathcal{G}, \mathcal{H}  $ $\sigma$-Algebren auf 
$ \Omega  $. Dann gelten:
\begin{enumerate}[label=\alph*)]
	\item Linearität: $ \mathbb{E} \left[ aX + bY | \mathcal{G} \right] = a \mathbb{E} \left[ X | \mathcal{G} \right] + b \mathbb{E} \left[ Y | \mathcal{G}\right] $ 
	\item Monotonie: $ X \leq Y \implies \mathbb{E} \left[ X | \mathcal{G} \right] \leq \mathbb{E} \left[ Y | \mathcal{G} \right] $ 
	\item $ X  $ $ \mathcal{G}$-messbar $ \implies \mathbb{E} \left[ X | \mathcal{G} \right] = X $  
	\item $ X $ unabhängig von $ \mathcal{G}  \implies \mathbb{E} \left[ X  | \mathcal{G} \right] \mathbb{E} \left[ X \right]$ 
	\item Turmeigenschaft: $ \mathcal{H} \subseteq \mathcal{G} \implies \mathbb{E} \left[ \mathbb{E} \left[ X | Y \right] | \mathcal{H} \right] = \mathbb{E} \left[ X | \mathcal{H} \right] $ und $ \mathbb{E} \left[ \mathbb{E} \left[ X | \mathcal{H} \right]| \mathcal{G} \right] = \mathbb{E} \left[ X | \mathcal{H} \right] $ 
	\item $ Y \; \mathcal{G}$-messbar $ \implies \mathbb{E} \left[ X Y | \mathcal{G} \right] = Y \mathbb{E} \left[ X | \mathcal{G} \right] $
	\item Jensen-Ungleichung: $ f : \mathbb{R} \to \mathbb{R}  $ konvex $ \implies \mathbb{E} \left[ f (x) | \mathcal{G} \right] \geq 
		f \left( \mathbb{E} \left[ X | \mathcal{G} \right] \right)$ 
\end{enumerate}

\underline{Beweis:} \begin{enumerate}[label=\alph*)]
	\item Übungsaufgabe
	\item Es gelte $ \forall \omega  \in \Omega  $, dass $ X (\omega) \leq Y (\omega) $. Sei $ \mathcal{D} (\mathcal{G}) =  \left\{ D_1 , \cdots,  D_n \right\}$. Es gilt $ \forall i \in \left\{ 1 , \cdots,  n \right\} $, dass 
		$$ \mathbb{E} \left[ X | D_i \right] = \sum_{\omega  \in \Omega } X (\omega) \mathbb{P} \left[ \left\{ \omega  \right\} | D_i \right] \leq \sum_{\omega  \in \Omega } Y (\omega) \mathbb{P} \left[ \left\{ \omega  \right\} | D_i \right] = \mathbb{E} \left[ Y | D_i \right] $$
Es gilt $ \forall \omega  \in \Omega  $ dass
$$ \mathbb{E} \left[ X | \mathcal{G} \right] (\omega) = \sum_{i=1}^{n} \mathbb{E} \left[ X | D_i \right] \mathbb{1}_{D_i} (\omega) \leq \sum_{i=1}^{n} \mathbb{E} \left[ Y | D_i \right] \mathbb{1}_{D_i} (\omega) = \mathbb{E} \left[ Y | \mathcal{G} \right] (\omega)$$
\end{enumerate}

\smalltitle[2.10]{Bemerkung}
Als spezialfälle in Proposition 2.9 erhält man $ \left| \mathbb{E} \left[ X | Y \right] \right| \leq \mathbb{E} \left[ X | \mathcal{G} \right] $ (aus 2. oder 7), $ \mathbb{E} \left[ X | \mathcal{P} (\omega) \right] $ (aus 3.) und $ \mathbb{E} \left[ X | \left\{ \emptyset, \Omega \right\} \right] = \mathbb{E} \left[ X \right] $ (aus 4) sowie $ \mathbb{E} \left[ \mathbb{E} \left[ X | \mathcal{G} \right] \right] = \mathbb{E} \left[ X \right]$ (aus 4 und 5)

\begin{ibox}[2.11]{Lemma}{CLemma}
    Sei $ X : \Omega \to \mathbb{R} $ eine Zufallsvariable und $ \mathcal{G} $ eine $\sigma$-Algebra auf $ \Omega $. Dann gelten: 
		\begin{enumerate}[label=\alph*)]
			\item $ \mathbb{E} \left[ X | \mathcal{G} \right] $ ist $ \mathcal{G}$-messbar und 
			\item $ \forall A \in \mathcal{G}: \mathbb{E} \left[ X \mathds{1}_A \right] = \mathbb{E} \left[ \mathbb{E} \left[ X | \mathcal{G} \right] \mathds{1}_A \right] $ 
		\end{enumerate}
\end{ibox}
\underline{Beweis:} 
\begin{enumerate}[label=\alph*)]
	\item klar.
	\item Sei $ A \in \mathcal{G} $. Dann gilt $ \mathbb{E} \left[ X \mathds{1}_{A} \right] = \mathbb{E} \left[ \mathbb{E} \left[ X \mathds{1}_{A} | \mathcal{G} \right] \right] = \mathbb{E} \left[ \mathds{1}_{A} \mathbb{E} \left[ X | \mathcal{G} \right] \right]$ 
\end{enumerate}
Im Allgemeinen (also auch wenn $ |\Omega| = \infty $ ) definiert man die bedingte Erwartung über die Eigenschaft (i) und (ii) aus Lemma 2.11 


Außerdem kann man die bedingte Erwartung als eine $ L^2 $-Projektion auffassen (vgl. auch "beste" Schätzung von $ X $ unter der
Information $ \mathcal{G} $)

\smalltitle[2.12]{Proposition}
Seien $ X, Y : \Omega \to \mathbb{R} $ Zufallsvariablen, $ \mathcal{G} $ eine $\sigma$-Algebra auf $ \Omega $ und $ Y \; \mathcal{G}$-messbar. Dann gilt: 
$$ \mathbb{E} \left[ \left( X - \mathbb{E} \left[ X | \mathcal{G} \right] \right)^{2} \right] \leq \mathbb{E} \left[  \left( X - Y \right)^{2} \right] $$

\centerline{\includesvg[width=0.8\linewidth]{./figures/fima1.svg}}

\underline{Beweis:}
Für alle $ \mathcal{G}$-messbaren $ Z: \Omega \to \mathbb{R} $ gilt
\begin{align*}
	\mathbb{E} \left[ Z \left( \mathbb{E} \left[ X | \mathcal{G} \right] - X \right) \right] \stackrel{ \text{2.9 (6)}}{=} 
	\mathbb{E} \left[ \mathbb{E} \left[ Z X | \mathcal{G} \right] - Z X \right] = \mathbb{E} \left[ \mathbb{E} \left[ ZX | \mathcal{G}
	\right]\right] - \mathbb{E} \left[ Z X \right] \stackrel{ \text{Bem. 2.10}}{=} \mathbb{E} \left[ Z X \right] - \mathbb{E} \left[ Z X \right] = 0
\end{align*}
Verwende dies man mit 
\begin{align*}
Z &= \mathbb{E} \left[ Y - X | \mathcal{G} \right] = \mathbb{E} \left[ Y | \mathcal{G} \right] - \mathbb{E} \left[ X | \mathcal{G} \right]
= Y - \mathbb{E} \left[ X | \mathcal{G} \right] \\
	&= \mathbb{E} \left[ \left( Y - X \right)^{2} \right] = \mathbb{E} \left[ \left( Y - \mathbb{E} \left[ X | \mathcal{G} \right] + \mathbb{E} \left[ X | \mathcal{G} \right] - X \right)^{2} \right] \\
	&=\mathbb{E} \left[ \left( Z + \left( \mathbb{E} \left[ X | \mathcal{G} \right] - X \right) \right)^{2} \right] \cdots \cdots
\end{align*}

\para{III.3}{Prozesse}
Sie  $ \left( \Omega , \mathcal{P} (\Omega), \mathbb{P}  \right) $ ein Wahrscheinlichkeitsraum   mit 
$ \left| \Omega  \right| < \infty $ und $ N \in \mathbb{N} $ . 

\begin{ibox}[3.1]{Definition}{CDefinition}
	Eine Abbildung $ X : \left\{ 0 , \cdots, N \right\} \times \Omega \to \mathbb{R} $ heißt \underline{stochastischer Prozess}.
	Schreibweise: $ X = \left( X_n \right)_{n \in \left( 0 , \cdots, N \right)} $ 
\end{ibox}
Wenn  $ X = \left( X_n \right)_{n \in \left( 0 , \cdots, N \right)} $ stochastischer Prozess:
\begin{itemize}
	\item Für festes $ n \in \left( 0 , \cdots, N \right) $ ist $ X_n : \Omega \to \mathbb{R}, X_n (\omega) = X (n,\omega) $, eine 
		Zufallsvariable. 
	\item Für festes $ \omega \in \Omega $ ist $ X (\omega): \left\{ 0 , \cdots, N \right\} \to \mathbb{R}, X (\omega) (n) = X (n,\omega) $ 
		ein "Pfand"
\end{itemize}

\begin{ibox}[3.2]{Definition}{CDefinition}
	Eine \underline{filtration} ist eine aufsteigende Folge $ \left( \mathcal{F}_{n} \right)_{n \in \left\{ 0 , \cdots, N \right\}} $ von
	$\sigma$-Algebra d.h 
	\begin{align*}
		\forall n \in \left\{ 0 , \cdots, N \right\}:& \mathcal{F}_{n} \text{ ist eine $\sigma$-Algebra und }  \\
		\forall n \in \left\{ 0 , \cdots, N \right\}:& \mathcal{F}_{n-1} \subseteq \mathcal{F}_n 
	\end{align*}
Weiter heißt dann $ \left( \Omega, \mathcal{P} (\Omega), \left( \mathcal{F}_{n \in \left( 0 , \cdots, N \right)} \right), \mathbb{P}  \right) $ eine \underline{filtrierter Wahrscheinlichkeitsraum}.
\end{ibox}

\begin{ibox}[3.3]{Definition}{CDefinition}
    Sei $ x = \left( X_n \right)_{n \in \left\{ 0 , \cdots, N \right\}} $ ein stochastischer Prozess: Die \underline{von $ X $ erzeugte
		Filtration} $ \left( \mathcal{F}_{n}^{\times} \right)_{n \in \left\{ 0 , \cdots, N \right\}}$ ist gegeben durch :
		$$ \mathcal{F}_{n}^{\times} := \sigma \left( X_0, X_1 , \cdots, X_n \right) := \sigma \left( X_k ; k \in \left\{ 0 , \cdots, n \right\} \right), n \in \left\{ 0 , \cdots, N \right\} $$
\end{ibox}

\begin{ibox}[3.4]{Definition}{CDefinition}
	Sei $ \left( \mathcal{F}_{n}^{\times} \right)_{n \in \left\{ 0 , \cdots, N \right\}}$ eine Filtration und $ x = \left( X_n \right)_{n \in \left\{ 0 , \cdots, N \right\}} $ ein stoch. Prozess. Dann heißt $ X  $ \underline{adaptiert} ( an $ \left( \mathcal{F}_{n}^{\times} \right)_{n \in \left\{ 0 , \cdots, N \right\}}$  ) falls gilt: 
	$$ \forall n \in \left\{ 0 , \cdots, N \right\} : X_n \text{ ist } \mathcal{F}_{n} \text{ -messbar }   $$

	$ X $ heißt \underline{vorhersagbar} (bzgl. $ \left( \mathcal{F}_{n}^{\times} \right)_{n \in \left\{ 0 , \cdots, N \right\}}$ ), falls 
	$ X_0 $ konstant ist und gilt: 
	$$ \forall n \in \left\{ 1 , \cdots, N \right\} : X_n \text{ ist } \mathcal{F}_{n-1} \text{ -messbar }   $$
\end{ibox}

\para{III.4}{Martingale}
In diesem Abschnitt setzen wir einen filtrierten Wahrscheinlichkeitsraum $ \left( \Omega, \mathcal{P} (\Omega), \left( \mathcal{F}_{n \in \left( 0 , \cdots, N \right)} \right), \mathbb{P}  \right) $ mit $ \left| \Omega \right| < \infty $ und $ \forall \omega \in \Omega : 
\mathbb{P} \left[ \left\{ \omega \right\} \right] > 0 $ voraus.

\begin{ibox}[4.1]{Definition}{CDefinition}
	Ein stochastischer Prozess $ X = \left( X_n \right)_{n \in \left\{  0 , \cdots, N \right\}} $ heißt \underline{Martingal} (bzw.
	\underline{Submartingal} bzw. \underline{Supermartingal}), falls gelten: 
	\begin{itemize}
		\item $ X $ ist adaptiert ( an $ \left( \mathcal{F}_{n} \right)^{n \in \left\{ 0 , \cdots, N \right\}} $ und 
		\item $ \forall n \in \left\{  1 , \cdots, N \right\} $ : $ \mathbb{E} \left[ X_n | \mathcal{F}_{n-1} \right] = X_{n-1} $ (bzw. 
			$ \mathbb{E} \left[ X_n | \mathcal{F}_{n-1} \right] \geq  X_{n-1}$ bzw. $ \mathbb{E} \left[ X_n | \mathcal{F}_{n-1} \geq X_{n-1} \right] $ 
	\end{itemize}
\end{ibox}

