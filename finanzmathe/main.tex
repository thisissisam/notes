\documentclass[11.5 pt, a4paper]{memoir}

\usepackage[ngerman]{babel}
\usepackage{bookmark}
\usepackage{amsmath}
\usepackage{amssymb}
\usepackage{amsthm}
\usepackage[T1]{fontenc}
\usepackage{makeidx}
\usepackage{enumitem}
\usepackage{mathtools}
\usepackage{upgreek}

\usepackage{microtype}
\usepackage{svg}
\usepackage{parskip}
\usepackage{hyperref} % Make TOC clickable

\usepackage{lipsum}
\usepackage{fancyhdr} % Heading customization
\usepackage{geometry} % Adjust page padding 
\usepackage{adjustbox}

\usepackage{xcolor}
\usepackage[most]{tcolorbox}

\definecolor{Black}{HTML}{292939}
\definecolor{CTheorem}{HTML}{FFFFFF}
\definecolor{CLemma}{HTML}{FDFFED}
\definecolor{CDefinition}{HTML}{EFF9F0}
\definecolor{DarkGray}{HTML}{6B6969}
\definecolor{ImportantBorder}{HTML}{F14747}


% Set up page layout
\geometry{
    a4paper,
    left=2cm,
    right=2cm,
    top=2cm,
    bottom=2cm
}
\linespread{1.1}
\makeindex

\pagestyle{fancy}
\fancyhf{} % Clear headre/footer
\fancyheadoffset[LE,RO]{0pt} % Adjust headsep for page number and title
\fancyhead[RE,LO]{}
\fancyhead[RE,RO]{\rightmark} % Add pape header title
\fancyhead[LE,LO]{\thepage}  % Add page number to the left
\renewcommand{\headrulewidth}{0pt} % Delete the line in the header 

% Define the \para command
\newcommand{\para}[2]{
    \clearpage % Start on a new page
    \thispagestyle{empty}% removes the top left page number and top right chapter name
    \begin{center} % Center the chapter title
        \vspace*{11em}
        \Huge   \bfseries \S #1 \quad  #2 % Chapter title with symbol and counter
    \end{center}
    \vspace{8em}
    \addcontentsline{toc}{chapter}{#1.\hspace{0.6em} #2} % Add chapter to table of contents
    \fancyhead[RE,RO]{#2} % Add pape header title
}

\newcommand{\nsec}[2]{
    \section*{\large  #1 \hspace{0.3em} #2}
    %\phantomsection
    \addcontentsline{toc}{section}{#1 \hspace{0.3em} #2}
}
\newcommand{\smalltitle}[2][]{
    %\phantomsection
    \subsubsection*{ #1 \hspace{0.2em} #2}
}
\newtcolorbox{ibox}[3][]{
	enhanced jigsaw,
	colback=#3,
	colframe=DarkGray,
	coltitle=Black,
	drop shadow,
	title={\textbf{#1\hspace{0.4em}#2}},
	%before skip = 1em,
	attach title to upper,
	after title={: \quad  },
}
\renewcommand{\d}[1]{\,\mathrm{d} #1}


\title{Finanzmathe Mitschrift vom 24.10.24}
\author{Sisam Khanal}
\date{\today}
\begin{document}
\raggedright
\color{Black}
\maketitle


\smalltitle[II.1]{Einführung des Einperiodenmodells} 
Sei $ d,p \in \mathbb{N} $. Wir betrachten einen Finanzmarkt mit $ d+1 $ Wertpapieren. Beim Einperiodenmodell gibt es genau zwei Zweitpunkt, den Anfangszeitpunkt $ 0 $ und der Endzeitpunkt $ 1 $. Es kann nur zum Zeitpunkt $ 0 $ gehandelt werden
wobei man die Preise zum Zeitpunkt $ 0 $  kennt, aber i.A noch nicht klar ist, welches von $ l $ Szenarien für die Preise zum Zeitpunkt $ 1 $ eintreten wird. \break

Preisvektor zur Zeit $ 0 $ :
$$ \overline{ S_0} = \left( S_0^0, S_0^1 , \cdots,  S_0^d \right)^T = \left( S_0^0, S_0^T \right)^T \in \mathbb{R}^{d+1}  $$
wobei $ S_0^i $ der Preis des $ i $-ten Wertpapiers zur Zeit $ 0 $ für $ i \in \left\{ 0, 1 , \cdots,  d \right\}  $ ist.\break

Sei $ \left( \Omega , \mathcal{F}, \mathbb{P}  \right)$ ein endlicher Wahrscheinlichkeitsraum mit 
\begin{align*}
	&\left| \Omega  \right| = l,  \Omega  = \left\{ \omega_1 , \cdots, \omega_l \right\}, \\
	&	\mathcal{F} = 2^{\Omega } = \mathcal{P} \left( \Omega  \right),\\
	&	\mathbb{P} : \mathcal{F} \to \left[ 0, 1 \right],\\
	&	\mathbb{P} [A] = \sum_{\omega_k \in A} \mathbb{P} \left[ \left\{ \omega_k \right\}  \right]  \text{ für  } A \subseteq \Omega,\\
	&	\mathbb{P} \left[ \left\{ \omega_k \right\}  \right]  > 0 \;  \forall k \in \left\{ 1 , \cdots, l \right\} 
\end{align*}

Zufallsvektor der Preise zur Zeit $ l $ :
$$ \overline{ S_1} = \left( S_1^0, S_1^1 , \cdots,  S_1^d\right) ^{T} = \left( S_1^0, S_1^T \right)^{T} : \Omega \to \mathbb{R}^{d+1}  $$
wobei $ S_1^i \left( \omega_k \right) \in \mathbb{R} $ der Preis der $ i $-ten Wertpapiers zur Zeit $ 1 $ unter Szenario $ k $ ist für $ i \in \left( 0,1 , \cdots,  d \right) , k \in \left( 1 , \cdots,  l \right)  $ 

Alternative können wir die Preise zur Zeit $ 1 $ als eine Preismatrix auffassen:

$$ S_1 = \left[ \;  \overline{ S_1} (\omega_1) \;  \overline{ S_1} (\omega_2) \; \cdots \;  \overline{ S_1} (\omega_l) \;  \right] =  \begin{pmatrix}
	S_1^0 (\omega_1) & S_1^0 (\omega_2) & \cdots & S_1^0 (\omega_l) \\
	S_1^1 (\omega_1) & S_1^1 (\omega_2) & \cdots & S_1^1 (\omega_l) \\
	\vdots & \vdots & & \vdots \\
	S_1^d (\omega_1) & S_1^d (\omega_2) & \cdots & S_1^d (\omega_l) 
\end{pmatrix} \in \mathbb{R}^{(d+1) \times l}
 $$
Wir nehmen an, dass das Orte Wertpapiers eine Anleihe mit fester Verzinsung $ r \geq 0 $ (Bankkonto) ist mit 

$$ S_0^0 = 1 \text{ und }  \forall \omega  \in \Omega : S_1^{0}(\omega ) = 1 + r $$

Diskontierte Preise: 
$$ X_0 = \left( X_0^1 , \cdots,  X_0^{d} \right)^{T}, X_1 = \left( X_1^1 , \cdots,  X_1^{d} \right)^{T}, \Delta X_1 = \left(\Delta X_1^1 , \cdots, \Delta X_1^{d} \right)^{T},  $$
$$ \text{ mit } X_0^i = S_0^i, X_1^i = \frac{S_1^i}{1+r} , \Delta X_1^i = X_1^{i} - X_0^i, i \in \left\{ 1 , \cdots,  d \right\}  $$
Zum Zeitpunkt $ 0 $ nählt man ein Portfolio.

\begin{ibox}[1.1]{Definition}{CDefinition}
    Eine Handelsstrategie oder ein Portfolio (im Einperiodenmodell) ist ein Vektor 
		$$ \overline{ H} = \left( H^0, H^1 , \cdots,  H^d \right)^{T} = \left( H^0, H^T \right)^T \in \mathbb{R}^{d+1}  $$
\end{ibox}
Bei einer Handelsstrategie $ \overline{ H} = \left(  H^0, H^1 , \cdots,  H^d \right)^{T}  $ beschreibt $ H^{i} \in \mathbb{R} $ die Stückzahl von Wertpapier $ i  $ in Portfolio Zwischen den beiden Zeitpunkten (für $ i \in \left\{ 0, 1 , \cdots,  d \right\}  $. Dabei ist $ H^{i} $ nicht zufällig. \\

Falls $ H^0 < 0 $ : Kreditaufnahme, \\
Falls $ H^{i} < 0 $ für ein $ i \in \left\{ 1 , \cdots,  d \right\}  $ : Leerverkauf (short sell)

\begin{ibox}[1.2]{Definition}{CDefinition}
    Sei $ \overline{ H} \in \mathbb{R}^{d+1}  $. Der Wert der Handelsstrategie $ \overline{ H}  $ ist zur Zeit $ 0 $ durch:
		$$ V_0^{ \overline{ H} } = \left< \overline{ S_0}, \overline{ H}   \right> = \overline{ S_0^T} \overline{ H} = \sum_{i=0}^{d} S_0^i H^i   $$
	Und zur Zeit $ 1 $ durch die Zufallsvariable $ V_1^{ \overline{ H} } : \Omega \to \mathbb{R} $,
	$$ V_1^{ \overline{ H} } = \left< \overline{ S_1} (\omega ), \overline{ H}   \right> =  \sum_{i=0}^{d} S_0^i (\omega ) H^i   $$
	definiert. Weiter definieren wir die diskontierten Werte als 
	$$ D_0^{ \overline{ H} } = V_0^{ \overline{ H} } \text{ und }  D_1^{ \overline{ H} } = \frac{V_1^{ \overline{ H} }}{1 +r} $$
\end{ibox}

\begin{ibox}[1.3]{Lemma}{CLemma}
    Sei $ \overline{ H} = \left( H^0, H^T \right)^T \in \mathbb{R}^{d+1}  $ eine Handelsstrategie. Dann gilt 
		$$ D_1^{ \overline{ H} } = D_0 ^{ \overline{ H} } + \left< \Delta X_1, H \right> $$
\end{ibox}
Beweis: Übung


\smalltitle[]{Beispiel 1.4}
Seien $ d =1 , l=1, p \in (0,1), \mathbb{P} \left[ \left\{ \omega_1 \right\}  \right] = p, \mathbb{P} \left[ \left\{ \omega_2 \right\}  \right] = 1-p, S_0^{1} = 1, S_1^1 (\omega_1) = 2, S_1^1 (\omega_2) = \frac{1}{2}  $
\break
Sei $ r=2 $ und wähle die Handelsstrategie $ \overline{ H} (1,-1)  $ Dann ist,

\begin{align*}
	V_0^{ \overline{ H} } &= 1 \cdot 1 + 1 \cdot (-1) = 0, \\
	V_1^{ \overline{ H} } (\omega_1) &= (1+r) \cdot 1 + S_1^1 (\omega_1) \cdot (-1) = 1, \\
	V_1^{ \overline{ H} } (\omega_2) &= (1+r) \cdot 1 + S_1^1 (\omega_2) \cdot (-1) = \frac{5}{2}
\end{align*}



\smalltitle[II.2]{No-Arbitrage und FTAP1}
\begin{itemize}
	\item Ziel: Charakterirrung vom Markt, in dem es keine Arbitrage-Möglichkeiten gibt.
	\item Hilfsmittel: äquivalente Martingallmaße
	\item Hauptresultat: First Fundamental theorem of asset pricing (FTAP1)
	\item Praktische Umsetzung: prüfe Gleichungssystem auf Lösbarkeit (unter Nebenbedingungen)
\end{itemize}

\begin{ibox}[2.1]{Definition}{CDefinition}
	Eine Handelsstrategie $ \overline{H} \in \mathbb{R}^{d+1} $ heißt \underline{Arbitragemöglichkeit} falls gelte:
	\begin{enumerate}[label=\alph*)]
		\item $ V^{ \overline{ H} }_0 \leq 0$ ,
		\item $ \forall \omega \in \Omega : V^{ \overline{H } }_{1}(\omega) \geq 0 $ und
		\item $ \exists \omega \in \Omega :  V^{ \overline{H } } _{1}(\omega) > 0  $ 
	\end{enumerate}
\end{ibox}
Wir sagen, es gilt \underline{No-Arbitrage (NA)}, falls keine Arbitragemöglichkeit existieren. 
\begin{ibox}[2.2]{Lemma}{CLemma}
    Folgende Aussagen sind äquivalent
		\begin{enumerate}[label=\alph*)]
			\item Es gibt eine Arbitragemöglichkeit.
			\item Es gibt ein $ \alpha \in \mathbb{R}^{d} $, sodass 
				\begin{itemize}
					\item $ \forall \omega \in \Omega : \left<\Delta X_1(\omega), \alpha \right> \geq 0 $  und 
					\item $ \exists \omega \in \Omega :  \left<\Delta X_1(\omega), \alpha \right> > 0  $ 
				\end{itemize}
			\item Es gibt eine Arbitragemöglichkeit $ \overline{ H} \in \mathbb{R}^{d+1}  $ mit $ V^{ \overline{ H} }_0 = 0 $ 	
		\end{enumerate}
\end{ibox}
\subsection*{Beweis} Übung
Eine Wahrscheinlichkeitsmaß $ Q $ auf $ \left( \Omega, \mathcal{F} \right)  $ können wir durch den Vektoren $ q \in \mathbb{R}^{l} $ 
mit $ q_k = Q \left[ \left\{ \omega_k \right\}  \right] , k \in \left\{ 1 , \cdots,  l \right\}  $ charakterisieren. Für eine Zufallsvariable $ Y : \Omega \to \mathbb{R} $ notieren wir mit 
$$ \mathbb{E}^{Q} \left[ Y \right] = \sum_{k=1}^{l} q_k Y \left( \omega_{k} \right)  $$
den Erwartungswert von $ Y $ bzgl. $ Q $. Für $ m \in \mathbb{N} $ und einen $ m $-dim Zufallsvektor $ Y = \left( Y^{1}, , \cdots,  
Y^{m}\right)^{T}: \Omega to \mathbb{R}^{m} $ setzen wir 
$$ \mathbb{E}^{Q} \left[ Y \right] = \left( \mathbb{E}^{Q} \left[ Y^{1} \right], \cdots,  \mathbb{E}^{Q} \left[ Y^{m} \right]  \right) ^{T}  $$
\begin{ibox}[2.3]{Definition}{CDefinition}
	Ein Wahrscheinlichkeitsmaß $ Q $  auf $ \left( \Omega, \mathcal{F} \right) $ heißt \underline{risikoneutral} oder 
	\underline{Martingalmaße}, falls $ \forall i \in \left\{ 1 , \cdots,  d \right\}  $ :
	$$ \mathbb{E}^{Q} \left[ X_1^{i} \right] = X_{0}^{i}$$
\end{ibox}

\begin{ibox}[2.4]{Lemma}{CLemma}
    Sei $ Q $ ein Wahrscheinlichkeitsmaß auf $ \left\{ \Omega, \mathcal{F} \right\}  $ und $ q_k = Q \left[ \left\{ \omega_k \right\}  \right] , k \in \left\{ 1 , \cdots,  l \right\}  $. Es bezeichne $ e_i $ den $ i$-ten Einheitsvektor im $ \mathbb{R}^{d} $. Dann sind äquivalent:
		\begin{enumerate}[label=\alph*)]
			\item $ Q $ ist eine Martingalmaße
			\item $ \forall i \in \left\{ 1 , \cdots,  d \right\} : \sum_{k=1}^{l} q_k \left< \Delta X_1 (\omega_k), e_i \right>  = 0$ 
			\item $ \forall \alpha \in \mathbb{R}^{d} : \sum_{k=1}^{l} q_k \left< \Delta X_1 (\omega_k), \alpha \right>  = 0$ 
		\end{enumerate}
\end{ibox}
\subsection*{Beweis} Übung
\begin{ibox}[2.5]{Definition}{CDefinition}
	Ein Wahrscheinlichkeitsmaß $ Q $ auf $ \left( \Omega, \mathcal{F} \right)  $ heißt (zu $ \mathbb{P} $) \underline{äquivalentes Wahrscheinlichkeitsmaß}, wenn $ \forall k \in \left\{ 1 , \cdots,  l \right\} : q_k = Q \left[ \left\{ \omega_k \right\}  \right] > 0  $  
\end{ibox}
Wir hatten angenommen, dass $ \forall k \in \left\{ 1 , \cdots,  l \right\} : \mathbb{P} \left[ \left\{ \omega_k \right\}  \right] > 0 $  ein $ \mathbb{P} $ äquivalentes Wahrscheinlichkeitsmaß auf $ \left\{ \Omega, \mathcal{F} \right\}  $ ist, dann stimmen also die Mengen der möglichen Szenarien unter $ \mathbb{P} \text{ und }  Q $ überein (aber die genauen Wahrscheinlichkeiten sind in der Regel unterschiedlich)

\begin{ibox}[2.6]{Definition}{CDefinition}
	Ein Wahrscheinlichkeitsmaß $ Q $ auf $ \left\{ \Omega, \mathcal{F} \right\}  $ heißt \underline{äquivalentes Martingalmaße (ÄMM)}
	falls $ Q $ risikoneutral und äquivalent ist. Wir definieren 
	$$ \mathcal{P} = \left\{ Q : Q \text{ ist ÄMM }  \right\}  $$
	
\end{ibox}

\subsection*{Beispiel 2.7:} Seien $ d = 1, l = 2, p \in (0,1), \mathbb{P} \left[ \left\{ \omega_1 \right\}  \right] = p, \mathbb{P} \left[ \left\{ \omega_2 \right\}  \right] = 1 - p, S_0^{1} = 1, S_1^{1}(\omega_1) = 2, S_1^{1}(\omega_2)= \frac{1}{2} \text{ und }  r = 0  $. Wir versuchen, ein ÄMM zu finden. Wenn $ Q $ ein ÄMM (charakterisiert durch $ q \in \mathbb{R}_2 $ ist, dann müssen gelten:  
$ q_1 > 0, q_2 >0 $ 
 $$ \sum_{k=1}^{2} q_k S_1^{1} (\omega_k) = q_1 \underbrace{ S_1^{1} (\omega_1)}_{=2} + q_2 \underbrace{ S_1^{1} (\omega_2)}_{=\frac{1}{2} } = 1 $$
LGS lösen: 
\begin{align*}
	q_1 + q_2 &= 1 \\
	 2q_1 + \frac{1}{2} q_2 &= 1 
\end{align*}

Andere Schreibweise: 
\begin{align*}
\begin{pmatrix}
	1 & 1 \\
	2 & \frac{1}{2} 
\end{pmatrix} 
\begin{pmatrix}
	q_1 \\ q_2 
\end{pmatrix}
 = \begin{pmatrix}
	 1 \\ 1  
 \end{pmatrix} \implies \begin{pmatrix}
 	q_1 \\ q_2 
 \end{pmatrix}
 = \begin{pmatrix}
 	 \frac{2}{3} \\ \frac{1}{2} 
 \end{pmatrix}
\end{align*}
Im ersten fundamentalsatz wird ein Zusammenhang zwischen der Existenz von ÄMMs und NA hergestellt:

\begin{ibox}[2.8]{Satz (FTAP1)}{CTheorem}
	Es gilt: $ NA \iff \mathcal{P} \neq \emptyset $.
	Andereseits gilt: Ein Vektor $ q \in \mathbb{R}^{l} $ definiert genau dann ein ÄMM (via $ Q \left[ \left\{ \omega_k \right\}  \right] : = q_k, k \in \left\{ 1, , \cdots,  l \right\}  $ , wenn $ k \in \left\{  1, \cdots,  l \right\} : q_k > 0 $ gilt, und $ q $ 
	eine Lösung des Systems: $ \forall i \in\left\{ 1 , \cdots,  d \right\}  $ 

	\begin{align*}
		q_1 + q_2 + \cdots + q_{l} &= 1 \\
		\frac{1}{1+r} \left( S_1^{i} (\omega_1) q_1 +   S_1^{i} (\omega_2) q_2  + \cdots +   S_1^{i} (\omega_l) q_l \right) &= S_0^{i}
	\end{align*}
ist. Um zu überprüfung, ab NA gilt, kann man wegen FTAP1 also testen, ob 
$$ S_1 q = (1+r) \overline{ S_0}  $$
eine Lösung $ q \in \mathbb{R}^{l} $ mit $ \forall k \in \left\{ 1 , \cdots,  l \right\} : q_k > 0 $ besitzt
\end{ibox}
Für den Beweis des FTAP1 benötigen wir folgenden Trennungssatz (in $ \mathbb{R}^{l} $ )

\begin{ibox}[2.9]{Satz (Trennungssatz)}{CTheorem}
	Sei $ n \in \mathbb{N} $ 
	\begin{enumerate}[label=\alph*)]
		\item Sei $ C \subseteq \mathbb{R}^n $ abgeschlossen, konvex und nichtleer mit $ 0 \notin C $. Dann existieren $ y \in C $ und 
			$ \delta > 0 $ sodass $ \forall x \in C : \left<x,y \right> \geq \delta > 0 $ 
		\item Sei $ K \subseteq \mathbb{R}^n $ kompakt, konvex und nicht leer und sei $ U \subseteq \mathbb{R}^n $ eine linearen Unterraum
			mit $ K \cap U = \emptyset $. Dann existiert $ y \in \mathbb{R}^n $ sodass
			\begin{itemize}
				\item $ \forall x \in K : \left<x, y \right> >0$,
				\item $ \forall x \in U : \left<x,y \right> = 0 $ 
			\end{itemize}
	\end{enumerate}
\end{ibox}

\smalltitle[2.10]{Proposition}
Die Menge $ \mathcal{P} $ aller ÄMMs ist konvex. Insbesondere gilt:
$$ \left| \mathcal{P} \right| \implies | \mathcal{P}| = \infty $$
Beweis: Übung

\smalltitle[II.3]{Arbitragefreie Preise}
Wir Identifizieren im Folgenden ein Derivat mit seiner Auszahlung zur Zeit $ T=1 $ 

\begin{ibox}[3.1]{Definition}{CDefinition}
    Ein \textit{Derivat} ist eine Zufallsvariable $ \xi : \Omega \to \mathbb{R} $. Für $ x, y \in \mathbb{R} $ notieren wir 
		$$ \left( x - y \right)^{+} = \text{ max } \left\{ x - y, 0 \right\}  $$
\end{ibox}

\smalltitle[3.2]{Beispiel}
Beispiele für Derivate:                                                                                              
\begin{enumerate}[label=\alph*)]
	\item \underline{Forward contract}: Vereinbarung, zu einem zukünftigen, festgelegten Termin $ T $ ein Gut (z.B Wertpapier) zu einem heute vereinbarten Preis $ k $ zu kaufen bzw. zu verkaufen. Im Einperiodenmodell: Forward auf Papier
		$$ i: \xi = S_1^{i} - k  $$
	\item \underline{Call-Option}: Recht, ein Gut zu festgelegt Preis $ K $ ( dem \textit{Strike} ) zu zukünftigen Zeitpunkt $T$ zu kaufen.
		Im Einperiodenmodell: Call auf Papier 
		$$ i : \xi = \left( S_1^{i} - k \right) ^{+} $$
	\item \underline{Put-Option}: Analog zu Call, aber Verkaufen. Im Einperiodenmodell: Put auf Papier
		$$ i : \xi = \left\{ k - S_1^{i} \right\} ^{+} $$
	\item \underline{Basket-Option}: Option auf einen Index von Wertpapier z.B Basket-Call-Option mit Strike $ k $ und Gewicht
		$ \alpha \in \mathbb{R}^{d} $ im Einperiodenmodell :
		$$ i : \xi = \left( K - S_1^{i} \right)  $$
	\item \underline{Basket-Option}: Option auf einen Index von Wertpapier z.B Basket-Call-Option mit Strike $ k $ und Gewicht $ \alpha
		\in \mathbb{R}^{d}$ im Einperiodenmodell :
		$$ \left(  \left< S_1, \alpha \right> - k \right) ^{+} $$
\end{enumerate}
\smalltitle[3.3]{Beispiel}
Seien $ d = 1, l =2, \mathbb{P} \left[ \left\{ \omega_1 \right\}  \right] = \frac{1}{2} = \mathbb{P} \left[ \left\{ \omega_1 \right\}  \right] , S_0^{0} = 1, S_1^{0} = 1 (r=0), S_0^{1} = 1, S_1^{1}\left( \omega_1 \right) = 2, S_1^{1} (\omega_1) = \frac{1}{2}  $.
Betrachte Call $ \xi = \left( S_1^{1} \right)^{+} $. Was ist ein angemessener Preis $ P_0 (\xi) $ für $ \xi $ zur Zeit 0? \break
\underline{Ersten Ansatz}: Preis von $ \xi $ ist mittlere Auszahlung von $ \xi $ bei wiederhaltern Spiel, d.h 
$$ p_0 (\xi) = \mathbb{E} ^{ \mathbb{P} } \left[ \xi \right] = \frac{1}{2} \cdot \left( 2 - 1 \right)^{+} + \frac{1}{2} \cdot 
\left( \frac{1}{2} -1 \right)^{+} = \frac{1}{2}  $$
Dadurch ergibt sich allerdings eine Arbitragemöglichkeit.
\break 
\underline{Zeit 0}: Verkaufe $ 6 \times $ Call zum Preis $ \frac{1}{2}  $, Kaufe $ 3 \times $ die Aktie; Startkapital 0. \\
\underline{Zeit 1} Szenario $ \omega_1 $ : Endkapital ist $ 3 \cdot 2 - 6 \cdot 1 (2-1)^{+} = 0 $ \\
Szenario $ \omega_2 $: Endkapital ist $ 3 \cdot \frac{1}{2} - 6 \cdot 1 \left( \frac{1}{2} - 1 \right) ^{+} = 1,5 $ 
\break

\textbf{Fischer Ansatz}: Preis von $\xi$ ist im Mittelwert unabhängig von $\xi$ bei wiederholtem Spiel, d.h.
\[
p_0(\xi) = \mathbb{E}^{ \mathbb{P} } \left[ \xi \right] = \frac{1}{2} \left(2 - 1\right)^+ + \frac{1}{2} \left(\frac{1}{2} - 1\right)^+ = \frac{1}{2}
\]
Dadurch ergibt sich allerdings eine Arbitragemöglichkeit:

\textbf{Zeit 0}:
\begin{itemize}
	\item Verkaufe $6 \times$ Call zum Preis $\frac{1}{2}$.
	\item Kaufe $3 \times$ die Aktie; Startkapital 0.
\end{itemize}
\textbf{Zeit 1}: 
\begin{itemize}
    \item Szenario $w_1$: Endkapital ist $3 \cdot 2 - 6 \left(2 - 1\right)^+ = 0$
    \item Szenario $w_2$: Endkapital ist $3 \cdot \frac{1}{2} - 6 \left(\frac{1}{2} - 1\right)^+ = 1.5$
\end{itemize}
Wir verwenden im Folgenden das No-Arbitrage-Prinzip zur Bewertung von Derivaten. Wähle den Preis $P_0(\xi)$ von $\xi$ so, dass sich im um $\left( S_0^{d^0}, S_1^{d^1} \right) = \left( p_0(\xi), \xi \right)$ erweiterten Modell keine Arbitragemöglichkeit ergibt.

\begin{ibox}[3.4]{Definition}{CDefinition}
Sei $\xi$ ein Derivat. Der Wert $p_0(\xi) \in \mathbb{R}$ heißt arbitragefreier/fairer Preis von $\xi$, falls das um 
$\left( S_0^{d+0}, S_1^{d+1} \right) = \left( p_0(\xi), \xi \right)$ erweiterte Marktmodell mit den Preisen 
$\left( S_0^0, \dots, S_0^d, S_0^{d+1} \right)^T$ und $\left( S_1^0, \dots, S_1^{d+1} \right)^T$ NA erfüllt.
Wir bezeichnen mit $\Pi(\xi)$ die Menge aller arbitragefreien Preise von $\xi$.
\end{ibox}
\textbf{Beachte:} Wenn $\Pi(\xi)$ nicht leer ist, dann gilt im ursprünglichen Modell NA.

\textbf{Also:} Falls NA im ursprünglichen Modell nicht erfüllt ist, dann ist $\Pi(\xi) = \emptyset$.

Im Folgenden sei $ \mathcal{P}$ bezeichnet die Menge aller ÄMMs im ursprünglichen Modell.

\begin{ibox}[3.5]{Satz}{CTheorem}
Es gelte NA. Sei $\xi$ ein Derivat. Dann ist $\Pi(\xi)$ nicht leer, und es gilt
\[
\Pi(\xi) = \left\{ \mathbb{E}^Q \left[ \frac{\xi}{1+r} \right] \middle| Q \in \mathcal{P} \right\}.
\]
\end{ibox}

\smalltitle{Beweis:}
$ " \subseteq" $: Sei $Q \in \mathcal{P}$ und $p_0(\xi) = \mathbb{E}^Q \left[ \frac{\xi}{1+r} \right]$.
Dann ist $Q$ auch ein ÄMM in dem um $\left( p_0(\xi), \xi \right)$ erweiterten Markt. FTAP1 impliziert, dass im erweiterten 
Markt NA gilt, also $p_0(\xi) \in \Pi(\xi)$.\break

$ "\supseteq"$: $p_0(\xi) \in \mathbb{R}$ ist ein arbitragefreier Preis. Der um $\left( p_0(\xi), \xi \right)$ erweiterte Modell 
erfüllt dann NA. Wende FTAP1 auf dem erweiterten Markt an. Dann existiert ein Wahrscheinlichkeitsmaß auf $ Q $ auf $ \left( \Omega , 
\mathcal{F}\right) $ sodass $\forall k \in \{1, \dots, l \} : Q \left[ \left\{ \omega_k \right\}  \right] > 0$ und
$\forall k \in \{1, \dots, d\}: \mathbb{E}^Q \left[ X_1^{i} \right] = X_0^{i} = 0$ 

Insbesondere gilt $ Q \in \mathcal{P} $ 
\[
p_0(\xi) = S_0^{d+1} = X_0^{d+1} = \mathbb{E}^{Q} \left[ X_1^{d+1} \right] =  \mathbb{E}^Q \left[ \frac{S_1^{d+1}}{1+r} \right] = \mathbb{E}^Q \left[ \frac{\xi}{1+r} \right].
\]
\smalltitle[3.6]{Beispiel} Fortsetzung von Bsp 3.3

\underline{Neuer Ansatz}: NA-Prinzip Aus Bsp 2.7 ist bekannt, dass $ q = \left( \frac{1}{3} , \frac{2}{3}  \right) ^{T}
\in \mathbb{R}^{2 }$ in diesem Marktmodell ein ÄMM $ Q $ def. Somit ist (siehe Satz 3.5)
$ p_0 (\xi) = \mathbb{E} ^{Q} \left[ \frac{S}{1+r}  \right] =  \mathbb{E} ^{Q} \left[  \left( S_1^{1} - 1 \right) ^{+}
\right] = \frac{1}{3} \left( S_1^{1} (\omega_1) -1 \right)^{+} + \frac{2}{3} \left( S_1^{1} (\omega_2) -1 \right)^{+}
= \frac{1}{3} $ ein arbitragefreier Preis. 

\smalltitle[3.5]{Proposition}
Es gelte NA. Sei $ \xi $ ein Derivat. Dann $ \emptyset \neq \Pi (\xi) \subset \mathbb{R} $ Konvex und somit ein Intervall. \\
\underline{Beweis}: Übung

Zum Folgenden wollen wir die Intervallgrenzen $ \Pi ^{X} (\xi_j) := \text{ sup } \Pi (\xi) $ und $ \Pi_{X} (\xi) $ 
für ein keivert $ \xi $ 

\begin{ibox}[3.5]{Definition}{CDefinition}
    Wir sagen, es gilt Nicht-Redundanz \textit{NR} wenn die Vektoren 
		$$ \begin{pmatrix}
			\Delta x_1^{i} (\omega_1) \\
			\vdots \\
			\Delta x_1^{i} (\omega_l) 
		\end{pmatrix}
	, i \in \left( 1 , \cdots,  d \right) 	 $$
linear unabhängig sind.		
\end{ibox}
Damit NR erfüllt sein kann, muss notwendigerweise $ d \leq l $ gelten. Falls NR nicht erfüllt ist, dann existiert
eine strikte  Teilmenge dieser Vektoren. 
		\begin{align*}
		\begin{pmatrix}
			\Delta x_1^{i} (\omega_1) \\
			\vdots \\
			\Delta x_1^{i} (\omega_l) 
		\end{pmatrix}
	, i \in \left( 1 , \cdots,  d \right) &\text{ s.d } \forall j \in \left\{ m +1 , \cdots,  d \right\} \exists 
	\alpha_1 , \cdots,  \alpha_n \in \mathbb{R} \text{ mit } \\	
	&\begin{pmatrix}
			\Delta x_1^{j} (\omega_1) \\
			\vdots \\
			\Delta x_1^{j} (\omega_l) 
		\end{pmatrix} = \sum_{i = 1}^{m} \alpha_i  \begin{pmatrix}
			\Delta x_1^{i} (\omega_1) \\
			\vdots \\
			\Delta x_1^{i} (\omega_l) 
		\end{pmatrix}
		\end{align*}
Die Vektoren (Wertpapier) für $ j \in \left\{ m +1 , \cdots,  d \right\}  $ sind also überflüssig redundant. Durch 
\textit{Wegwerfen} von redundanten Wertpapieren können wir jeden Markt zu einem Markt, in dem NR gilt reduzieren.

\smalltitle[]{Erinnerung:}
Für eine Handelsstrategie $ \overline{ H} = \left( H^{O}, H^{T} \right)^{T}  $ ist 
$$ V_{0}^{ \overline{ H} } = \left< \Delta_1, H \right> = \frac{V_1^{ \overline{ H} }}{1 +r }  $$

\begin{ibox}[3.9]{Definition}{CDefinition}
    Sei $ \xi $ ein Derivat. Ein Vektor $ \left( X, H^{T} \right) ^{T} \in \mathbb{R}^{d+1} $ mit $  x \in \mathbb{R}$
		(Anfangskapital) und $ H \in \mathbb{R}^{d}  $ (Position in Aktien) wird als \underline{Superhedging-Strategie} 
		für $ \xi $ bezeichnet, wenn 

\end{ibox}

\smalltitle[]{Vorlesung \today}
\begin{ibox}[5.3]{Definition}{CDefinition}
    Seien $ K, L \in \mathbb{R} $ mit $ K \geq L $, sei $ \lambda \in \left[ 0,1 \right]  $ und sei $ M = \lambda K + \left( 1- \lambda \right)L  $. Wir nehem an, dass zur Zeit 0 die Call-Option $ \left( S_1^1 - K \right)^{+}, \left( S_1^1 -L \right)^{+} $ zu dem Preisen $ C (K), C (L) \text{ und }  C (M) $ gehandelt werden. Der um $ \left( C (K), \left( S_1^1 - K \right)^{+} \right) , \left( C (L), \left( S_1^1 - L \right)^{+} \right)  $ erweiterte Markt NA. Dann gilt: 
		\begin{enumerate}[label=\alph*)]
			\item $ C (K) \leq C (L) $ 
			\item $ \left( 1 + r \right) \left( C (K) - C (L) \right) \leq L - K $ 
			\item $ C \left( \lambda K + (1- \lambda) L \right) \leq \lambda C (K) + (1- \lambda) C (L)  $ 
		\end{enumerate}
\end{ibox}

\para{3}{Elemente der Wahrscheinlichkeitstheorie}
Wahrscheinlichkeitstheoretische Grundlagen, um das Einperiodenmodell auf mehrere Handelszeitpunkte zu erweitern und Informationszunahme im Lauf der Zeit zu modellieren.

\smalltitle[]{Literatur:}
\begin{itemize}
	\item Kremer - Einführung in die diskrete Finanzmathe 
	\item Shiryaev - Probability-1
	\item Klenke - Wahrscheinlichkeitstheorie
	\item Mentrup und Schäffler - Stochastik
\end{itemize}

\smalltitle[III.1]{ $ \sigma $-Algebra}
Sei $ \Omega \neq \emptyset $ eine Menge. Bezeichne mit $ \mathcal{P} (\Omega) $ die Potenzmenge von $ \Omega $ 

\begin{ibox}[1.1]{Definition}{CDefinition}
	Ein Mengensystem $ G \subseteq \mathcal{P} (\Omega) $ heißt \underline{Algebra} auf $ \Omega  $, falls gelten:
	\begin{enumerate}[label=\alph*)]
		\item $ \Omega  \in G $ 
		\item $ A \in G \implies A^{c} \in G $ 
		\item $ A, B \in G \implies A \cup B \in G $ 
	\end{enumerate}
\end{ibox}
\begin{ibox}[1.2]{Definition}{CDefinition}
	Eine Algebra $ G $ auf $ \Omega  $ heißt \underline{ $ \sigma $-Algebra} auf $ \Omega $, falls zusätzlich gilt:
	$$ A_n \in G \forall n \in \mathbb{N} \implies \cup_{ n \in \mathbb{N}} A_n \in G $$
\end{ibox}
Beispiele für $ \sigma $-Algebra (und damit auch für Algebra) sind die "triviale" 	$ \sigma $-Algebra und die Potenzmenge $ \mathcal{P} (\Omega) $. \\
Wenn $ G $ eine Algebra ist, dann gilt $ \emptyset \in G $ und $ \forall A, B \in G $ sind $ A \cap B = \left( A^\complement \cup B^\complement \right)^\complement \in G $ und $ A \setminus B = A \cap B^\complement \in G $. Wenn $ G $ eine $ \sigma $-Algebra ist, dann gilt
zusätzlich $ A_n \in G \; \forall n \in \mathbb{N} \implies \cap_{n \in \mathbb{N}}A_n \in G $ 

\begin{ibox}[1.3]{Lemma}{CLemma}
    Sei $ \Omega \neq \emptyset $ eine Menge und $ \left\{ G_{\lambda} \right\}_{\lambda \in \Omega } $ eine Familie von Algebren
		(bzw. $ \sigma $-Algebra) auf $ \Omega  $. Dann ist auch $ \cap_{\lambda \in \Omega } G_{\lambda} $ eine Algebra (bzw. $ \sigma $-Algebra) auf $ \Omega  $.
\end{ibox}
Beweis: Übung

\begin{ibox}[1.4]{Definition}{CDefinition}
    Sei $ \mathcal{A} \subseteq \mathcal{P} (\Omega) $ eine Mengensystem. Wir definieren: 
		\begin{align*}
			\mathcal{B}^{ \mathcal{A}}_{\sigma} &= \left\{ G \subseteq \mathcal{P} (\Omega) : G \text{ Algebra und  } G \supseteq \mathcal{A} \right\} \\
			\mathcal{B}^{ \mathcal{A}}_{\sigma} &= \left\{ G \subseteq \mathcal{P} (\Omega) : G \; \sigma -\text{ Algebra und  } G \supseteq \mathcal{A} \right\} \\
		\end{align*}
		Weiter definieren wir die \underline{durch $ \mathcal{A} $ erzeugte Algebra}:
		$$ \alpha ( \mathcal{A}) = \bigcap_{G \in 	\mathcal{B}^{ \mathcal{A}}_{\sigma}} G $$
\end{ibox}
Lemma 13 Zeigt, dass $ \alpha (A) $ eine Algebra und $ \sigma (A) $ eine $ \sigma $ -Algebra ist.

\begin{ibox}[1.5]{Lemma}{CLemma}
    Es gelte $ |\Omega | < \infty $  Sei $ G $ ein Algebra. Dann ist $ G  $ eine $ \sigma $ -Algebra.
\end{ibox}
\underline{Beweis}: Seien $ A_n, n \in \mathbb{N} $, Elemente von $ G $ und $ B = \cup_{n \in \mathbb{N}}A_n $. Weil $ \Omega  $ 
endlich ist, hat $ G $ nur endlich viele Element und daher geht die Vereinigung nur über endlich viele verschieden Elemente von $ G $.
Per Induktion und (iii) in Def 1.1 folgt, dass $ B \in G $  


Also, Auf endlichen $ \Omega  $ sind Algebren $ \sigma $ -Algebren äquivalent. Für allgemeines $ \Omega  $ gibt es aber Algebra , die
keine $ \sigma $ -Algebra sind z.B:

\smalltitle[1.6]{Bsp.}
Sei $ \Omega = \mathbb{N}, \mathcal{A} = \left\{ A \subseteq \mathbb{N} : |A| = 1 \right\} \text{ und }  G := \alpha (A) $. Dann
gilt (Übung). 
$$ G = \left\{ B \subseteq \mathbb{N} : |B| < \infty \text{ oder } |B^\mathsf{c}| < \infty  \right\} $$.
Aus Lemma 1.2 wissen wir, dass $ G $ eine Algebra ist. Aber $ G $ ist keine $ \sigma $ -Algebra, denn $ \forall k \in \mathbb{N}:
\left\{ 2k \right\} \in G$ und (wir setzen $ \cup_{\lambda \in \emptyset} \cdots = \emptyset $ 
$$ \bigcup_{k \in \mathbb{N}} \left\{ 2k \right\} = 2 \mathbb{N} \notin G $$

\begin{ibox}[1.7]{Definition}{CDefinition}
	Sei $ \Omega  $ ein Menge. Ein Mengensystem $ D = \left\{ D_{\lambda} \right\}_{\lambda \in \Omega } $ heißt \underline{Partition}
	oder \underline{Zerlegung} von $ \Omega  $, falls gelten:
	\begin{enumerate}[label=\alph*)]
		\item $ \forall \lambda \in \Omega : D_{\lambda} \neq \emptyset $ 
		\item $ \forall \lambda_1, \lambda_2 \in \omega  $ mit $ \lambda_1 \neq \lambda_2: D_{\lambda_1} \cap D_{\lambda_2} = \emptyset $ 
		\item $ \cup_{\lambda \in \omega } D_{\lambda} = \Omega $ 
	\end{enumerate}
\end{ibox}
\smalltitle[1.8]{Proposition}
	Sei $ \Omega  $ höchstens abzählbar und $ D = \left\{ D_{\lambda} \right\}_{\lambda \in \Omega } $ eine Partition. Dann gilt 
	$$ \sigma (\mathcal{D}) = \left\{ \bigcup_{\lambda \in A} D_{\lambda} : A \subseteq \Omega  \right\}$$
	\underline{Beweis:} Sei $ \mathcal{F} := \left\{ \cup_{\lambda \in A} D_{\lambda}: A \subseteq \Omega \right\} $  \\
	 \underline{$ \supseteq $} : \;   Trivial\\

	 \underline{$ \subseteq $} : Zeige, dass $ \mathcal{F} \in \mathcal{B}^{ \mathcal{D}}_{\sigma}  $. Klar: $ \mathcal{D} \subset 
	\mathcal{F}$. Noch zu zeigen: $ \mathcal{F} $ ist $ \sigma $ -Algebra.
	\begin{enumerate}[label=\alph*)]
		\item Wegen (iii) in Def 1.7 gilt $ \Omega  = \cup_{\lambda \in \Omega } D_{\lambda} \in \mathcal{F} $ 
		\item Sei $ A \subseteq \mathcal{A} $. Es gilt (wegen (ii) in Def 1.7)
			$$ \left( \bigcup_{\lambda \in A} D_{\lambda} \right)^{\mathsf{c}} = \bigcap_{\lambda \in A}D_{\lambda}^{\mathsf{c}} = 
			\bigcup_{\lambda \in \Omega \setminus A} D_{\lambda} \in \mathcal{F} $$
		\item Seien $ \mathcal{B}_{n} \in \mathcal{F}, n \in \mathbb{N} $. Für jedes $ n \in \mathbb{N} $  gibt es dann $ A_n \subseteq \Omega  $ sodass $ \mathcal{B}_{n}
			= \cup_{\lambda \in A_n} D_{\lambda}$. Es folgt, dass 
			$$ \bigcup_{n \in \mathbb{N}} \mathcal{B}_{n} = \bigcup_{n \in \mathbb{N}} \bigcup_{\lambda \in A_n} D_{\lambda} = \bigcup_{\lambda \in \cup_{n \in \mathbb{N}}A_n} D_{\lambda} \in \mathcal{F} $$
			
	\end{enumerate}
\begin{ibox}[1.9]{Lemma}{CLemma} 
	Es gelte $ \left| \Omega  \right| \leq \infty $. Seien $ \mathcal{D} \text{ und }  \tilde{\mathcal{D}} $ zwei Partitionen mit $ \sigma (\mathcal{D}) = \sigma (\tilde{\mathcal{D}}) $. Dann gilt $ \mathcal{D} = \tilde{\mathcal{D}} $ 
\end{ibox}
\underline{Beweis: }	Betrachte, dass die Partitionen nur endlich viele Elemente haben, und daher die Darstellung aus Proposition 
1.8 gilt. Angenommen, es gibt ein $ B \in \tilde{\mathcal{D}} \text{  mit } B \notin \mathcal{D} $.
\begin{itemize}
	\item Falls es ein $ D \in \mathcal{D} $ gibt, sodass $ B \subsetneq D $, dann ist $ B \notin \sigma ( \mathcal{D}) = \sigma ( \tilde{
		\mathcal{D}}$. Widerspruch.
	\item Andernfalls gilt $ \forall D \in \mathcal{D} $, dass $ B $ keine Teilmenge von $ D $ ist.
		\begin{itemize}
			\item Falls es ein $ C \in \mathcal{D} $ gibt mit $ C \subsetneq B $, dann ist $ C \notin \sigma (\tilde{ \mathcal{D}}) 
				= \sigma ( \mathcal{D})$. Widerspruch.  
			\item Falls es kein $ C \in \mathcal{D} $ gibt, das Teilmenge von $ B $ ist, dann folgt $ B \notin \sigma ( \mathcal{D}) = \sigma
				(\tilde{ \mathcal{D}})$. Widerspruch. $ \qed $ 
		\end{itemize}
\end{itemize}

Jede $ \sigma $ -Algebra auf einem endlichen $ \Omega $ kann durch eine Partition erzeugt werden: 

\smalltitle[1.10]{Proposition}
Es gelte $ \left| \Omega  \right| < \infty $. Sei $ G $ eine $ \sigma $-Algebra auf $ \Omega  $. Dann existiert genau eine Partition 
$ \mathcal{D} $ von $ \Omega  $, sodass $ G = \sigma ( \mathcal{D}) $ 

\underline{Beweis: } Für alle $ x \in \Omega  $ sei $ \mathcal{M}_{x} := \left\{ A \in G : x \in A \right\} $ und $ A_{x} := \cap_{A \in \mathcal{M}_{x}} A $. Für $ x, y \in \Omega $ definiere $ x ~ y : \iff y \in A_{x} $. Dann ist $ ~ $ eine Äquivalenzrelation auf $\Omega$
denn: \begin{itemize}
	\item Reflexivität: Sei $ x \in \Omega  $. Es gilt $ \forall A \in \mathcal{M}_{x} $, dass $ x \in A $ also $ x \in A_{x} $. Somit 
		gilt $ x \sim x $ 
	\item Symmetrie: Seien $ x, y \in \Omega  $ mit $ x \sim y $. Dann ist $ y \in A_{x} $. Deshalb gilt $ \forall A \in G $ mit $ x \in A $, dass $ y \in A $. Angenommen, $ x \notin A_{y} $. Dann ist $ x \in A_{x} \setminus A_{y} $ . Da $ G $ endlich ist, gilt $ A_{x} \setminus A_{y} \in G $. Es folgt, dass $ y \in A_{x} \setminus A_{y} $ Widerspruch zu $ y \in A_{y} $. Somit gilt $ y \sim x $. 
	\item Transitivität: Seien $ x, y, z \in \Omega  $ mit $ x \sim y $ und $ y \sim z $. Sei $ C \in \mathcal{M}_{x} $. Dann ist $ C \in G $ und $ x \in C $. Wegen $ y \in A_{x} $ folgt heraus $ y \in C $, also $ C \in \mathcal{M}_{y} $. Hieraus folgt mit $ z \in A_{y} $, dass $ z \in C $. Somit gilt $ x \sim z $. 
\end{itemize}

Die Menge der Äquivalenzklassen $ \left\{ A_{x} : x \in \Omega  \right\} $ ist eine Partition $ \mathcal{D} $ von $ \Omega  $. Sei 
$ G' \in G $. Dann gilt $ G' = \cup_{x \in G} \{x\} \subseteq \cup_{x \in G} A_{x} $. Für alle $ x \in G' $ ist $ G' \in \mathcal{M}_{x}
$ und daher $ A_{x} = \cap_{A \in \mathcal{M}_{x}} A \subseteq G' $. Also ist $ G = \cup_{x \in G} A_{x} \in \sigma ( \mathcal{D}) $. 
Es folgt $ G \subseteq \sigma ( \mathcal{D}) $. Für alle $ x \in \Omega  $ gilt $ A_{x} = \cap_{A \in \mathcal{M}_{x}} A \in G $ und 
$ G $ ist eine $ \sigma $-Algebra, somit $ \sigma ( \mathcal{D})  \in G $. Eindeutigkeit folgt aus Lemma 1.9. \qed

\begin{ibox}[1.11]{Definition}{CDefinition}
    Es gelte $ \left| \Omega  \right| < \infty $. Sei $ G $ eine $ \sigma $-Algebra auf $ \Omega  $. Die Partition $ \mathcal{D} $,
		für die $ G = \sigma ( \mathcal{D}) $ gilt, bezeichnen wir mit $ \mathcal{D} (G) $ und sagen dass $ \mathcal{D} (G) $ die $ \sigma $ 
		-Algebra $ G $ erzeugt.
\end{ibox}

\smalltitle[1.12]{Beispiel}
Wenn $ \left| \Omega  \right| < \infty $ gilt, dass ist $ \mathcal{D} ( \mathcal{P} ( \Omega )) = \left\{ \left\{ \omega  \right\}: \omega  \in \Omega  \right\} $.

\begin{ibox}[1.13]{Definition}{CDefinition}
	Es gelte  $ \left| \Omega  \right| < \infty $. Sei $ G $ eine $ \sigma $ -Algebra auf $ \Omega  $. Eine Abbildung $ \xi : \Omega  \to \mathbb{R} $ heißt \underline{$ \mathcal{G}$-messbar}, falls $ \xi $ konstant auf den Mengen von $ \mathcal{D} ( \mathcal{G}) $ ist.
	(genauer: falls es für jedes $ D \in \mathcal{D} ( \mathcal{G}) $ ein $ c \in \mathbb{R} $ gibt, sodass $ \forall \omega \in D :
	\xi (\omega) = c$ )
\end{ibox}

Beachte: Wenn  $ \left| \Omega  \right| < \infty $ gilt und $ \xi : \Omega  \in \mathbb{R} $ eine Abbildung ist, dann gibt es $ m \in
\mathbb{N}$ und $ x_1 , \cdots,  x_m \in \mathbb{R} $, sodass $ \xi (\Omega) := \left\{ \xi (\omega): \omega \in \Omega  \right\}
= \left\{ x_1 , \cdots, x_m \right\}$ 

\begin{ibox}[1.14]{Lemma}{CLemma}
    Es gelte $ \left| \Omega  \right| < \infty $. Sei $ \mathcal{G} $ eine $ \sigma $-Algebra auf $ \Omega  $. Seien $ m \in \mathbb{N} $
	, $ x_1 , \cdots, x_{m} \in \mathbb{R} $ und $ \xi : \Omega  \to \left\{ x_1 , \cdots,  x_{m} \right\} $. Dann sind äquivalent:
	\begin{enumerate}[label=\alph*)]
		\item $ \xi $ ist $ \mathcal{G} $-messbar
		\item Für alle $ A \subseteq \left\{ x_1 , \cdots,  x_m \right\} $ gilt $ \xi^{-1} (A) := \left\{ \omega \in \Omega : \xi (\omega) \in A \right\} \in \mathcal{G} $ 
	\end{enumerate}
\end{ibox}
Beweis: Übung. 

Im Allgemeinen (also auch wenn $ \left| \Omega  \right| = \infty $ definiert man Messbarkeit ähnlich zu (ii) in Lemma 1.14.) 

\begin{ibox}[1.15]{Definition}{CDefinition}
    Sei $ \Omega \neq \emptyset $ eine Menge. Für jedes $ \lambda \in \Omega  $ sei $ \xi_{\lambda}: \Omega  \to \mathbb{R} $ eine 
		Abbildung. Dann definieren wir 
		$$ \sigma \left( \xi_{\lambda} ; \lambda \in \Omega  \right) := \cap \left\{ \mathcal{G} \subseteq \mathcal{P} ( \Omega ) : 
		\mathcal{G} \text{ ist } \sigma \text{-messbar und } \forall \lambda \in \Omega : \xi_\lambda \text{ ist } \mathcal{G} \text{-messbar }   \right\} $$
\end{ibox}
In der Def 1.15 kann $ \sigma \left( \xi_\lambda ; \lambda \in \Omega  \right) $ interpretiert werden als die kleinste $ \sigma $-Algebra
$ \mathcal{G} $, sodass alle $ \xi_\lambda , \lambda \in \Omega \; \mathcal{G} $-messbar sind. 

\smalltitle[1.16]{Beispiel}
Wir betrachten einen Würfelwurf: Sei $ \Omega  = \left\{ 1, 2, 3,4,5,6 \right\} $ und $ \xi : \Omega  \to \left\{ 0 ,1 \right\} $ und 
$ \xi ( \omega ) $ ist $ 1 $, wenn $ \omega  $ und $ 0 $, wenn $ \omega  $ ungerade. \\
Dann ist $ \mathcal{D} = \left\{ \left\{ 1,3,5 \right\}, \left\{ 2,4,6 \right\} \right\} $ eine Partition von $ \Omega  $ und es gilt 
$ \sigma (\xi) = \sigma( \mathcal{D}) = \left\{ \emptyset, \omega , \left\{ 1,3,5 \right\}, \left\{ 2,4,6 \right\} \right\} $ 

Interpretation:  Nach Beobachtung von $ \xi (\omega ) $ können wir für jedes Ereignis $ A \in \sigma (\xi) $ sagen, ob es eingetreten ist oder nicht. Mit anderen Worten: Wir kennen zwar das eingetretene $ \omega  $ nicht, aber wir wissen, ob $ \omega  \in \emptyset $ (nie), $ \omega  \in \left\{ 1,3,5 \right\}, \omega \in \left\{ 2,4,6 \right\} $ und $ \omega  \in \Omega  $ (immer). In diesen Sinne spiegelt 
$ \sigma (\xi) $ die Information wider, die durch $ \xi $ erzeugt wird. Jede $ \sigma (\xi) $-messbare Zufallsvariable $ \hat{\xi}: \Omega  \to \mathbb{R} $ verwendet nur Information, die bereits in $ \xi $ enthalten ist. Man kann zeigen, dass $ \hat{\xi} = g \circ \xi $ für eine geeignete Funktion $ g : \left\{ 0 ,1 \right\} \to \mathbb{R} $.

\begin{ibox}[1.17]{Lemma}{CLemma}
  Es gelte $ \left| \Omega  \right| < \infty $. Sei $ \mathcal{G} $ eine $ \sigma $-Algebra auf $ \Omega $ und seien $ \xi , \eta :
	\Omega \to \mathbb{R} \; \mathcal{G}$-messbar. Dann:
	\begin{enumerate}[label=\alph*)]
		\item $ \xi + \eta, \xi - \eta \text{ und } \xi \cdot \eta $ sind $ \mathcal{G} $ -messbar.
		\item Falls $ \forall \omega \in \Omega : \eta (\omega ) \neq 0 $ so ist $ \frac{\xi}{\eta} \; \mathcal{G} $-messbar.
		\item Für eine Funktion $ f : \mathbb{R} \to \mathbb{R} $ ist $ f (\xi) = f \circ \xi : \Omega  \to \mathbb{R} \; \mathcal{G} $-messbar.
	\end{enumerate}
\end{ibox}
	\underline{Beweis}: 
	\begin{enumerate}[label=\alph*)]
		\item Sei $ D \in \mathcal{D} ( \mathcal{G} ) $. Es gibt $ a, c \in \mathbb{R} $, sodass $ \forall \omega  \in D: \xi ( \omega ) = c \text{ und } \eta (\omega ) = a$. Somit gilt $ \omega  \in D: \left(  \xi + \eta  \right) (\omega ) = \xi (\omega ) + \eta (\omega ) = c + a$. Rest analog.
		\item Analog
		\item Sei $ D \in \mathcal{D} \left( \mathcal{G} \right). $ Es gibt $ c \in \mathbb{R} $, sodass $ \forall \omega \in D : \xi  (\omega ) = c $. Also : $ \forall \omega \in D: \left( f \circ \xi \right)  (\omega ) = f \left( \xi  (\omega ) \right) = f (c) $ \qed
	\end{enumerate}

	\smalltitle[]{Wiederholung:}
	Wenn $ \mathcal{G}$ eine $ \sigma $-Algebra auf $ \Omega  $ ist, dann heißt ein Abbildung $ \mathbb{P} : \mathcal{G} \to \left[ 0,1 \right] $  Wahrscheinlichkeitsmaß (auf dem messbaren Raum $ \left( \Omega , \mathcal{G} \right) $ ), falls $ \mathbb{P} [\omega ] = 1 $ 
	und aus $ A_j \in \mathcal{G}, \forall j \in \mathbb{N} $ mit $ \forall i \neq j : A_i \cap A_j = \emptyset $ folgt, dass 
 $$ \mathbb{P} \left[ \bigcup_{j = 1}^{\infty} A_j \right] = \sum_{j = 1}^{\infty} \mathbb{P} \left[ A_j \right] $$
 Weiter heißt dann $ \left( \Omega , \mathcal{G}, \mathbb{P}  \right) $ Wahrscheinlichkeitsraum.

 \begin{ibox}[1.18]{Definition}{CDefinition}
    Sei $ \left( \Omega , \mathcal{G}, \mathbb{P}  \right) $ Wahrscheinlichkeitsraum mit $ \left| \Omega  \right| < \infty $. Zwei 
		$ \sigma $ -Algebren $ \mathcal{G}_{1}, \mathcal{G}_{2} $ heißen \underline{unabhängig}, falls $ \forall A_1 \in \mathcal{G}_{1} $ 
		und $ A_2 \in \mathcal{G}_{i}, \; \mathbb{P} \left[ A_1 \cap A_2 \right] = \mathbb{P} \left[ A_1 \right] \cdot \mathbb{P} \left[ A_2 \right] $. Eine Zufallsvariable $ \xi : \Omega \to \mathbb{R} $ heißt \underline{unabhängig} von einer $ \sigma $-Algebra $ \mathcal{G} $
		wenn die $\sigma$-Algebra $ \sigma (\xi) \text{ und } \mathcal{G} $ unabhängig sind. Zwei Zufallsvariable $ \xi, \eta : \sigma \to 
		\mathbb{R} $ heißen \underline{unabhängig}, wenn die $\sigma$-Algebra $ \sigma (\xi) \text{ und } \sigma (\eta) $ unabhängig sind.
 \end{ibox}
 \documentclass[11.5 pt, a4paper]{memoir}

\usepackage[ngerman]{babel}
\usepackage{bookmark}
\usepackage{amsmath}
\usepackage{amssymb}
\usepackage{amsthm}
\usepackage[T1]{fontenc}
\usepackage{makeidx}
\usepackage{enumitem}
\usepackage{mathtools}
\usepackage{upgreek}

\usepackage{microtype}
\usepackage{svg}
\usepackage{parskip}
\usepackage{hyperref} % Make TOC clickable

\usepackage{lipsum}
\usepackage{fancyhdr} % Heading customization
\usepackage{geometry} % Adjust page padding 
\usepackage{adjustbox}

\usepackage{xcolor}
\usepackage[most]{tcolorbox}

\definecolor{Black}{HTML}{292939}
\definecolor{CTheorem}{HTML}{FFFFFF}
\definecolor{CLemma}{HTML}{FDFFED}
\definecolor{CDefinition}{HTML}{EFF9F0}
\definecolor{DarkGray}{HTML}{6B6969}
\definecolor{ImportantBorder}{HTML}{F14747}


% Set up page layout
\geometry{
    a4paper,
    left=2cm,
    right=2cm,
    top=2cm,
    bottom=2cm
}
\linespread{1.1}
\makeindex

\pagestyle{fancy}
\fancyhf{} % Clear headre/footer
\fancyheadoffset[LE,RO]{0pt} % Adjust headsep for page number and title
\fancyhead[RE,LO]{}
\fancyhead[RE,RO]{\rightmark} % Add pape header title
\fancyhead[LE,LO]{\thepage}  % Add page number to the left
\renewcommand{\headrulewidth}{0pt} % Delete the line in the header 

% Define the \para command
\newcommand{\para}[2]{
    \clearpage % Start on a new page
    \thispagestyle{empty}% removes the top left page number and top right chapter name
    \begin{center} % Center the chapter title
        \vspace*{11em}
        \Huge   \bfseries \S #1 \quad  #2 % Chapter title with symbol and counter
    \end{center}
    \vspace{8em}
    \addcontentsline{toc}{chapter}{#1.\hspace{0.6em} #2} % Add chapter to table of contents
    \fancyhead[RE,RO]{#2} % Add pape header title
}

\newcommand{\nsec}[2]{
    \section*{\large  #1 \hspace{0.3em} #2}
    %\phantomsection
    \addcontentsline{toc}{section}{#1 \hspace{0.3em} #2}
}
\newcommand{\smalltitle}[2][]{
    %\phantomsection
    \subsubsection*{ #1 \hspace{0.2em} #2}
}
\newtcolorbox{ibox}[3][]{
	enhanced jigsaw,
	colback=#3,
	colframe=DarkGray,
	coltitle=Black,
	drop shadow,
	title={\textbf{#1\hspace{0.4em}#2}},
	%before skip = 1em,
	attach title to upper,
	after title={: \quad  },
}
\renewcommand{\d}[1]{\,\mathrm{d} #1}


\title{Finanzmathe Mitschrift vom 24.10.24}
\author{Sisam Khanal}
\date{\today}
\begin{document}
\raggedright
\color{Black}
\maketitle


\smalltitle[II.1]{Einführung des Einperiodenmodells} 
Sei $ d,p \in \mathbb{N} $. Wir betrachten einen Finanzmarkt mit $ d+1 $ Wertpapieren. Beim Einperiodenmodell gibt es genau zwei Zweitpunkt, den Anfangszeitpunkt $ 0 $ und der Endzeitpunkt $ 1 $. Es kann nur zum Zeitpunkt $ 0 $ gehandelt werden
wobei man die Preise zum Zeitpunkt $ 0 $  kennt, aber i.A noch nicht klar ist, welches von $ l $ Szenarien für die Preise zum Zeitpunkt $ 1 $ eintreten wird. \break

Preisvektor zur Zeit $ 0 $ :
$$ \overline{ S_0} = \left( S_0^0, S_0^1 , \cdots,  S_0^d \right)^T = \left( S_0^0, S_0^T \right)^T \in \mathbb{R}^{d+1}  $$
wobei $ S_0^i $ der Preis des $ i $-ten Wertpapiers zur Zeit $ 0 $ für $ i \in \left\{ 0, 1 , \cdots,  d \right\}  $ ist.\break

Sei $ \left( \Omega , \mathcal{F}, \mathbb{P}  \right)$ ein endlicher Wahrscheinlichkeitsraum mit 
\begin{align*}
	&\left| \Omega  \right| = l,  \Omega  = \left\{ \omega_1 , \cdots, \omega_l \right\}, \\
	&	\mathcal{F} = 2^{\Omega } = \mathcal{P} \left( \Omega  \right),\\
	&	\mathbb{P} : \mathcal{F} \to \left[ 0, 1 \right],\\
	&	\mathbb{P} [A] = \sum_{\omega_k \in A} \mathbb{P} \left[ \left\{ \omega_k \right\}  \right]  \text{ für  } A \subseteq \Omega,\\
	&	\mathbb{P} \left[ \left\{ \omega_k \right\}  \right]  > 0 \;  \forall k \in \left\{ 1 , \cdots, l \right\} 
\end{align*}

Zufallsvektor der Preise zur Zeit $ l $ :
$$ \overline{ S_1} = \left( S_1^0, S_1^1 , \cdots,  S_1^d\right) ^{T} = \left( S_1^0, S_1^T \right)^{T} : \Omega \to \mathbb{R}^{d+1}  $$
wobei $ S_1^i \left( \omega_k \right) \in \mathbb{R} $ der Preis der $ i $-ten Wertpapiers zur Zeit $ 1 $ unter Szenario $ k $ ist für $ i \in \left( 0,1 , \cdots,  d \right) , k \in \left( 1 , \cdots,  l \right)  $ 

Alternative können wir die Preise zur Zeit $ 1 $ als eine Preismatrix auffassen:

$$ S_1 = \left[ \;  \overline{ S_1} (\omega_1) \;  \overline{ S_1} (\omega_2) \; \cdots \;  \overline{ S_1} (\omega_l) \;  \right] =  \begin{pmatrix}
	S_1^0 (\omega_1) & S_1^0 (\omega_2) & \cdots & S_1^0 (\omega_l) \\
	S_1^1 (\omega_1) & S_1^1 (\omega_2) & \cdots & S_1^1 (\omega_l) \\
	\vdots & \vdots & & \vdots \\
	S_1^d (\omega_1) & S_1^d (\omega_2) & \cdots & S_1^d (\omega_l) 
\end{pmatrix} \in \mathbb{R}^{(d+1) \times l}
 $$
Wir nehmen an, dass das Orte Wertpapiers eine Anleihe mit fester Verzinsung $ r \geq 0 $ (Bankkonto) ist mit 

$$ S_0^0 = 1 \text{ und }  \forall \omega  \in \Omega : S_1^{0}(\omega ) = 1 + r $$

Diskontierte Preise: 
$$ X_0 = \left( X_0^1 , \cdots,  X_0^{d} \right)^{T}, X_1 = \left( X_1^1 , \cdots,  X_1^{d} \right)^{T}, \Delta X_1 = \left(\Delta X_1^1 , \cdots, \Delta X_1^{d} \right)^{T},  $$
$$ \text{ mit } X_0^i = S_0^i, X_1^i = \frac{S_1^i}{1+r} , \Delta X_1^i = X_1^{i} - X_0^i, i \in \left\{ 1 , \cdots,  d \right\}  $$
Zum Zeitpunkt $ 0 $ nählt man ein Portfolio.

\begin{ibox}[1.1]{Definition}{CDefinition}
    Eine Handelsstrategie oder ein Portfolio (im Einperiodenmodell) ist ein Vektor 
		$$ \overline{ H} = \left( H^0, H^1 , \cdots,  H^d \right)^{T} = \left( H^0, H^T \right)^T \in \mathbb{R}^{d+1}  $$
\end{ibox}
Bei einer Handelsstrategie $ \overline{ H} = \left(  H^0, H^1 , \cdots,  H^d \right)^{T}  $ beschreibt $ H^{i} \in \mathbb{R} $ die Stückzahl von Wertpapier $ i  $ in Portfolio Zwischen den beiden Zeitpunkten (für $ i \in \left\{ 0, 1 , \cdots,  d \right\}  $. Dabei ist $ H^{i} $ nicht zufällig. \\

Falls $ H^0 < 0 $ : Kreditaufnahme, \\
Falls $ H^{i} < 0 $ für ein $ i \in \left\{ 1 , \cdots,  d \right\}  $ : Leerverkauf (short sell)

\begin{ibox}[1.2]{Definition}{CDefinition}
    Sei $ \overline{ H} \in \mathbb{R}^{d+1}  $. Der Wert der Handelsstrategie $ \overline{ H}  $ ist zur Zeit $ 0 $ durch:
		$$ V_0^{ \overline{ H} } = \left< \overline{ S_0}, \overline{ H}   \right> = \overline{ S_0^T} \overline{ H} = \sum_{i=0}^{d} S_0^i H^i   $$
	Und zur Zeit $ 1 $ durch die Zufallsvariable $ V_1^{ \overline{ H} } : \Omega \to \mathbb{R} $,
	$$ V_1^{ \overline{ H} } = \left< \overline{ S_1} (\omega ), \overline{ H}   \right> =  \sum_{i=0}^{d} S_0^i (\omega ) H^i   $$
	definiert. Weiter definieren wir die diskontierten Werte als 
	$$ D_0^{ \overline{ H} } = V_0^{ \overline{ H} } \text{ und }  D_1^{ \overline{ H} } = \frac{V_1^{ \overline{ H} }}{1 +r} $$
\end{ibox}

\begin{ibox}[1.3]{Lemma}{CLemma}
    Sei $ \overline{ H} = \left( H^0, H^T \right)^T \in \mathbb{R}^{d+1}  $ eine Handelsstrategie. Dann gilt 
		$$ D_1^{ \overline{ H} } = D_0 ^{ \overline{ H} } + \left< \Delta X_1, H \right> $$
\end{ibox}
Beweis: Übung


\smalltitle[]{Beispiel 1.4}
Seien $ d =1 , l=1, p \in (0,1), \mathbb{P} \left[ \left\{ \omega_1 \right\}  \right] = p, \mathbb{P} \left[ \left\{ \omega_2 \right\}  \right] = 1-p, S_0^{1} = 1, S_1^1 (\omega_1) = 2, S_1^1 (\omega_2) = \frac{1}{2}  $
\break
Sei $ r=2 $ und wähle die Handelsstrategie $ \overline{ H} (1,-1)  $ Dann ist,

\begin{align*}
	V_0^{ \overline{ H} } &= 1 \cdot 1 + 1 \cdot (-1) = 0, \\
	V_1^{ \overline{ H} } (\omega_1) &= (1+r) \cdot 1 + S_1^1 (\omega_1) \cdot (-1) = 1, \\
	V_1^{ \overline{ H} } (\omega_2) &= (1+r) \cdot 1 + S_1^1 (\omega_2) \cdot (-1) = \frac{5}{2}
\end{align*}



\smalltitle[II.2]{No-Arbitrage und FTAP1}
\begin{itemize}
	\item Ziel: Charakterirrung vom Markt, in dem es keine Arbitrage-Möglichkeiten gibt.
	\item Hilfsmittel: äquivalente Martingallmaße
	\item Hauptresultat: First Fundamental theorem of asset pricing (FTAP1)
	\item Praktische Umsetzung: prüfe Gleichungssystem auf Lösbarkeit (unter Nebenbedingungen)
\end{itemize}

\begin{ibox}[2.1]{Definition}{CDefinition}
	Eine Handelsstrategie $ \overline{H} \in \mathbb{R}^{d+1} $ heißt \underline{Arbitragemöglichkeit} falls gelte:
	\begin{enumerate}[label=\alph*)]
		\item $ V^{ \overline{ H} }_0 \leq 0$ ,
		\item $ \forall \omega \in \Omega : V^{ \overline{H } }_{1}(\omega) \geq 0 $ und
		\item $ \exists \omega \in \Omega :  V^{ \overline{H } } _{1}(\omega) > 0  $ 
	\end{enumerate}
\end{ibox}
Wir sagen, es gilt \underline{No-Arbitrage (NA)}, falls keine Arbitragemöglichkeit existieren. 
\begin{ibox}[2.2]{Lemma}{CLemma}
    Folgende Aussagen sind äquivalent
		\begin{enumerate}[label=\alph*)]
			\item Es gibt eine Arbitragemöglichkeit.
			\item Es gibt ein $ \alpha \in \mathbb{R}^{d} $, sodass 
				\begin{itemize}
					\item $ \forall \omega \in \Omega : \left<\Delta X_1(\omega), \alpha \right> \geq 0 $  und 
					\item $ \exists \omega \in \Omega :  \left<\Delta X_1(\omega), \alpha \right> > 0  $ 
				\end{itemize}
			\item Es gibt eine Arbitragemöglichkeit $ \overline{ H} \in \mathbb{R}^{d+1}  $ mit $ V^{ \overline{ H} }_0 = 0 $ 	
		\end{enumerate}
\end{ibox}
\subsection*{Beweis} Übung
Eine Wahrscheinlichkeitsmaß $ Q $ auf $ \left( \Omega, \mathcal{F} \right)  $ können wir durch den Vektoren $ q \in \mathbb{R}^{l} $ 
mit $ q_k = Q \left[ \left\{ \omega_k \right\}  \right] , k \in \left\{ 1 , \cdots,  l \right\}  $ charakterisieren. Für eine Zufallsvariable $ Y : \Omega \to \mathbb{R} $ notieren wir mit 
$$ \mathbb{E}^{Q} \left[ Y \right] = \sum_{k=1}^{l} q_k Y \left( \omega_{k} \right)  $$
den Erwartungswert von $ Y $ bzgl. $ Q $. Für $ m \in \mathbb{N} $ und einen $ m $-dim Zufallsvektor $ Y = \left( Y^{1}, , \cdots,  
Y^{m}\right)^{T}: \Omega to \mathbb{R}^{m} $ setzen wir 
$$ \mathbb{E}^{Q} \left[ Y \right] = \left( \mathbb{E}^{Q} \left[ Y^{1} \right], \cdots,  \mathbb{E}^{Q} \left[ Y^{m} \right]  \right) ^{T}  $$
\begin{ibox}[2.3]{Definition}{CDefinition}
	Ein Wahrscheinlichkeitsmaß $ Q $  auf $ \left( \Omega, \mathcal{F} \right) $ heißt \underline{risikoneutral} oder 
	\underline{Martingalmaße}, falls $ \forall i \in \left\{ 1 , \cdots,  d \right\}  $ :
	$$ \mathbb{E}^{Q} \left[ X_1^{i} \right] = X_{0}^{i}$$
\end{ibox}

\begin{ibox}[2.4]{Lemma}{CLemma}
    Sei $ Q $ ein Wahrscheinlichkeitsmaß auf $ \left\{ \Omega, \mathcal{F} \right\}  $ und $ q_k = Q \left[ \left\{ \omega_k \right\}  \right] , k \in \left\{ 1 , \cdots,  l \right\}  $. Es bezeichne $ e_i $ den $ i$-ten Einheitsvektor im $ \mathbb{R}^{d} $. Dann sind äquivalent:
		\begin{enumerate}[label=\alph*)]
			\item $ Q $ ist eine Martingalmaße
			\item $ \forall i \in \left\{ 1 , \cdots,  d \right\} : \sum_{k=1}^{l} q_k \left< \Delta X_1 (\omega_k), e_i \right>  = 0$ 
			\item $ \forall \alpha \in \mathbb{R}^{d} : \sum_{k=1}^{l} q_k \left< \Delta X_1 (\omega_k), \alpha \right>  = 0$ 
		\end{enumerate}
\end{ibox}
\subsection*{Beweis} Übung
\begin{ibox}[2.5]{Definition}{CDefinition}
	Ein Wahrscheinlichkeitsmaß $ Q $ auf $ \left( \Omega, \mathcal{F} \right)  $ heißt (zu $ \mathbb{P} $) \underline{äquivalentes Wahrscheinlichkeitsmaß}, wenn $ \forall k \in \left\{ 1 , \cdots,  l \right\} : q_k = Q \left[ \left\{ \omega_k \right\}  \right] > 0  $  
\end{ibox}
Wir hatten angenommen, dass $ \forall k \in \left\{ 1 , \cdots,  l \right\} : \mathbb{P} \left[ \left\{ \omega_k \right\}  \right] > 0 $  ein $ \mathbb{P} $ äquivalentes Wahrscheinlichkeitsmaß auf $ \left\{ \Omega, \mathcal{F} \right\}  $ ist, dann stimmen also die Mengen der möglichen Szenarien unter $ \mathbb{P} \text{ und }  Q $ überein (aber die genauen Wahrscheinlichkeiten sind in der Regel unterschiedlich)

\begin{ibox}[2.6]{Definition}{CDefinition}
	Ein Wahrscheinlichkeitsmaß $ Q $ auf $ \left\{ \Omega, \mathcal{F} \right\}  $ heißt \underline{äquivalentes Martingalmaße (ÄMM)}
	falls $ Q $ risikoneutral und äquivalent ist. Wir definieren 
	$$ \mathcal{P} = \left\{ Q : Q \text{ ist ÄMM }  \right\}  $$
	
\end{ibox}

\subsection*{Beispiel 2.7:} Seien $ d = 1, l = 2, p \in (0,1), \mathbb{P} \left[ \left\{ \omega_1 \right\}  \right] = p, \mathbb{P} \left[ \left\{ \omega_2 \right\}  \right] = 1 - p, S_0^{1} = 1, S_1^{1}(\omega_1) = 2, S_1^{1}(\omega_2)= \frac{1}{2} \text{ und }  r = 0  $. Wir versuchen, ein ÄMM zu finden. Wenn $ Q $ ein ÄMM (charakterisiert durch $ q \in \mathbb{R}_2 $ ist, dann müssen gelten:  
$ q_1 > 0, q_2 >0 $ 
 $$ \sum_{k=1}^{2} q_k S_1^{1} (\omega_k) = q_1 \underbrace{ S_1^{1} (\omega_1)}_{=2} + q_2 \underbrace{ S_1^{1} (\omega_2)}_{=\frac{1}{2} } = 1 $$
LGS lösen: 
\begin{align*}
	q_1 + q_2 &= 1 \\
	 2q_1 + \frac{1}{2} q_2 &= 1 
\end{align*}

Andere Schreibweise: 
\begin{align*}
\begin{pmatrix}
	1 & 1 \\
	2 & \frac{1}{2} 
\end{pmatrix} 
\begin{pmatrix}
	q_1 \\ q_2 
\end{pmatrix}
 = \begin{pmatrix}
	 1 \\ 1  
 \end{pmatrix} \implies \begin{pmatrix}
 	q_1 \\ q_2 
 \end{pmatrix}
 = \begin{pmatrix}
 	 \frac{2}{3} \\ \frac{1}{2} 
 \end{pmatrix}
\end{align*}
Im ersten fundamentalsatz wird ein Zusammenhang zwischen der Existenz von ÄMMs und NA hergestellt:

\begin{ibox}[2.8]{Satz (FTAP1)}{CTheorem}
	Es gilt: $ NA \iff \mathcal{P} \neq \emptyset $.
	Andereseits gilt: Ein Vektor $ q \in \mathbb{R}^{l} $ definiert genau dann ein ÄMM (via $ Q \left[ \left\{ \omega_k \right\}  \right] : = q_k, k \in \left\{ 1, , \cdots,  l \right\}  $ , wenn $ k \in \left\{  1, \cdots,  l \right\} : q_k > 0 $ gilt, und $ q $ 
	eine Lösung des Systems: $ \forall i \in\left\{ 1 , \cdots,  d \right\}  $ 

	\begin{align*}
		q_1 + q_2 + \cdots + q_{l} &= 1 \\
		\frac{1}{1+r} \left( S_1^{i} (\omega_1) q_1 +   S_1^{i} (\omega_2) q_2  + \cdots +   S_1^{i} (\omega_l) q_l \right) &= S_0^{i}
	\end{align*}
ist. Um zu überprüfung, ab NA gilt, kann man wegen FTAP1 also testen, ob 
$$ S_1 q = (1+r) \overline{ S_0}  $$
eine Lösung $ q \in \mathbb{R}^{l} $ mit $ \forall k \in \left\{ 1 , \cdots,  l \right\} : q_k > 0 $ besitzt
\end{ibox}
Für den Beweis des FTAP1 benötigen wir folgenden Trennungssatz (in $ \mathbb{R}^{l} $ )

\begin{ibox}[2.9]{Satz (Trennungssatz)}{CTheorem}
	Sei $ n \in \mathbb{N} $ 
	\begin{enumerate}[label=\alph*)]
		\item Sei $ C \subseteq \mathbb{R}^n $ abgeschlossen, konvex und nichtleer mit $ 0 \notin C $. Dann existieren $ y \in C $ und 
			$ \delta > 0 $ sodass $ \forall x \in C : \left<x,y \right> \geq \delta > 0 $ 
		\item Sei $ K \subseteq \mathbb{R}^n $ kompakt, konvex und nicht leer und sei $ U \subseteq \mathbb{R}^n $ eine linearen Unterraum
			mit $ K \cap U = \emptyset $. Dann existiert $ y \in \mathbb{R}^n $ sodass
			\begin{itemize}
				\item $ \forall x \in K : \left<x, y \right> >0$,
				\item $ \forall x \in U : \left<x,y \right> = 0 $ 
			\end{itemize}
	\end{enumerate}
\end{ibox}

\smalltitle[2.10]{Proposition}
Die Menge $ \mathcal{P} $ aller ÄMMs ist konvex. Insbesondere gilt:
$$ \left| \mathcal{P} \right| \implies | \mathcal{P}| = \infty $$
Beweis: Übung

\smalltitle[II.3]{Arbitragefreie Preise}
Wir Identifizieren im Folgenden ein Derivat mit seiner Auszahlung zur Zeit $ T=1 $ 

\begin{ibox}[3.1]{Definition}{CDefinition}
    Ein \textit{Derivat} ist eine Zufallsvariable $ \xi : \Omega \to \mathbb{R} $. Für $ x, y \in \mathbb{R} $ notieren wir 
		$$ \left( x - y \right)^{+} = \text{ max } \left\{ x - y, 0 \right\}  $$
\end{ibox}

\smalltitle[3.2]{Beispiel}
Beispiele für Derivate:                                                                                              
\begin{enumerate}[label=\alph*)]
	\item \underline{Forward contract}: Vereinbarung, zu einem zukünftigen, festgelegten Termin $ T $ ein Gut (z.B Wertpapier) zu einem heute vereinbarten Preis $ k $ zu kaufen bzw. zu verkaufen. Im Einperiodenmodell: Forward auf Papier
		$$ i: \xi = S_1^{i} - k  $$
	\item \underline{Call-Option}: Recht, ein Gut zu festgelegt Preis $ K $ ( dem \textit{Strike} ) zu zukünftigen Zeitpunkt $T$ zu kaufen.
		Im Einperiodenmodell: Call auf Papier 
		$$ i : \xi = \left( S_1^{i} - k \right) ^{+} $$
	\item \underline{Put-Option}: Analog zu Call, aber Verkaufen. Im Einperiodenmodell: Put auf Papier
		$$ i : \xi = \left\{ k - S_1^{i} \right\} ^{+} $$
	\item \underline{Basket-Option}: Option auf einen Index von Wertpapier z.B Basket-Call-Option mit Strike $ k $ und Gewicht
		$ \alpha \in \mathbb{R}^{d} $ im Einperiodenmodell :
		$$ i : \xi = \left( K - S_1^{i} \right)  $$
	\item \underline{Basket-Option}: Option auf einen Index von Wertpapier z.B Basket-Call-Option mit Strike $ k $ und Gewicht $ \alpha
		\in \mathbb{R}^{d}$ im Einperiodenmodell :
		$$ \left(  \left< S_1, \alpha \right> - k \right) ^{+} $$
\end{enumerate}
\smalltitle[3.3]{Beispiel}
Seien $ d = 1, l =2, \mathbb{P} \left[ \left\{ \omega_1 \right\}  \right] = \frac{1}{2} = \mathbb{P} \left[ \left\{ \omega_1 \right\}  \right] , S_0^{0} = 1, S_1^{0} = 1 (r=0), S_0^{1} = 1, S_1^{1}\left( \omega_1 \right) = 2, S_1^{1} (\omega_1) = \frac{1}{2}  $.
Betrachte Call $ \xi = \left( S_1^{1} \right)^{+} $. Was ist ein angemessener Preis $ P_0 (\xi) $ für $ \xi $ zur Zeit 0? \break
\underline{Ersten Ansatz}: Preis von $ \xi $ ist mittlere Auszahlung von $ \xi $ bei wiederhaltern Spiel, d.h 
$$ p_0 (\xi) = \mathbb{E} ^{ \mathbb{P} } \left[ \xi \right] = \frac{1}{2} \cdot \left( 2 - 1 \right)^{+} + \frac{1}{2} \cdot 
\left( \frac{1}{2} -1 \right)^{+} = \frac{1}{2}  $$
Dadurch ergibt sich allerdings eine Arbitragemöglichkeit.
\break 
\underline{Zeit 0}: Verkaufe $ 6 \times $ Call zum Preis $ \frac{1}{2}  $, Kaufe $ 3 \times $ die Aktie; Startkapital 0. \\
\underline{Zeit 1} Szenario $ \omega_1 $ : Endkapital ist $ 3 \cdot 2 - 6 \cdot 1 (2-1)^{+} = 0 $ \\
Szenario $ \omega_2 $: Endkapital ist $ 3 \cdot \frac{1}{2} - 6 \cdot 1 \left( \frac{1}{2} - 1 \right) ^{+} = 1,5 $ 

\end{document}
 
 \documentclass[11.5 pt, a4paper]{memoir}

\usepackage[ngerman]{babel}
\usepackage{bookmark}
\usepackage{amsmath}
\usepackage{amssymb}
\usepackage{amsthm}
\usepackage[T1]{fontenc}
\usepackage{makeidx}
\usepackage{enumitem}
\usepackage{mathtools}
\usepackage{upgreek}

\usepackage{microtype}
\usepackage{svg}
\usepackage{parskip}
\usepackage{hyperref} % Make TOC clickable

\usepackage{lipsum}
\usepackage{fancyhdr} % Heading customization
\usepackage{geometry} % Adjust page padding 
\usepackage{adjustbox}

\usepackage{xcolor}
\usepackage[most]{tcolorbox}

\definecolor{Black}{HTML}{292939}
\definecolor{CTheorem}{HTML}{FFFFFF}
\definecolor{CLemma}{HTML}{FDFFED}
\definecolor{CDefinition}{HTML}{EFF9F0}
\definecolor{DarkGray}{HTML}{6B6969}
\definecolor{ImportantBorder}{HTML}{F14747}


% Set up page layout
\geometry{
    a4paper,
    left=2cm,
    right=2cm,
    top=2cm,
    bottom=2cm
}
\linespread{1.1}
\makeindex

\pagestyle{fancy}
\fancyhf{} % Clear headre/footer
\fancyheadoffset[LE,RO]{0pt} % Adjust headsep for page number and title
\fancyhead[RE,LO]{}
\fancyhead[RE,RO]{\rightmark} % Add pape header title
\fancyhead[LE,LO]{\thepage}  % Add page number to the left
\renewcommand{\headrulewidth}{0pt} % Delete the line in the header 

% Define the \para command
\newcommand{\para}[2]{
    \clearpage % Start on a new page
    \thispagestyle{empty}% removes the top left page number and top right chapter name
    \begin{center} % Center the chapter title
        \vspace*{11em}
        \Huge   \bfseries \S #1 \quad  #2 % Chapter title with symbol and counter
    \end{center}
    \vspace{8em}
    \addcontentsline{toc}{chapter}{#1.\hspace{0.6em} #2} % Add chapter to table of contents
    \fancyhead[RE,RO]{#2} % Add pape header title
}

\newcommand{\nsec}[2]{
    \section*{\large  #1 \hspace{0.3em} #2}
    %\phantomsection
    \addcontentsline{toc}{section}{#1 \hspace{0.3em} #2}
}
\newcommand{\smalltitle}[2][]{
    %\phantomsection
    \subsubsection*{ #1 \hspace{0.2em} #2}
}
\newtcolorbox{ibox}[3][]{
	enhanced jigsaw,
	colback=#3,
	colframe=DarkGray,
	coltitle=Black,
	drop shadow,
	title={\textbf{#1\hspace{0.4em}#2}},
	%before skip = 1em,
	attach title to upper,
	after title={: \quad  },
}
\renewcommand{\d}[1]{\,\mathrm{d} #1}


\title{Finanzmathe Mitschrift vom 24.10.24}
\author{Sisam Khanal}
\date{\today}
\begin{document}
\raggedright
\color{Black}
\maketitle


\smalltitle[II.1]{Einführung des Einperiodenmodells} 
Sei $ d,p \in \mathbb{N} $. Wir betrachten einen Finanzmarkt mit $ d+1 $ Wertpapieren. Beim Einperiodenmodell gibt es genau zwei Zweitpunkt, den Anfangszeitpunkt $ 0 $ und der Endzeitpunkt $ 1 $. Es kann nur zum Zeitpunkt $ 0 $ gehandelt werden
wobei man die Preise zum Zeitpunkt $ 0 $  kennt, aber i.A noch nicht klar ist, welches von $ l $ Szenarien für die Preise zum Zeitpunkt $ 1 $ eintreten wird. \break

Preisvektor zur Zeit $ 0 $ :
$$ \overline{ S_0} = \left( S_0^0, S_0^1 , \cdots,  S_0^d \right)^T = \left( S_0^0, S_0^T \right)^T \in \mathbb{R}^{d+1}  $$
wobei $ S_0^i $ der Preis des $ i $-ten Wertpapiers zur Zeit $ 0 $ für $ i \in \left\{ 0, 1 , \cdots,  d \right\}  $ ist.\break

Sei $ \left( \Omega , \mathcal{F}, \mathbb{P}  \right)$ ein endlicher Wahrscheinlichkeitsraum mit 
\begin{align*}
	&\left| \Omega  \right| = l,  \Omega  = \left\{ \omega_1 , \cdots, \omega_l \right\}, \\
	&	\mathcal{F} = 2^{\Omega } = \mathcal{P} \left( \Omega  \right),\\
	&	\mathbb{P} : \mathcal{F} \to \left[ 0, 1 \right],\\
	&	\mathbb{P} [A] = \sum_{\omega_k \in A} \mathbb{P} \left[ \left\{ \omega_k \right\}  \right]  \text{ für  } A \subseteq \Omega,\\
	&	\mathbb{P} \left[ \left\{ \omega_k \right\}  \right]  > 0 \;  \forall k \in \left\{ 1 , \cdots, l \right\} 
\end{align*}

Zufallsvektor der Preise zur Zeit $ l $ :
$$ \overline{ S_1} = \left( S_1^0, S_1^1 , \cdots,  S_1^d\right) ^{T} = \left( S_1^0, S_1^T \right)^{T} : \Omega \to \mathbb{R}^{d+1}  $$
wobei $ S_1^i \left( \omega_k \right) \in \mathbb{R} $ der Preis der $ i $-ten Wertpapiers zur Zeit $ 1 $ unter Szenario $ k $ ist für $ i \in \left( 0,1 , \cdots,  d \right) , k \in \left( 1 , \cdots,  l \right)  $ 

Alternative können wir die Preise zur Zeit $ 1 $ als eine Preismatrix auffassen:

$$ S_1 = \left[ \;  \overline{ S_1} (\omega_1) \;  \overline{ S_1} (\omega_2) \; \cdots \;  \overline{ S_1} (\omega_l) \;  \right] =  \begin{pmatrix}
	S_1^0 (\omega_1) & S_1^0 (\omega_2) & \cdots & S_1^0 (\omega_l) \\
	S_1^1 (\omega_1) & S_1^1 (\omega_2) & \cdots & S_1^1 (\omega_l) \\
	\vdots & \vdots & & \vdots \\
	S_1^d (\omega_1) & S_1^d (\omega_2) & \cdots & S_1^d (\omega_l) 
\end{pmatrix} \in \mathbb{R}^{(d+1) \times l}
 $$
Wir nehmen an, dass das Orte Wertpapiers eine Anleihe mit fester Verzinsung $ r \geq 0 $ (Bankkonto) ist mit 

$$ S_0^0 = 1 \text{ und }  \forall \omega  \in \Omega : S_1^{0}(\omega ) = 1 + r $$

Diskontierte Preise: 
$$ X_0 = \left( X_0^1 , \cdots,  X_0^{d} \right)^{T}, X_1 = \left( X_1^1 , \cdots,  X_1^{d} \right)^{T}, \Delta X_1 = \left(\Delta X_1^1 , \cdots, \Delta X_1^{d} \right)^{T},  $$
$$ \text{ mit } X_0^i = S_0^i, X_1^i = \frac{S_1^i}{1+r} , \Delta X_1^i = X_1^{i} - X_0^i, i \in \left\{ 1 , \cdots,  d \right\}  $$
Zum Zeitpunkt $ 0 $ nählt man ein Portfolio.

\begin{ibox}[1.1]{Definition}{CDefinition}
    Eine Handelsstrategie oder ein Portfolio (im Einperiodenmodell) ist ein Vektor 
		$$ \overline{ H} = \left( H^0, H^1 , \cdots,  H^d \right)^{T} = \left( H^0, H^T \right)^T \in \mathbb{R}^{d+1}  $$
\end{ibox}
Bei einer Handelsstrategie $ \overline{ H} = \left(  H^0, H^1 , \cdots,  H^d \right)^{T}  $ beschreibt $ H^{i} \in \mathbb{R} $ die Stückzahl von Wertpapier $ i  $ in Portfolio Zwischen den beiden Zeitpunkten (für $ i \in \left\{ 0, 1 , \cdots,  d \right\}  $. Dabei ist $ H^{i} $ nicht zufällig. \\

Falls $ H^0 < 0 $ : Kreditaufnahme, \\
Falls $ H^{i} < 0 $ für ein $ i \in \left\{ 1 , \cdots,  d \right\}  $ : Leerverkauf (short sell)

\begin{ibox}[1.2]{Definition}{CDefinition}
    Sei $ \overline{ H} \in \mathbb{R}^{d+1}  $. Der Wert der Handelsstrategie $ \overline{ H}  $ ist zur Zeit $ 0 $ durch:
		$$ V_0^{ \overline{ H} } = \left< \overline{ S_0}, \overline{ H}   \right> = \overline{ S_0^T} \overline{ H} = \sum_{i=0}^{d} S_0^i H^i   $$
	Und zur Zeit $ 1 $ durch die Zufallsvariable $ V_1^{ \overline{ H} } : \Omega \to \mathbb{R} $,
	$$ V_1^{ \overline{ H} } = \left< \overline{ S_1} (\omega ), \overline{ H}   \right> =  \sum_{i=0}^{d} S_0^i (\omega ) H^i   $$
	definiert. Weiter definieren wir die diskontierten Werte als 
	$$ D_0^{ \overline{ H} } = V_0^{ \overline{ H} } \text{ und }  D_1^{ \overline{ H} } = \frac{V_1^{ \overline{ H} }}{1 +r} $$
\end{ibox}

\begin{ibox}[1.3]{Lemma}{CLemma}
    Sei $ \overline{ H} = \left( H^0, H^T \right)^T \in \mathbb{R}^{d+1}  $ eine Handelsstrategie. Dann gilt 
		$$ D_1^{ \overline{ H} } = D_0 ^{ \overline{ H} } + \left< \Delta X_1, H \right> $$
\end{ibox}
Beweis: Übung


\smalltitle[]{Beispiel 1.4}
Seien $ d =1 , l=1, p \in (0,1), \mathbb{P} \left[ \left\{ \omega_1 \right\}  \right] = p, \mathbb{P} \left[ \left\{ \omega_2 \right\}  \right] = 1-p, S_0^{1} = 1, S_1^1 (\omega_1) = 2, S_1^1 (\omega_2) = \frac{1}{2}  $
\break
Sei $ r=2 $ und wähle die Handelsstrategie $ \overline{ H} (1,-1)  $ Dann ist,

\begin{align*}
	V_0^{ \overline{ H} } &= 1 \cdot 1 + 1 \cdot (-1) = 0, \\
	V_1^{ \overline{ H} } (\omega_1) &= (1+r) \cdot 1 + S_1^1 (\omega_1) \cdot (-1) = 1, \\
	V_1^{ \overline{ H} } (\omega_2) &= (1+r) \cdot 1 + S_1^1 (\omega_2) \cdot (-1) = \frac{5}{2}
\end{align*}



\smalltitle[II.2]{No-Arbitrage und FTAP1}
\begin{itemize}
	\item Ziel: Charakterirrung vom Markt, in dem es keine Arbitrage-Möglichkeiten gibt.
	\item Hilfsmittel: äquivalente Martingallmaße
	\item Hauptresultat: First Fundamental theorem of asset pricing (FTAP1)
	\item Praktische Umsetzung: prüfe Gleichungssystem auf Lösbarkeit (unter Nebenbedingungen)
\end{itemize}

\begin{ibox}[2.1]{Definition}{CDefinition}
	Eine Handelsstrategie $ \overline{H} \in \mathbb{R}^{d+1} $ heißt \underline{Arbitragemöglichkeit} falls gelte:
	\begin{enumerate}[label=\alph*)]
		\item $ V^{ \overline{ H} }_0 \leq 0$ ,
		\item $ \forall \omega \in \Omega : V^{ \overline{H } }_{1}(\omega) \geq 0 $ und
		\item $ \exists \omega \in \Omega :  V^{ \overline{H } } _{1}(\omega) > 0  $ 
	\end{enumerate}
\end{ibox}
Wir sagen, es gilt \underline{No-Arbitrage (NA)}, falls keine Arbitragemöglichkeit existieren. 
\begin{ibox}[2.2]{Lemma}{CLemma}
    Folgende Aussagen sind äquivalent
		\begin{enumerate}[label=\alph*)]
			\item Es gibt eine Arbitragemöglichkeit.
			\item Es gibt ein $ \alpha \in \mathbb{R}^{d} $, sodass 
				\begin{itemize}
					\item $ \forall \omega \in \Omega : \left<\Delta X_1(\omega), \alpha \right> \geq 0 $  und 
					\item $ \exists \omega \in \Omega :  \left<\Delta X_1(\omega), \alpha \right> > 0  $ 
				\end{itemize}
			\item Es gibt eine Arbitragemöglichkeit $ \overline{ H} \in \mathbb{R}^{d+1}  $ mit $ V^{ \overline{ H} }_0 = 0 $ 	
		\end{enumerate}
\end{ibox}
\subsection*{Beweis} Übung
Eine Wahrscheinlichkeitsmaß $ Q $ auf $ \left( \Omega, \mathcal{F} \right)  $ können wir durch den Vektoren $ q \in \mathbb{R}^{l} $ 
mit $ q_k = Q \left[ \left\{ \omega_k \right\}  \right] , k \in \left\{ 1 , \cdots,  l \right\}  $ charakterisieren. Für eine Zufallsvariable $ Y : \Omega \to \mathbb{R} $ notieren wir mit 
$$ \mathbb{E}^{Q} \left[ Y \right] = \sum_{k=1}^{l} q_k Y \left( \omega_{k} \right)  $$
den Erwartungswert von $ Y $ bzgl. $ Q $. Für $ m \in \mathbb{N} $ und einen $ m $-dim Zufallsvektor $ Y = \left( Y^{1}, , \cdots,  
Y^{m}\right)^{T}: \Omega to \mathbb{R}^{m} $ setzen wir 
$$ \mathbb{E}^{Q} \left[ Y \right] = \left( \mathbb{E}^{Q} \left[ Y^{1} \right], \cdots,  \mathbb{E}^{Q} \left[ Y^{m} \right]  \right) ^{T}  $$
\begin{ibox}[2.3]{Definition}{CDefinition}
	Ein Wahrscheinlichkeitsmaß $ Q $  auf $ \left( \Omega, \mathcal{F} \right) $ heißt \underline{risikoneutral} oder 
	\underline{Martingalmaße}, falls $ \forall i \in \left\{ 1 , \cdots,  d \right\}  $ :
	$$ \mathbb{E}^{Q} \left[ X_1^{i} \right] = X_{0}^{i}$$
\end{ibox}

\begin{ibox}[2.4]{Lemma}{CLemma}
    Sei $ Q $ ein Wahrscheinlichkeitsmaß auf $ \left\{ \Omega, \mathcal{F} \right\}  $ und $ q_k = Q \left[ \left\{ \omega_k \right\}  \right] , k \in \left\{ 1 , \cdots,  l \right\}  $. Es bezeichne $ e_i $ den $ i$-ten Einheitsvektor im $ \mathbb{R}^{d} $. Dann sind äquivalent:
		\begin{enumerate}[label=\alph*)]
			\item $ Q $ ist eine Martingalmaße
			\item $ \forall i \in \left\{ 1 , \cdots,  d \right\} : \sum_{k=1}^{l} q_k \left< \Delta X_1 (\omega_k), e_i \right>  = 0$ 
			\item $ \forall \alpha \in \mathbb{R}^{d} : \sum_{k=1}^{l} q_k \left< \Delta X_1 (\omega_k), \alpha \right>  = 0$ 
		\end{enumerate}
\end{ibox}
\subsection*{Beweis} Übung
\begin{ibox}[2.5]{Definition}{CDefinition}
	Ein Wahrscheinlichkeitsmaß $ Q $ auf $ \left( \Omega, \mathcal{F} \right)  $ heißt (zu $ \mathbb{P} $) \underline{äquivalentes Wahrscheinlichkeitsmaß}, wenn $ \forall k \in \left\{ 1 , \cdots,  l \right\} : q_k = Q \left[ \left\{ \omega_k \right\}  \right] > 0  $  
\end{ibox}
Wir hatten angenommen, dass $ \forall k \in \left\{ 1 , \cdots,  l \right\} : \mathbb{P} \left[ \left\{ \omega_k \right\}  \right] > 0 $  ein $ \mathbb{P} $ äquivalentes Wahrscheinlichkeitsmaß auf $ \left\{ \Omega, \mathcal{F} \right\}  $ ist, dann stimmen also die Mengen der möglichen Szenarien unter $ \mathbb{P} \text{ und }  Q $ überein (aber die genauen Wahrscheinlichkeiten sind in der Regel unterschiedlich)

\begin{ibox}[2.6]{Definition}{CDefinition}
	Ein Wahrscheinlichkeitsmaß $ Q $ auf $ \left\{ \Omega, \mathcal{F} \right\}  $ heißt \underline{äquivalentes Martingalmaße (ÄMM)}
	falls $ Q $ risikoneutral und äquivalent ist. Wir definieren 
	$$ \mathcal{P} = \left\{ Q : Q \text{ ist ÄMM }  \right\}  $$
	
\end{ibox}

\subsection*{Beispiel 2.7:} Seien $ d = 1, l = 2, p \in (0,1), \mathbb{P} \left[ \left\{ \omega_1 \right\}  \right] = p, \mathbb{P} \left[ \left\{ \omega_2 \right\}  \right] = 1 - p, S_0^{1} = 1, S_1^{1}(\omega_1) = 2, S_1^{1}(\omega_2)= \frac{1}{2} \text{ und }  r = 0  $. Wir versuchen, ein ÄMM zu finden. Wenn $ Q $ ein ÄMM (charakterisiert durch $ q \in \mathbb{R}_2 $ ist, dann müssen gelten:  
$ q_1 > 0, q_2 >0 $ 
 $$ \sum_{k=1}^{2} q_k S_1^{1} (\omega_k) = q_1 \underbrace{ S_1^{1} (\omega_1)}_{=2} + q_2 \underbrace{ S_1^{1} (\omega_2)}_{=\frac{1}{2} } = 1 $$
LGS lösen: 
\begin{align*}
	q_1 + q_2 &= 1 \\
	 2q_1 + \frac{1}{2} q_2 &= 1 
\end{align*}

Andere Schreibweise: 
\begin{align*}
\begin{pmatrix}
	1 & 1 \\
	2 & \frac{1}{2} 
\end{pmatrix} 
\begin{pmatrix}
	q_1 \\ q_2 
\end{pmatrix}
 = \begin{pmatrix}
	 1 \\ 1  
 \end{pmatrix} \implies \begin{pmatrix}
 	q_1 \\ q_2 
 \end{pmatrix}
 = \begin{pmatrix}
 	 \frac{2}{3} \\ \frac{1}{2} 
 \end{pmatrix}
\end{align*}
Im ersten fundamentalsatz wird ein Zusammenhang zwischen der Existenz von ÄMMs und NA hergestellt:

\begin{ibox}[2.8]{Satz (FTAP1)}{CTheorem}
	Es gilt: $ NA \iff \mathcal{P} \neq \emptyset $.
	Andereseits gilt: Ein Vektor $ q \in \mathbb{R}^{l} $ definiert genau dann ein ÄMM (via $ Q \left[ \left\{ \omega_k \right\}  \right] : = q_k, k \in \left\{ 1, , \cdots,  l \right\}  $ , wenn $ k \in \left\{  1, \cdots,  l \right\} : q_k > 0 $ gilt, und $ q $ 
	eine Lösung des Systems: $ \forall i \in\left\{ 1 , \cdots,  d \right\}  $ 

	\begin{align*}
		q_1 + q_2 + \cdots + q_{l} &= 1 \\
		\frac{1}{1+r} \left( S_1^{i} (\omega_1) q_1 +   S_1^{i} (\omega_2) q_2  + \cdots +   S_1^{i} (\omega_l) q_l \right) &= S_0^{i}
	\end{align*}
ist. Um zu überprüfung, ab NA gilt, kann man wegen FTAP1 also testen, ob 
$$ S_1 q = (1+r) \overline{ S_0}  $$
eine Lösung $ q \in \mathbb{R}^{l} $ mit $ \forall k \in \left\{ 1 , \cdots,  l \right\} : q_k > 0 $ besitzt
\end{ibox}
Für den Beweis des FTAP1 benötigen wir folgenden Trennungssatz (in $ \mathbb{R}^{l} $ )

\begin{ibox}[2.9]{Satz (Trennungssatz)}{CTheorem}
	Sei $ n \in \mathbb{N} $ 
	\begin{enumerate}[label=\alph*)]
		\item Sei $ C \subseteq \mathbb{R}^n $ abgeschlossen, konvex und nichtleer mit $ 0 \notin C $. Dann existieren $ y \in C $ und 
			$ \delta > 0 $ sodass $ \forall x \in C : \left<x,y \right> \geq \delta > 0 $ 
		\item Sei $ K \subseteq \mathbb{R}^n $ kompakt, konvex und nicht leer und sei $ U \subseteq \mathbb{R}^n $ eine linearen Unterraum
			mit $ K \cap U = \emptyset $. Dann existiert $ y \in \mathbb{R}^n $ sodass
			\begin{itemize}
				\item $ \forall x \in K : \left<x, y \right> >0$,
				\item $ \forall x \in U : \left<x,y \right> = 0 $ 
			\end{itemize}
	\end{enumerate}
\end{ibox}

\smalltitle[2.10]{Proposition}
Die Menge $ \mathcal{P} $ aller ÄMMs ist konvex. Insbesondere gilt:
$$ \left| \mathcal{P} \right| \implies | \mathcal{P}| = \infty $$
Beweis: Übung

\smalltitle[II.3]{Arbitragefreie Preise}
Wir Identifizieren im Folgenden ein Derivat mit seiner Auszahlung zur Zeit $ T=1 $ 

\begin{ibox}[3.1]{Definition}{CDefinition}
    Ein \textit{Derivat} ist eine Zufallsvariable $ \xi : \Omega \to \mathbb{R} $. Für $ x, y \in \mathbb{R} $ notieren wir 
		$$ \left( x - y \right)^{+} = \text{ max } \left\{ x - y, 0 \right\}  $$
\end{ibox}

\smalltitle[3.2]{Beispiel}
Beispiele für Derivate:                                                                                              
\begin{enumerate}[label=\alph*)]
	\item \underline{Forward contract}: Vereinbarung, zu einem zukünftigen, festgelegten Termin $ T $ ein Gut (z.B Wertpapier) zu einem heute vereinbarten Preis $ k $ zu kaufen bzw. zu verkaufen. Im Einperiodenmodell: Forward auf Papier
		$$ i: \xi = S_1^{i} - k  $$
	\item \underline{Call-Option}: Recht, ein Gut zu festgelegt Preis $ K $ ( dem \textit{Strike} ) zu zukünftigen Zeitpunkt $T$ zu kaufen.
		Im Einperiodenmodell: Call auf Papier 
		$$ i : \xi = \left( S_1^{i} - k \right) ^{+} $$
	\item \underline{Put-Option}: Analog zu Call, aber Verkaufen. Im Einperiodenmodell: Put auf Papier
		$$ i : \xi = \left\{ k - S_1^{i} \right\} ^{+} $$
	\item \underline{Basket-Option}: Option auf einen Index von Wertpapier z.B Basket-Call-Option mit Strike $ k $ und Gewicht
		$ \alpha \in \mathbb{R}^{d} $ im Einperiodenmodell :
		$$ i : \xi = \left( K - S_1^{i} \right)  $$
	\item \underline{Basket-Option}: Option auf einen Index von Wertpapier z.B Basket-Call-Option mit Strike $ k $ und Gewicht $ \alpha
		\in \mathbb{R}^{d}$ im Einperiodenmodell :
		$$ \left(  \left< S_1, \alpha \right> - k \right) ^{+} $$
\end{enumerate}
\smalltitle[3.3]{Beispiel}
Seien $ d = 1, l =2, \mathbb{P} \left[ \left\{ \omega_1 \right\}  \right] = \frac{1}{2} = \mathbb{P} \left[ \left\{ \omega_1 \right\}  \right] , S_0^{0} = 1, S_1^{0} = 1 (r=0), S_0^{1} = 1, S_1^{1}\left( \omega_1 \right) = 2, S_1^{1} (\omega_1) = \frac{1}{2}  $.
Betrachte Call $ \xi = \left( S_1^{1} \right)^{+} $. Was ist ein angemessener Preis $ P_0 (\xi) $ für $ \xi $ zur Zeit 0? \break
\underline{Ersten Ansatz}: Preis von $ \xi $ ist mittlere Auszahlung von $ \xi $ bei wiederhaltern Spiel, d.h 
$$ p_0 (\xi) = \mathbb{E} ^{ \mathbb{P} } \left[ \xi \right] = \frac{1}{2} \cdot \left( 2 - 1 \right)^{+} + \frac{1}{2} \cdot 
\left( \frac{1}{2} -1 \right)^{+} = \frac{1}{2}  $$
Dadurch ergibt sich allerdings eine Arbitragemöglichkeit.
\break 
\underline{Zeit 0}: Verkaufe $ 6 \times $ Call zum Preis $ \frac{1}{2}  $, Kaufe $ 3 \times $ die Aktie; Startkapital 0. \\
\underline{Zeit 1} Szenario $ \omega_1 $ : Endkapital ist $ 3 \cdot 2 - 6 \cdot 1 (2-1)^{+} = 0 $ \\
Szenario $ \omega_2 $: Endkapital ist $ 3 \cdot \frac{1}{2} - 6 \cdot 1 \left( \frac{1}{2} - 1 \right) ^{+} = 1,5 $ 

\end{document}
 

\end{document}
