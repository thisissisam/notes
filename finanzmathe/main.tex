\documentclass[11.5 pt, a4paper]{memoir}

\usepackage[ngerman]{babel}
\usepackage{bookmark}
\usepackage{amsmath}
\usepackage{amssymb}
\usepackage{amsthm}
\usepackage[T1]{fontenc}
\usepackage{makeidx}
\usepackage{enumitem}
\usepackage{mathtools}
\usepackage{upgreek}

\usepackage{microtype}
\usepackage{svg}
\usepackage{parskip}
\usepackage{hyperref} % Make TOC clickable

\usepackage{lipsum}
\usepackage{fancyhdr} % Heading customization
\usepackage{geometry} % Adjust page padding 
\usepackage{adjustbox}

\usepackage{xcolor}
\usepackage[most]{tcolorbox}

\definecolor{Black}{HTML}{292939}
\definecolor{CTheorem}{HTML}{FFFFFF}
\definecolor{CLemma}{HTML}{FDFFED}
\definecolor{CDefinition}{HTML}{EFF9F0}
\definecolor{DarkGray}{HTML}{6B6969}
\definecolor{ImportantBorder}{HTML}{F14747}


% Set up page layout
\geometry{
    a4paper,
    left=2cm,
    right=2cm,
    top=2cm,
    bottom=2cm
}
\linespread{1.1}
\makeindex

\pagestyle{fancy}
\fancyhf{} % Clear headre/footer
\fancyheadoffset[LE,RO]{0pt} % Adjust headsep for page number and title
\fancyhead[RE,LO]{}
\fancyhead[RE,RO]{\rightmark} % Add pape header title
\fancyhead[LE,LO]{\thepage}  % Add page number to the left
\renewcommand{\headrulewidth}{0pt} % Delete the line in the header 

% Define the \para command
\newcommand{\para}[2]{
    \clearpage % Start on a new page
    \thispagestyle{empty}% removes the top left page number and top right chapter name
    \begin{center} % Center the chapter title
        \vspace*{11em}
        \Huge   \bfseries \S #1 \quad  #2 % Chapter title with symbol and counter
    \end{center}
    \vspace{8em}
    \addcontentsline{toc}{chapter}{#1.\hspace{0.6em} #2} % Add chapter to table of contents
    \fancyhead[RE,RO]{#2} % Add pape header title
}

\newcommand{\nsec}[2]{
    \section*{\large  #1 \hspace{0.3em} #2}
    %\phantomsection
    \addcontentsline{toc}{section}{#1 \hspace{0.3em} #2}
}
\newcommand{\smalltitle}[2][]{
    %\phantomsection
    \subsubsection*{ #1 \hspace{0.2em} #2}
}
\newtcolorbox{ibox}[3][]{
	enhanced jigsaw,
	colback=#3,
	colframe=DarkGray,
	coltitle=Black,
	drop shadow,
	title={\textbf{#1\hspace{0.4em}#2}},
	%before skip = 1em,
	attach title to upper,
	after title={: \quad  },
}
\renewcommand{\d}[1]{\,\mathrm{d} #1}


\title{Finanzmathe Mitschrift vom 24.10.24}
\author{Sisam Khanal}
\date{\today}
\begin{document}
\raggedright
\color{Black}
\maketitle


\smalltitle[II.1]{Einführung des Einperiodenmodells} 
Sei $ d,p \in \mathbb{N} $. Wir betrachten einen Finanzmarkt mit $ d+1 $ Wertpapieren. Beim Einperiodenmodell gibt es genau zwei Zweitpunkt, den Anfangszeitpunkt $ 0 $ und der Endzeitpunkt $ 1 $. Es kann nur zum Zeitpunkt $ 0 $ gehandelt werden
wobei man die Preise zum Zeitpunkt $ 0 $  kennt, aber i.A noch nicht klar ist, welches von $ l $ Szenarien für die Preise zum Zeitpunkt $ 1 $ eintreten wird. \break

Preisvektor zur Zeit $ 0 $ :
$$ \overline{ S_0} = \left( S_0^0, S_0^1 , \cdots,  S_0^d \right)^T = \left( S_0^0, S_0^T \right)^T \in \mathbb{R}^{d+1}  $$
wobei $ S_0^i $ der Preis des $ i $-ten Wertpapiers zur Zeit $ 0 $ für $ i \in \left\{ 0, 1 , \cdots,  d \right\}  $ ist.\break

Sei $ \left( \Omega , \mathcal{F}, \mathbb{P}  \right)$ ein endlicher Wahrscheinlichkeitsraum mit 
\begin{align*}
	&\left| \Omega  \right| = l,  \Omega  = \left\{ \omega_1 , \cdots, \omega_l \right\}, \\
	&	\mathcal{F} = 2^{\Omega } = \mathcal{P} \left( \Omega  \right),\\
	&	\mathbb{P} : \mathcal{F} \to \left[ 0, 1 \right],\\
	&	\mathbb{P} [A] = \sum_{\omega_k \in A} \mathbb{P} \left[ \left\{ \omega_k \right\}  \right]  \text{ für  } A \subseteq \Omega,\\
	&	\mathbb{P} \left[ \left\{ \omega_k \right\}  \right]  > 0 \;  \forall k \in \left\{ 1 , \cdots, l \right\} 
\end{align*}

Zufallsvektor der Preise zur Zeit $ l $ :
$$ \overline{ S_1} = \left( S_1^0, S_1^1 , \cdots,  S_1^d\right) ^{T} = \left( S_1^0, S_1^T \right)^{T} : \Omega \to \mathbb{R}^{d+1}  $$
wobei $ S_1^i \left( \omega_k \right) \in \mathbb{R} $ der Preis der $ i $-ten Wertpapiers zur Zeit $ 1 $ unter Szenario $ k $ ist für $ i \in \left( 0,1 , \cdots,  d \right) , k \in \left( 1 , \cdots,  l \right)  $ 

Alternative können wir die Preise zur Zeit $ 1 $ als eine Preismatrix auffassen:

$$ S_1 = \left[ \;  \overline{ S_1} (\omega_1) \;  \overline{ S_1} (\omega_2) \; \cdots \;  \overline{ S_1} (\omega_l) \;  \right] =  \begin{pmatrix}
	S_1^0 (\omega_1) & S_1^0 (\omega_2) & \cdots & S_1^0 (\omega_l) \\
	S_1^1 (\omega_1) & S_1^1 (\omega_2) & \cdots & S_1^1 (\omega_l) \\
	\vdots & \vdots & & \vdots \\
	S_1^d (\omega_1) & S_1^d (\omega_2) & \cdots & S_1^d (\omega_l) 
\end{pmatrix} \in \mathbb{R}^{(d+1) \times l}
 $$
Wir nehmen an, dass das Orte Wertpapiers eine Anleihe mit fester Verzinsung $ r \geq 0 $ (Bankkonto) ist mit 

$$ S_0^0 = 1 \text{ und }  \forall \omega  \in \Omega : S_1^{0}(\omega ) = 1 + r $$

Diskontierte Preise: 
$$ X_0 = \left( X_0^1 , \cdots,  X_0^{d} \right)^{T}, X_1 = \left( X_1^1 , \cdots,  X_1^{d} \right)^{T}, \Delta X_1 = \left(\Delta X_1^1 , \cdots, \Delta X_1^{d} \right)^{T},  $$
$$ \text{ mit } X_0^i = S_0^i, X_1^i = \frac{S_1^i}{1+r} , \Delta X_1^i = X_1^{i} - X_0^i, i \in \left\{ 1 , \cdots,  d \right\}  $$
Zum Zeitpunkt $ 0 $ nählt man ein Portfolio.

\begin{ibox}[1.1]{Definition}{CDefinition}
    Eine Handelsstrategie oder ein Portfolio (im Einperiodenmodell) ist ein Vektor 
		$$ \overline{ H} = \left( H^0, H^1 , \cdots,  H^d \right)^{T} = \left( H^0, H^T \right)^T \in \mathbb{R}^{d+1}  $$
\end{ibox}
Bei einer Handelsstrategie $ \overline{ H} = \left(  H^0, H^1 , \cdots,  H^d \right)^{T}  $ beschreibt $ H^{i} \in \mathbb{R} $ die Stückzahl von Wertpapier $ i  $ in Portfolio Zwischen den beiden Zeitpunkten (für $ i \in \left\{ 0, 1 , \cdots,  d \right\}  $. Dabei ist $ H^{i} $ nicht zufällig. \\

Falls $ H^0 < 0 $ : Kreditaufnahme, \\
Falls $ H^{i} < 0 $ für ein $ i \in \left\{ 1 , \cdots,  d \right\}  $ : Leerverkauf (short sell)

\begin{ibox}[1.2]{Definition}{CDefinition}
    Sei $ \overline{ H} \in \mathbb{R}^{d+1}  $. Der Wert der Handelsstrategie $ \overline{ H}  $ ist zur Zeit $ 0 $ durch:
		$$ V_0^{ \overline{ H} } = \left< \overline{ S_0}, \overline{ H}   \right> = \overline{ S_0^T} \overline{ H} = \sum_{i=0}^{d} S_0^i H^i   $$
	Und zur Zeit $ 1 $ durch die Zufallsvariable $ V_1^{ \overline{ H} } : \Omega \to \mathbb{R} $,
	$$ V_1^{ \overline{ H} } = \left< \overline{ S_1} (\omega ), \overline{ H}   \right> =  \sum_{i=0}^{d} S_0^i (\omega ) H^i   $$
	definiert. Weiter definieren wir die diskontierten Werte als 
	$$ D_0^{ \overline{ H} } = V_0^{ \overline{ H} } \text{ und }  D_1^{ \overline{ H} } = \frac{V_1^{ \overline{ H} }}{1 +r} $$
\end{ibox}

\begin{ibox}[1.3]{Lemma}{CLemma}
    Sei $ \overline{ H} = \left( H^0, H^T \right)^T \in \mathbb{R}^{d+1}  $ eine Handelsstrategie. Dann gilt 
		$$ D_1^{ \overline{ H} } = D_0 ^{ \overline{ H} } + \left< \Delta X_1, H \right> $$
\end{ibox}
Beweis: Übung


\smalltitle[]{Beispiel 1.4}
Seien $ d =1 , l=1, p \in (0,1), \mathbb{P} \left[ \left\{ \omega_1 \right\}  \right] = p, \mathbb{P} \left[ \left\{ \omega_2 \right\}  \right] = 1-p, S_0^{1} = 1, S_1^1 (\omega_1) = 2, S_1^1 (\omega_2) = \frac{1}{2}  $
\break
Sei $ r=2 $ und wähle die Handelsstrategie $ \overline{ H} (1,-1)  $ Dann ist,

\begin{align*}
	V_0^{ \overline{ H} } &= 1 \cdot 1 + 1 \cdot (-1) = 0, \\
	V_1^{ \overline{ H} } (\omega_1) &= (1+r) \cdot 1 + S_1^1 (\omega_1) \cdot (-1) = 1, \\
	V_1^{ \overline{ H} } (\omega_2) &= (1+r) \cdot 1 + S_1^1 (\omega_2) \cdot (-1) = \frac{5}{2}
\end{align*}



\smalltitle[II.2]{No-Arbitrage und FTAP1}
\begin{itemize}
	\item Ziel: Charakterirrung vom Markt, in dem es keine Arbitrage-Möglichkeiten gibt.
	\item Hilfsmittel: äquivalente Martingallmaße
	\item Hauptresultat: First Fundamental theorem of asset pricing (FTAP1)
	\item Praktische Umsetzung: prüfe Gleichungssystem auf Lösbarkeit (unter Nebenbedingungen)
\end{itemize}

\begin{ibox}[2.1]{Definition}{CDefinition}
	Eine Handelsstrategie $ \overline{H} \in \mathbb{R}^{d+1} $ heißt \underline{Arbitragemöglichkeit} falls gelte:
	\begin{enumerate}[label=\alph*)]
		\item $ V^{ \overline{ H} }_0 \leq 0$ ,
		\item $ \forall \omega \in \Omega : V^{ \overline{H } }_{1}(\omega) \geq 0 $ und
		\item $ \exists \omega \in \Omega :  V^{ \overline{H } } _{1}(\omega) > 0  $ 
	\end{enumerate}
\end{ibox}
Wir sagen, es gilt \underline{No-Arbitrage (NA)}, falls keine Arbitragemöglichkeit existieren. 
\begin{ibox}[2.2]{Lemma}{CLemma}
    Folgende Aussagen sind äquivalent
		\begin{enumerate}[label=\alph*)]
			\item Es gibt eine Arbitragemöglichkeit.
			\item Es gibt ein $ \alpha \in \mathbb{R}^{d} $, sodass 
				\begin{itemize}
					\item $ \forall \omega \in \Omega : \left<\Delta X_1(\omega), \alpha \right> \geq 0 $  und 
					\item $ \exists \omega \in \Omega :  \left<\Delta X_1(\omega), \alpha \right> > 0  $ 
				\end{itemize}
			\item Es gibt eine Arbitragemöglichkeit $ \overline{ H} \in \mathbb{R}^{d+1}  $ mit $ V^{ \overline{ H} }_0 = 0 $ 	
		\end{enumerate}
\end{ibox}
\subsection*{Beweis} Übung
Eine Wahrscheinlichkeitsmaß $ Q $ auf $ \left( \Omega, \mathcal{F} \right)  $ können wir durch den Vektoren $ q \in \mathbb{R}^{l} $ 
mit $ q_k = Q \left[ \left\{ \omega_k \right\}  \right] , k \in \left\{ 1 , \cdots,  l \right\}  $ charakterisieren. Für eine Zufallsvariable $ Y : \Omega \to \mathbb{R} $ notieren wir mit 
$$ \mathbb{E}^{Q} \left[ Y \right] = \sum_{k=1}^{l} q_k Y \left( \omega_{k} \right)  $$
den Erwartungswert von $ Y $ bzgl. $ Q $. Für $ m \in \mathbb{N} $ und einen $ m $-dim Zufallsvektor $ Y = \left( Y^{1}, , \cdots,  
Y^{m}\right)^{T}: \Omega to \mathbb{R}^{m} $ setzen wir 
$$ \mathbb{E}^{Q} \left[ Y \right] = \left( \mathbb{E}^{Q} \left[ Y^{1} \right], \cdots,  \mathbb{E}^{Q} \left[ Y^{m} \right]  \right) ^{T}  $$
\begin{ibox}[2.3]{Definition}{CDefinition}
	Ein Wahrscheinlichkeitsmaß $ Q $  auf $ \left( \Omega, \mathcal{F} \right) $ heißt \underline{risikoneutral} oder 
	\underline{Martingalmaße}, falls $ \forall i \in \left\{ 1 , \cdots,  d \right\}  $ :
	$$ \mathbb{E}^{Q} \left[ X_1^{i} \right] = X_{0}^{i}$$
\end{ibox}

\begin{ibox}[2.4]{Lemma}{CLemma}
    Sei $ Q $ ein Wahrscheinlichkeitsmaß auf $ \left\{ \Omega, \mathcal{F} \right\}  $ und $ q_k = Q \left[ \left\{ \omega_k \right\}  \right] , k \in \left\{ 1 , \cdots,  l \right\}  $. Es bezeichne $ e_i $ den $ i$-ten Einheitsvektor im $ \mathbb{R}^{d} $. Dann sind äquivalent:
		\begin{enumerate}[label=\alph*)]
			\item $ Q $ ist eine Martingalmaße
			\item $ \forall i \in \left\{ 1 , \cdots,  d \right\} : \sum_{k=1}^{l} q_k \left< \Delta X_1 (\omega_k), e_i \right>  = 0$ 
			\item $ \forall \alpha \in \mathbb{R}^{d} : \sum_{k=1}^{l} q_k \left< \Delta X_1 (\omega_k), \alpha \right>  = 0$ 
		\end{enumerate}
\end{ibox}
\subsection*{Beweis} Übung
\begin{ibox}[2.5]{Definition}{CDefinition}
	Ein Wahrscheinlichkeitsmaß $ Q $ auf $ \left( \Omega, \mathcal{F} \right)  $ heißt (zu $ \mathbb{P} $) \underline{äquivalentes Wahrscheinlichkeitsmaß}, wenn $ \forall k \in \left\{ 1 , \cdots,  l \right\} : q_k = Q \left[ \left\{ \omega_k \right\}  \right] > 0  $  
\end{ibox}
Wir hatten angenommen, dass $ \forall k \in \left\{ 1 , \cdots,  l \right\} : \mathbb{P} \left[ \left\{ \omega_k \right\}  \right] > 0 $  ein $ \mathbb{P} $ äquivalentes Wahrscheinlichkeitsmaß auf $ \left\{ \Omega, \mathcal{F} \right\}  $ ist, dann stimmen also die Mengen der möglichen Szenarien unter $ \mathbb{P} \text{ und }  Q $ überein (aber die genauen Wahrscheinlichkeiten sind in der Regel unterschiedlich)

\begin{ibox}[2.6]{Definition}{CDefinition}
	Ein Wahrscheinlichkeitsmaß $ Q $ auf $ \left\{ \Omega, \mathcal{F} \right\}  $ heißt \underline{äquivalentes Martingalmaße (ÄMM)}
	falls $ Q $ risikoneutral und äquivalent ist. Wir definieren 
	$$ \mathcal{P} = \left\{ Q : Q \text{ ist ÄMM }  \right\}  $$
	
\end{ibox}

\subsection*{Beispiel 2.7:} Seien $ d = 1, l = 2, p \in (0,1), \mathbb{P} \left[ \left\{ \omega_1 \right\}  \right] = p, \mathbb{P} \left[ \left\{ \omega_2 \right\}  \right] = 1 - p, S_0^{1} = 1, S_1^{1}(\omega_1) = 2, S_1^{1}(\omega_2)= \frac{1}{2} \text{ und }  r = 0  $. Wir versuchen, ein ÄMM zu finden. Wenn $ Q $ ein ÄMM (charakterisiert durch $ q \in \mathbb{R}_2 $ ist, dann müssen gelten:  
$ q_1 > 0, q_2 >0 $ 
 $$ \sum_{k=1}^{2} q_k S_1^{1} (\omega_k) = q_1 \underbrace{ S_1^{1} (\omega_1)}_{=2} + q_2 \underbrace{ S_1^{1} (\omega_2)}_{=\frac{1}{2} } = 1 $$
LGS lösen: 
\begin{align*}
	q_1 + q_2 &= 1 \\
	 2q_1 + \frac{1}{2} q_2 &= 1 
\end{align*}

Andere Schreibweise: 
\begin{align*}
\begin{pmatrix}
	1 & 1 \\
	2 & \frac{1}{2} 
\end{pmatrix} 
\begin{pmatrix}
	q_1 \\ q_2 
\end{pmatrix}
 = \begin{pmatrix}
	 1 \\ 1  
 \end{pmatrix} \implies \begin{pmatrix}
 	q_1 \\ q_2 
 \end{pmatrix}
 = \begin{pmatrix}
 	 \frac{2}{3} \\ \frac{1}{2} 
 \end{pmatrix}
\end{align*}
Im ersten fundamentalsatz wird ein Zusammenhang zwischen der Existenz von ÄMMs und NA hergestellt:

\begin{ibox}[2.8]{Satz (FTAP1)}{CTheorem}
	Es gilt: $ NA \iff \mathcal{P} \neq \emptyset $.
	Andereseits gilt: Ein Vektor $ q \in \mathbb{R}^{l} $ definiert genau dann ein ÄMM (via $ Q \left[ \left\{ \omega_k \right\}  \right] : = q_k, k \in \left\{ 1, , \cdots,  l \right\}  $ , wenn $ k \in \left\{  1, \cdots,  l \right\} : q_k > 0 $ gilt, und $ q $ 
	eine Lösung des Systems: $ \forall i \in\left\{ 1 , \cdots,  d \right\}  $ 

	\begin{align*}
		q_1 + q_2 + \cdots + q_{l} &= 1 \\
		\frac{1}{1+r} \left( S_1^{i} (\omega_1) q_1 +   S_1^{i} (\omega_2) q_2  + \cdots +   S_1^{i} (\omega_l) q_l \right) &= S_0^{i}
	\end{align*}
ist. Um zu überprüfung, ab NA gilt, kann man wegen FTAP1 also testen, ob 
$$ S_1 q = (1+r) \overline{ S_0}  $$
eine Lösung $ q \in \mathbb{R}^{l} $ mit $ \forall k \in \left\{ 1 , \cdots,  l \right\} : q_k > 0 $ besitzt
\end{ibox}
Für den Beweis des FTAP1 benötigen wir folgenden Trennungssatz (in $ \mathbb{R}^{l} $ )

\begin{ibox}[2.9]{Satz (Trennungssatz)}{CTheorem}
	Sei $ n \in \mathbb{N} $ 
	\begin{enumerate}[label=\alph*)]
		\item Sei $ C \subseteq \mathbb{R}^n $ abgeschlossen, konvex und nichtleer mit $ 0 \notin C $. Dann existieren $ y \in C $ und 
			$ \delta > 0 $ sodass $ \forall x \in C : \left<x,y \right> \geq \delta > 0 $ 
		\item Sei $ K \subseteq \mathbb{R}^n $ kompakt, konvex und nicht leer und sei $ U \subseteq \mathbb{R}^n $ eine linearen Unterraum
			mit $ K \cap U = \emptyset $. Dann existiert $ y \in \mathbb{R}^n $ sodass
			\begin{itemize}
				\item $ \forall x \in K : \left<x, y \right> >0$,
				\item $ \forall x \in U : \left<x,y \right> = 0 $ 
			\end{itemize}
	\end{enumerate}
\end{ibox}

\smalltitle[2.10]{Proposition}
Die Menge $ \mathcal{P} $ aller ÄMMs ist konvex. Insbesondere gilt:
$$ \left| \mathcal{P} \right| \implies | \mathcal{P}| = \infty $$
Beweis: Übung

\smalltitle[II.3]{Arbitragefreie Preise}
Wir Identifizieren im Folgenden ein Derivat mit seiner Auszahlung zur Zeit $ T=1 $ 

\begin{ibox}[3.1]{Definition}{CDefinition}
    Ein \textit{Derivat} ist eine Zufallsvariable $ \xi : \Omega \to \mathbb{R} $. Für $ x, y \in \mathbb{R} $ notieren wir 
		$$ \left( x - y \right)^{+} = \text{ max } \left\{ x - y, 0 \right\}  $$
\end{ibox}

\smalltitle[3.2]{Beispiel}
Beispiele für Derivate:                                                                                              
\begin{enumerate}[label=\alph*)]
	\item \underline{Forward contract}: Vereinbarung, zu einem zukünftigen, festgelegten Termin $ T $ ein Gut (z.B Wertpapier) zu einem heute vereinbarten Preis $ k $ zu kaufen bzw. zu verkaufen. Im Einperiodenmodell: Forward auf Papier
		$$ i: \xi = S_1^{i} - k  $$
	\item \underline{Call-Option}: Recht, ein Gut zu festgelegt Preis $ K $ ( dem \textit{Strike} ) zu zukünftigen Zeitpunkt $T$ zu kaufen.
		Im Einperiodenmodell: Call auf Papier 
		$$ i : \xi = \left( S_1^{i} - k \right) ^{+} $$
	\item \underline{Put-Option}: Analog zu Call, aber Verkaufen. Im Einperiodenmodell: Put auf Papier
		$$ i : \xi = \left\{ k - S_1^{i} \right\} ^{+} $$
	\item \underline{Basket-Option}: Option auf einen Index von Wertpapier z.B Basket-Call-Option mit Strike $ k $ und Gewicht
		$ \alpha \in \mathbb{R}^{d} $ im Einperiodenmodell :
		$$ i : \xi = \left( K - S_1^{i} \right)  $$
	\item \underline{Basket-Option}: Option auf einen Index von Wertpapier z.B Basket-Call-Option mit Strike $ k $ und Gewicht $ \alpha
		\in \mathbb{R}^{d}$ im Einperiodenmodell :
		$$ \left(  \left< S_1, \alpha \right> - k \right) ^{+} $$
\end{enumerate}
\smalltitle[3.3]{Beispiel}
Seien $ d = 1, l =2, \mathbb{P} \left[ \left\{ \omega_1 \right\}  \right] = \frac{1}{2} = \mathbb{P} \left[ \left\{ \omega_1 \right\}  \right] , S_0^{0} = 1, S_1^{0} = 1 (r=0), S_0^{1} = 1, S_1^{1}\left( \omega_1 \right) = 2, S_1^{1} (\omega_1) = \frac{1}{2}  $.
Betrachte Call $ \xi = \left( S_1^{1} \right)^{+} $. Was ist ein angemessener Preis $ P_0 (\xi) $ für $ \xi $ zur Zeit 0? \break
\underline{Ersten Ansatz}: Preis von $ \xi $ ist mittlere Auszahlung von $ \xi $ bei wiederhaltern Spiel, d.h 
$$ p_0 (\xi) = \mathbb{E} ^{ \mathbb{P} } \left[ \xi \right] = \frac{1}{2} \cdot \left( 2 - 1 \right)^{+} + \frac{1}{2} \cdot 
\left( \frac{1}{2} -1 \right)^{+} = \frac{1}{2}  $$
Dadurch ergibt sich allerdings eine Arbitragemöglichkeit.
\break 
\underline{Zeit 0}: Verkaufe $ 6 \times $ Call zum Preis $ \frac{1}{2}  $, Kaufe $ 3 \times $ die Aktie; Startkapital 0. \\
\underline{Zeit 1} Szenario $ \omega_1 $ : Endkapital ist $ 3 \cdot 2 - 6 \cdot 1 (2-1)^{+} = 0 $ \\
Szenario $ \omega_2 $: Endkapital ist $ 3 \cdot \frac{1}{2} - 6 \cdot 1 \left( \frac{1}{2} - 1 \right) ^{+} = 1,5 $ 

\end{document}
