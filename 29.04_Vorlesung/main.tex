\documentclass[a4paper]{memoir}


\usepackage[ngerman]{babel}
\usepackage{bookmark}
\usepackage{amsmath}
\usepackage{amssymb}
\usepackage{amsthm}
\usepackage[T1]{fontenc}
\usepackage{makeidx}
\usepackage{enumitem}
\usepackage{mathtools}
\usepackage{upgreek}

\usepackage{microtype}
\usepackage{svg}
\usepackage{parskip}
\usepackage{hyperref} % Make TOC clickable

\usepackage{lipsum}
\usepackage{fancyhdr} % Heading customization
\usepackage{geometry} % Adjust page padding 
\usepackage{adjustbox}

\usepackage{xcolor}
\usepackage[most]{tcolorbox}

\definecolor{Black}{HTML}{292939}
\definecolor{CTheorem}{HTML}{FFFFFF}
\definecolor{CLemma}{HTML}{FDFFED}
\definecolor{CDefinition}{HTML}{EFF9F0}
\definecolor{DarkGray}{HTML}{6B6969}
\definecolor{ImportantBorder}{HTML}{F14747}


% Set up page layout
\geometry{
    a4paper,
    left=2cm,
    right=2cm,
    top=2cm,
    bottom=2cm
}
\linespread{1.1}
\makeindex

\pagestyle{fancy}
\fancyhf{} % Clear headre/footer
\fancyheadoffset[LE,RO]{0pt} % Adjust headsep for page number and title
\fancyhead[RE,LO]{}
\fancyhead[RE,RO]{\rightmark} % Add pape header title
\fancyhead[LE,LO]{\thepage}  % Add page number to the left
\renewcommand{\headrulewidth}{0pt} % Delete the line in the header 

% Define the \para command
\newcommand{\para}[2]{
    \clearpage % Start on a new page
    \thispagestyle{empty}% removes the top left page number and top right chapter name
    \begin{center} % Center the chapter title
        \vspace*{11em}
        \Huge   \bfseries \S #1 \quad  #2 % Chapter title with symbol and counter
    \end{center}
    \vspace{8em}
    \addcontentsline{toc}{chapter}{#1.\hspace{0.6em} #2} % Add chapter to table of contents
    \fancyhead[RE,RO]{#2} % Add pape header title
}

\newcommand{\nsec}[2]{
    \section*{\large  #1 \hspace{0.3em} #2}
    %\phantomsection
    \addcontentsline{toc}{section}{#1 \hspace{0.3em} #2}
}
\newcommand{\smalltitle}[2][]{
    %\phantomsection
    \subsubsection*{ #1 \hspace{0.2em} #2}
}
\newtcolorbox{ibox}[3][]{
	enhanced jigsaw,
	colback=#3,
	colframe=DarkGray,
	coltitle=Black,
	drop shadow,
	title={\textbf{#1\hspace{0.4em}#2}},
	%before skip = 1em,
	attach title to upper,
	after title={: \quad  },
}
\renewcommand{\d}[1]{\,\mathrm{d} #1}

\begin{document}
\pagecolor{White}
\color{Black}
\begin{ibox}[5.2]{Korrespondenzsatz}{CTheorem}
    Sei $ \varphi : G \to H $ surjektive Grphomo, $ U \leq G $ mit $ ker
	\varphi \leq  U j $ und $ V \leq  H $. Dann 
	\begin{enumerate}[label=\alph*)]
		\item $ \varphi (U) = V \iff \varphi^{-1}(V) = U $ 
		\item Gilt $ \varphi = V, dann: U \triangleleft G \iff V 
			\triangleleft H
 $ 	\end{enumerate}
	
\end{ibox}

\para{4}{Klassifikation der endlichen abelschen Gruppen}

\begin{ibox}[5.1]{Satz}{CTheorem}
    Ist $ G $ eine endliche abelsche Gruppe, so ist $ G $ isomorph zu 
	$ \mathbb{Z} \ n_1 \mathbb{Z} \times \mathbb{Z}  \ n_2 \mathbb{Z} 
	\cdots \mathbb{Z} \ n \mathbb{Z} \text{ , wobei  } \left| G \right| 
	= n_1 \cdots n_{r}$  und $ n_{i}| n_{i-1} $  
\end{ibox}
\begin{ibox}[5.3]{Lemma}{CLemma}
    Sei $ G $ eindlich und abelsch und $ p $ Primzahl mit $ p | 
	\left| G \right| $. Dann exist $ g \in G $ mit $ o(g) = p $ 
\end{ibox}

\begin{proof}
	(Induktion nach Anzahl von Teilen von $ \left| G \right|  $ 
	\textit{Induktionsanfang} : $ \left| G \right|  = p \leadsto G \cong
	\mathbb{Z} \ m \mathbb{Z} $\\
	\textit{Induktionsschritt} : Sei $ H $ max $ UG/NT $ von $ G $. Dann
	gilt $ \left| G/H \right|  = p' $ für eine Primzahl $ p' $ 
	gilt \\ $ p | \left| H \right|  $ , so exist $ g \in  H $ nach 
	Induktionvoraussetzung $ \implies $ \\ Behauptung 
	sonst $  p = \left| G/H \right|  $ , da $ G| = |H| 
	\left| G/H \right|  $ 


\end{proof}

\begin{ibox}[4.5]{Lemma}{CLemma}
    Ist $ G $ eine abelsche $ (p-) $ Gruppe mit einer einzigen UG N der 
	Ordnung $ p $, so ist $ G $ zykelisch. 
\end{ibox}

\begin{proof}
	Beweis nach $ \left| G \right|  $ . Die Abbildung $ f : G \to G $ 
	mit $ g \mapsto g^{p} $ ist eine Gruppenhomomorphismus. \\
	$ ker \; f $ besteht aus $ 1 $ und den Elementen der Ordnung $ p $ .
\end{proof}

\end{document}
